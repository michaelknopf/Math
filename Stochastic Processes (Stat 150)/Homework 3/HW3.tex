\documentclass[12pt]{article}
 
\usepackage[margin=.75in]{geometry} 
\usepackage{amsmath,amsthm,amssymb}
\usepackage{units}
\usepackage{bm}
\usepackage{enumitem}
\setenumerate{listparindent=\parindent}
 
\newcommand{\N}{\mathbb{N}}
\newcommand{\Z}{\mathbb{Z}}
\newcommand{\p}{\mathbb{P}}
\newcommand{\E}{\mathbb{E}}

\theoremstyle{definition}
\newtheorem*{prob1}{Q 1}
\newtheorem*{prob2}{Q 2}
\newtheorem*{prob3}{Q 3}
\newtheorem*{prob4}{Q 4}
\newtheorem*{prob5}{Q 5}
\newtheorem*{prob6}{Q 6}

 
\begin{document}

\title{Homework 3 \\ Stat 150}
\author{Michael Knopf \\ 24457981}
\maketitle

\begin{prob1}
On a game show there are two contestants. A contestant answers a series of questions until they make a mistake and then it becomes the other contestants turn. Contestant one answers questions correctly 70\% of the time while contestant two answers correctly 80\% of the time. Over the long run what proportion of questions are asked of contestant 1?
\end{prob1}

\begin{proof}

Let $X_n = 1$ if contestant 1 is asked question $n$, and 2 if contestant 2 is asked question $n$.  The transition matrix for $X_n$ is
$$P = \left( \begin{matrix}
.7 & .3 \\
.2 & .8
\end{matrix}
\right)$$.

Solve for the stationary distribution.  Let $\pi = (x, y)$.  Then $\pi P = \pi$ implies
$$

$$

\end{proof}

\end{document}



















