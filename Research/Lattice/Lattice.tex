\documentclass[10pt]{article}
\usepackage[margin=1in]{geometry}
\addtolength{\oddsidemargin}{-.2in} 
\usepackage{amsmath,amsthm,amssymb}
\usepackage{bm}
\usepackage{enumitem}
\usepackage{array}
\usepackage{lipsum}
\usepackage[]{units}
\usepackage{relsize}
\usepackage{verbatim}
\usepackage{bbm}

\usepackage{tikz}
\usetikzlibrary  {positioning}
\usepackage{graphicx}
\usepackage{xfrac}
\usetikzlibrary{cd}


\setenumerate{listparindent=\parindent}

\newcommand{\M}{\mathcal{M}}
\newcommand{\thefield}{\Q(\sqrt{-D})}
\newcommand{\thering}{\mathcal{O}_{\Q(\sqrt{-D})}}
\newcommand{\Ord}{\mathcal{O}}
\newcommand{\norm}[1]{\left[#1\right]}

\newcommand{\Q}{\mathbf{Q}}
\newcommand{\Z}{\mathbf{Z}}
\newcommand{\R}{\mathbf{R}}
\newcommand{\C}{\mathbf{C}}
\newcommand{\p}{\mathfrak{p}}
\renewcommand{\a}{\mathfrak{a}}
\renewcommand{\b}{\mathfrak{b}}
\newcommand{\m}{\mathfrak{m}}
\newcommand{\gen}[1]{\langle #1 \rangle}
\DeclareMathOperator*{\dom}{dom}
\DeclareMathOperator*{\Aut}{Aut}
\DeclareMathOperator*{\Ann}{Ann}
\DeclareMathOperator*{\Tor}{Tor}
\DeclareMathOperator*{\Gal}{Gal}
\DeclareMathOperator*{\Hom}{Hom}
\DeclareMathOperator*{\End}{End}
\DeclareMathOperator*{\im}{Im}
\let\ker\relax
\DeclareMathOperator*{\ker}{Ker}
\DeclareMathOperator*{\spn}{span}
\DeclareMathOperator*{\Perm}{Perm}
\DeclareMathOperator*{\card}{card}
\DeclareMathOperator*{\Alt}{Alt}
\DeclareMathOperator*{\id}{id}
\renewcommand{\bar}{\overline}

\newtheorem*{thm}{Theorem}
\newtheorem*{prop}{Proposition}

\usepackage{fancyhdr} % Required for custom headers 
%\usepackage{lastpage} % Required to determine the last page for the footer

\pagestyle{fancy}
\lhead{Math 250A (HW 6)}
\chead{Michael Knopf (24457981)}
\rhead{October $8^\text{th}$, 2015}
\lfoot{}
\cfoot{}
\rfoot{}
%\rfoot{Page\ \thepage\ of\ \pageref{LastPage}}
\renewcommand\headrulewidth{0.4pt}
%\renewcommand\footrulewidth{0.4pt}

\begin{document}

\begin{thm}
\normalfont
If $A$ be an additive subgroup of Euclidean space $\R^n$ such that every bounded region of space contains only finitely many elements of $A$, then $A$ is a lattice of dimension $\leq n$.
\end{thm}

\begin{proof}
Let $\{v_1, \dots , v_m\}$ be a maximal $\R$-linearly independent set of elements from $A$ (if $A=\{0\}$, take the empty set; otherwise, add linearly independent vectors until no new elements from $A$ can be added).  Let $m$ be the maximum possible size of such a set, since any such set has size $\leq n$.  We will prove the statement by induction on $m$.  Clearly, if $m = 0$ then $A = \{0\}$, hence is a lattice of dimension $0$.

Let $A_0 = A \cap \spn \{v_1, \dots , v_{m-1}\}$.  Then $\{v_1, \dots , v_{m-1}\}$ is a maximal linearly independent subset of $A_0$, so by induction $A_0$ is a lattice with some basis $\{u_1, \dots , u_k\}$, where $k \leq m-1$.  However, $\{v_1, \dots , v_{m-1}\} \subseteq A_0$, thus $\{v_1, \dots , v_{m-1}\}$ is in the vector space spanned by $\{u_1, \dots , u_k\}$ over $\R$.  Since $\{v_1, \dots , v_{m-1}\}$ is linearly independent over $\R$, this means that $k \geq m-1$.  Therefore, we know $k = m-1$.


%However, the vector space spanned by $\{u_1, \dots , u_k\}$ over $\R$ contains $\{v_1, \dots , v_{m-1}\}$, thus we must have $m-1 \leq k$ as well.  So $k = m-1$.

Let $S = A \cap \{a_1u_1 + \cdots + a_{m-1}u_{m-1} + a_mv_m \mid 0 \leq a_i < 1 \text{ for } 1 \leq i \leq m-1, 0 \leq a_m \leq 1 \}$.  By the triangle inequality, $S$ is contained within a ball of radius $|u_1| + \cdots + |u_{m-1}| + |v_m|$ about the origin, hence is finite.  Now, $\{u_1, \dots , u_{m-1}, v_m\}$ must be linearly independent, since if $v_m$ were in the span of the $u_i$s, then $\spn\{u_1, \dots , u_{m-1}\}$ would contain $\spn\{v_1, \dots , v_m\}$, contradicting that this latter set is linearly independent.  So every element of $S$ has a unique representation of the form
$$
a_1u_1 + \cdots + a_{m-1}u_{m-1} + a_mv_m$$
with $0 \leq a_i < 1 \text{ for } 1 \leq i \leq m-1$ and $0 \leq a_m \leq 1$.  So there is some $v_m' \in S$ which has a minimal but nonzero coefficient $a_m$ when expanded as
$$
v_m' = a_1u_1 + \cdots + a_{m-1}u_{m-1} + a_mv_m.
$$
We know this because these expansions are unique and $S$ is finite.

Replacing $v_m$ with $v_m'$, we now see that $\{u_1, \dots , u_{m-1}, v_m'\}$ is still linearly independent because $\{u_1, \dots , u_{m-1}\}$ is linearly independent and, due to the uniqueness of the representations we just discussed, $v_m'$ is not a linear combination of $\{u_1, \dots , u_{m-1}\}$.  Also, $\{u_1, \dots , u_{m-1}, v_m'\}$ spans $A$ over $\R$: if there were some $v \in A \setminus \spn (\{u_1, \dots , u_{m-1}, v_m'\})$ then $v$ would be linearly independent of $\{u_1, \dots , u_{m-1}, v_m'\}$, meaning that $\{u_1, \dots , u_{m-1} , v_m', v\}$ is a linearly independent set, contradicting that $m$ is the largest possible size of a linearly independent set in $A$.

Let $v \in A$.  Then $v$ can be expressed uniquely as a linear combination $v = b_1u_1 + \cdots + b_{m-1}u_{m-1} + b_mv_m'$.  Letting $v_m' = a_1u_1 + \cdots + a_{m-1}u_{m-1} + a_mv_m$ be the expansion of $v_m'$ given previously, we have
\begin{align*}
v &= b_1u_1 + \cdots + b_{m-1}u_{m-1} + b_mv_m' \\
&= b_1u_1 + \cdots + b_{m-1}u_{m-1} + b_m(a_1u_1 + \cdots + a_{m-1}u_{m-1} + a_mv_m) \\
&= (b_1 + b_ma_1)u_1 + \cdots + (b_{m-1} + b_ma_{m-1})u_{m-1} + (b_ma_m)v_m.
\end{align*}
Let $c_m = \lfloor b_m \rfloor$.  Note that the coefficient of $v_m$ in $v - c_m v_m'$ is $(b_m - c_m)a_m$, which satisfies $0 \leq (b_m - c_m)a_m < a_m$ since $0 \leq b_m - c_m < 1$.  Next, for each $i = 1, \dots , m-1$ let $c_i$ be the floor of the coefficient of $u_i$ in $v - c_m v_m'$.  Let
$$
v' = v - c_1u_1 - \cdots - c_{m-1}u_{m-1} - c_mv_m'
$$
Each $u_i$ is in $A$, $v_m'$ is in $A$, and each $c_i$ is an integer.  So $v'$ is a $\Z$-linear combination of elements in $A$, hence $v' \in A$.  Furthermore, the coeffecients of $u_1, \dots , u_{m-1}, v_m$ in $v'$ are all less than 1 and at least 0 by construction; therefore, $v' \in S$.  The coefficient of $v_m$ in $v'$ is the same as the coefficient of $v_m$ in $v$, which we previously noted is strictly less than $a_m$.  By the minimality of $a_m$ (recall how $a_m$ was defined), we realize that this coefficient must be $0$.  Therefore, $v'$ is a $\Z$-linear combination of $\{u_1, \dots , u_{m-1}\}$.  But also, $w' = c_1u_1 + \cdots + c_{m-1}u_{m-1} + c_m v_m'$ is in the span of $\{u_1, \dots , u_{m-1}, v_m'\}$ over $\Z$.  Thus, $v = v' + w$ is in the span of $ \spn\{u_1, \dots , u_{m-1}, v_m'\}$ over $\Z$, and so this set generates $A$.  We have already shown this set to be linearly independent over $\R$, and that $m \leq n$.  Therefore, $A$ is a lattice of dimension $\leq n$.
\end{proof}

\pagebreak

\begin{prop}
\normalfont
Let $\M \subseteq \R^n$ be such that $0 \in \M$ and $\norm{\alpha - \beta} \in \Z$ for all $\alpha, \beta \in \M$.  Then the additive group $\Z[\M]$ generated by $\M$ also satisfies $\norm{\alpha - \beta} \in \Z$ for all $\alpha,\beta \in \Z[\M]$, and contains finitely many points in any bounded region of $\R^n$.  Therefore, $\Z[\M]$ is a lattice in $\R^n$.
\end{prop}

\begin{proof}
Let $\alpha \in \Z[\M]$, so that $\alpha = \sum_{t=1}^m \alpha_t$ for some $\alpha_1, \dots , \alpha_m \in \M$.  We have
$$
\norm{\alpha} = \norm{ \sum_{t=1}^m \alpha_t } = \norm{\sum_{t=1}^m \alpha_t, \sum_{t=1}^m \alpha_t} = \sum_{s=1}^m \sum_{t=1}^m \norm{\alpha_s,\alpha_t} = 
\sum_{t=1}^m \norm{\alpha_t} +
\sum_{1 \leq s < t \leq m} 2\norm{\alpha_s,\alpha_t}.
$$
Since $\norm{\alpha_s} = \norm{\alpha_s - 0} \in \Z$, and $\norm{\alpha_s - \alpha_t} = \norm{\alpha_s} + \norm{\alpha_t} - 2\norm{\alpha_s,\alpha_t} \in \Z$ for all $s,t \in \Z$, we know $2 \norm{\alpha_s,\alpha_t} \in \Z$ for all $s,t \in \Z$.  Therefore, $\norm{\alpha} \in \Z$.  Since $\Z[\M]$ is a subgroup, we know that $\alpha - \beta \in \Z[\M]$ for any $\beta \in \Z[\M]$, thus $\norm{\alpha - \beta} \in \Z$ as well.

Let $R$ be any bounded region of $\R^n$.  We aim to show that $\Z[\M] \cap R$ is finite.  We may assume $R$ is closed, since $R$ is certainly contained within its closure and hence so is $\Z[\M] \cap R$.  Let $\mathcal{C}$ be the set of all open balls in $\R^n$ of radius $\frac12$.  $\mathcal{C}$ is an open cover of the compact set $R$, hence it has a finite subcover $\mathcal{C}' \subset \mathcal{C}$ containing $N \in \Z_{>0}$ elements.  If $B \in \mathcal{C}'$, then $B$ may contain at most one point of $\Z[\M]$, since we have $\norm{\alpha - \beta} \in \Z$, and thus $|\alpha - \beta| \geq 1 > \frac12$, for any distinct $\alpha, \beta \in \Z[\M]$.  Therefore, $\Z[\M] \cap R$ contains at most $N$ points.  By the previous theorem, we see that $\Z[\M]$ must be a lattice in $\R^n$.
\end{proof}
\end{document}