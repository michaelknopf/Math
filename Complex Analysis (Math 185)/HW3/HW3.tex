% Insert HW number and date into heading!

\documentclass[10pt]{article}
\usepackage[margin=1in]{geometry}
%\addtolength{\oddsidemargin}{-.1in} 
\usepackage{amsmath,amsthm,amssymb}
\usepackage{bm}
\usepackage{enumitem}
\usepackage{array}
\usepackage{lipsum}
\usepackage[]{units}
\usepackage{relsize}
\usepackage{verbatim}

\usepackage{tikz}
\usetikzlibrary{positioning}
\usepackage{graphicx}
\usepackage{xfrac}

\setenumerate{listparindent=\parindent}

\newcommand{\N}{\mathbb{N}}
\newcommand{\Z}{\mathbb{Z}}
\newcommand{\Q}{\mathbb{Q}}
\newcommand{\R}{\mathbb{R}}
\newcommand{\C}{\mathbb{C}}
\newcommand{\D}{\mathbb{D}}

\DeclareMathOperator*{\dom}{dom}
\DeclareMathOperator*{\re}{Re}
\DeclareMathOperator*{\im}{Im}
\DeclareMathOperator*{\Log}{Log}
%\DeclareMathOperator*{\arg}{arg}
\renewcommand{\bar}{\overline}

\definecolor{mygray}{rgb}{.8,.8,0.8}

\newtheorem*{lem}{Lemma}

%%% Heading %%%

\usepackage{fancyhdr}
\pagestyle{fancy}
\lhead{Math 185 (HW 3)}
\chead{Michael Knopf (24457981)}
\rhead{July $17^{\text{th}}$, 2015}
\lfoot{}
\cfoot{}
\rfoot{}
\renewcommand\headrulewidth{0.4pt}

\begin{document}

\begin{enumerate}
\setcounter{enumi}{16}






%%%%%%%%%%%%%%%% 17 %%%%%%%%%%%%%%%%







\item Compute, for all values of $n \in \Z$, $(1+i)^n + (1-i)^n$.

\begin{proof}

Observe that
$$
(1+i)^n + (1-i)^n
= 2^{\frac{n}{2}}( e^{i\frac{n\pi}{4}} + e^{-i\frac{n\pi}{4}})
= 2^{\frac{n}{2}} \cdot 2 \re (e^{i\frac{n\pi}{4}})
= 2^{\frac{n}{2} + 1} \cos \left( \frac{n\pi}{4} \right).
$$
Since $\cos \left( \frac{n\pi}{4} \right)$ has a period of $8$, we can evaluate the given expression based on the value of $n \pmod{8}$.
$$
(1+i)^n + (1-i)^n
=
2^{\frac{n}{2} + 1} \cos \left( \frac{n\pi}{4} \right)
=
\begin{cases}
2^{\frac{n}{2} + 1} \cdot 1 & n \equiv 0 \pmod{8} \\
2^{\frac{n}{2} + 1} \cdot (-1) & n \equiv 4 \pmod{8} \\
2^{\frac{n}{2} + 1} \cdot 2^{-\frac12} & n \equiv 1 \text{ or } 7 \pmod{8} \\
2^{\frac{n}{2} + 1} \cdot (-2^{-\frac12}) & n \equiv 3 \text{ or } 5 \pmod{8}
\end{cases}
$$
$$
=
\begin{cases}
2^{\frac{n}{2} + 1} & n \equiv 0 \pmod{8} \\
-2^{\frac{n}{2} + 1} & n \equiv 4 \pmod{8} \\
2^{\frac{n+1}{2}} & n \equiv 1 \text{ or } 7 \pmod{8} \\
-2^{\frac{n+1}{2}} & n \equiv 3 \text{ or } 5 \pmod{8}
\end{cases}
$$
\end{proof}





%%%%%%%%%%%%%%%% 18 %%%%%%%%%%%%%%%%







\item Prove that, for all $z \in \C$, $\sin^2(z) + \cos^2(z) = 1$.

\begin{proof}

\begin{align*}
\sin^2(z) + \cos^2(z)
&=
\left( \frac{e^{iz} + e^{-iz}}{2} \right)^2 + \left( \frac{e^{iz} + e^{-iz}}{2i} \right)^2
\\ &=
\frac{e^{2zi} + e^{-2zi} + 2}{4} + \frac{e^{2zi} + e^{-2zi} - 2}{-4}
\\ &=
\frac{e^{2zi} + e^{-2zi} + 2 - e^{2zi} - e^{-2zi} + 2}{4}
\\ &=
1
\end{align*}

\end{proof}




%%%%%%%%%%%%%%%% 19 %%%%%%%%%%%%%%%%







\item Find the real and imaginary parts of $\sin(z)$, $\cos(z)$, and $e^{e^z}$.

\begin{proof}

For any $x \in \R$, $\cos(ix) = \dfrac{e^{i(ix)} + e^{-i(ix)}}{2} = \dfrac{e^{-x} + e^{x}}{2} = \cosh(x)$, where $\cosh$ is the hyperbolic cosine function.  Similarly, $\sin(ix) = \dfrac{e^{i(ix)} - e^{-i(ix)}}{2i} = \dfrac{e^{-x} - e^x}{2i} = i\dfrac{e^{x} - e^{-x}}{2} = i\sinh(x)$, where $\sinh$ is the hyperbolic sine function.

Letting $z = x + iy$, we have
\begin{align*}
\cos(z) + i\sin(z)
&= e^{iz} \\
&= e^{ix}e^{i(iy)} \\
&= (\cos (x) + i \sin (x)) (\cos (iy) + i \sin(iy)) \\
&= [\cos (x) \cos(iy) - \sin (x) \sin(iy) ] + i [ \sin (x) \cos (iy) + \cos (x) \sin (iy) ].
\end{align*}
Thus, $\cos (z) = \cos (x) \cos(iy) - \sin (x) \sin(iy)$ and $\sin(z) = \sin (x) \cos (iy) + \cos (x) \sin (iy)$.  From the first paragraph, this implies
$$
\cos (z) = \cos(x)\cosh(y) - i\sin(x)\sinh(y)
$$
$$
\sin (z) = \sin(x)\cosh(y) + i\cos(x)\sinh(y).
$$
Therefore,
$$
\re(\cos (z)) = \cos(x)\cosh(y)
$$
$$
\im(\cos(z)) = \sin(x)\sinh(y)
$$
$$
\re(\sin (z)) = \sin(x)\cosh(y)
$$
$$
\im(\sin(z)) = \cos(x)\sinh(y)
$$
\end{proof}




%%%%%%%%%%%%%%%% 20 %%%%%%%%%%%%%%%%







\item Prove that $e^{i\frac{\pi}{4}} = \frac{\sqrt{2}}{2} + \frac{\sqrt{2}}{2}i$.

\begin{proof}

In the proof of Theorem I.5.1, we proved the following:
\begin{itemize}
\item $e^{i\frac{\pi}{2}} = i$.
\item $\cos(x)$ and $\sin(x)$ are strictly positive for $x \in (0, \frac{\pi}{2})$.
\end{itemize}
We also showed, immediately afterwards, that every complex number has exactly two square roots (which are negatives of each other, as we saw on the previous homework).  Since $e^{i\frac{\pi}{4}} = (e^{i\frac{\pi}{2}})^{\frac12}$, this number is a square root of $i$, and therefore must be either $\frac{\sqrt{2}}{2} + \frac{\sqrt{2}}{2}i$ or $-\frac{\sqrt{2}}{2} - \frac{\sqrt{2}}{2}i$.  However, the second option implies that $\cos(\frac{\pi}{2})$ and $\sin(\frac{\pi}{2})$ are negative, contradicting that these functions are positive for $x \in (0,\frac{\pi}{2})$.  Thus we must have $e^{i\frac{\pi}{4}} = \frac{\sqrt{2}}{2} + \frac{\sqrt{2}}{2}i$.
\end{proof}




%%%%%%%%%%%%%%%% 21 %%%%%%%%%%%%%%%%







\item Find all $z \in \C$ such that $\cos(z) = 2$.

\begin{proof}

We want to find all solutions to $\cos(z) = \frac{e^{iz} + e^{-iz}}{2} = 2$.  Since $e^{iz}$ is never 0 for $z \in \C$, we can subtract $2$ and multiply by $e^{iz}$ to obtain an equation with the same solution set.  This equation is
$$
(e^{iz})^2 - 4(e^{iz})^2 + 1 = 0.
$$
The quadratic formula tells us that this equation is equivalent to
$$
e^{iz} = \frac{4 \pm \sqrt{16 - 4}}{2} = 2 \pm \sqrt{3},
$$
which is equivalent to
$$
iz = \log(2 \pm \sqrt{3})
$$
where, here, $\log$ is the ``multi-valued" function of complex numbers.  All solutions to this equation are given by
$$
iz = \log |2 \pm \sqrt{3} | + 2\pi n i
$$
for $n \in \Z$, where $\log$ here is the real-valued function of real numbers.  Multiplying by $-i$ gives $z = 2\pi n - i \log (2 \pm \sqrt{3})$, since $2 \pm \sqrt{3} > 0$.  So the solution set is
$$
\{ 2\pi n - i \log (2 \pm \sqrt{3}) : n \in \Z \}.
$$
\end{proof}




%%%%%%%%%%%%%%%% 22 %%%%%%%%%%%%%%%%







\item Let $\Log$ denote the principal branch of the logarithm.  Find $z_1, z_2 \in \C \setminus (-\infty, 0]$ such that $\Log(z_1 z_2) \neq \Log z_1 + \Log z_2$.

\begin{proof}

Let $z_1 = z_2 = e^{i\frac{2\pi}{3}}$.  Then $$\Log(z_1z_2) = \Log(e^{i\frac{4\pi}{3}}) = -\frac{2\pi}{3}i.$$  However, $$\Log(z_1) + \Log(z_2) = 2\Log(e^{i\frac{2\pi}{3}}) = 2\left(\frac{2\pi}{3} \right)i = \frac{4\pi}{3}i.$$
\end{proof}




%%%%%%%%%%%%%%%% 23 %%%%%%%%%%%%%%%%






\noindent Note: I abuse notation in the following problem.  When $n$ appears in an expression, that expression actually represents ``the set of all numbers of that form" for $n \in \Z$.  Sometimes, two expressions I claim are equal are actually only equal up to a relabeling of $n$; but this is okay, since the sets they represent (under this notation) are actually equal.  I only do this to turn as many negatives into positives as possible without having to create new variables.
\item Compute all values of
\begin{enumerate}
\item $\sin i$, $\cos i$, $\tan(1+i)$.
\begin{proof}

In exercise 19, we showed that for any $x \in \R$, $\cos(ix) = \cosh(x)$ and $\sin(ix) = i \sinh(x)$.  So
$$
\cos(i) = \cosh(i) = \frac{e + e^{-1}}{2}.
$$
$$
\sin(i) = i\sinh(i) = i\frac{e - e^{-1}}{2}.
$$
$$
\tan(1+i) = \frac{\sin(1+i)}{\cos(1+i)} = 
\frac{\frac{e^{(1+i)i} - e^{-(1+i)i}}{2i}}{\frac{e^{(1+i)i} + e^{-(1+i)i}}{2}}
=
\frac{e^{1-i} - e^{-(1-i)}}{e^{1-i} + e^{-(1-i)}}i
$$

\end{proof}
\item $\log(-1)$, $\log(i)$, $\log(1+i)$, $\log(\log(i))$.
\begin{proof}
$$
\log(-1) = \log |-1| + i\arg(-1) = 0 + (\pi + 2\pi n)i = (2n+1)\pi i
$$
$$
\log(i) = \log |i| + i \arg(i) = 0 + \left( \frac{\pi}{2} + 2\pi n \right) i = \left( \frac{\pi}{2} + 2\pi n \right) i
$$
$$
\log(1+i) = \log |1+i| + i \arg(1+i) = \log(\sqrt{2}) + \left( \frac{\pi}{4} + 2\pi n \right) i
$$
for any $n \in \Z$.
$$
\log(\log(i)) = \log \left(\left( \frac{\pi}{2} + 2\pi n \right) i \right) = \log \left| \left( \frac{\pi}{2} + 2\pi n \right) i \right| + i \arg \left( \left( \frac{\pi}{2} + 2\pi n \right) i \right)
$$
$$
= \log \left| \frac{\pi}{2} + 2\pi n \right| + i\left( \frac{\pi}{2} + 2\pi k \right) i
$$
for any $n,k \in \Z$, where $\log$ in the final expressions denotes the real-valued function of real numbers, and $\left| \frac{\pi}{2} + 2\pi n \right|$ denotes the absolute value of this number, since it will be negative if $n < 0$.
\end{proof}
\item $2^i$, $i^i$, $(-1)^{2i}$, $(1+i)^i$, $(-1)^{\frac{1}{\pi}}$.
\begin{proof}
$$
2^i = e^{i \log(2)} = e^{i(\log(2) + 2\pi n i)} = e^{2\pi n + i\log(2)}
$$
$$
i^i = e^{i \log(i)} = e^{i\left( \frac{\pi}{2} + 2\pi n \right) i} = e^{2\pi n - \frac{\pi}{2}}
$$
$$
(-1)^{2i} = e^{2i \log(-1)} = e^{2i(2n+1)\pi i} = e^{2(2n+1)\pi}
$$
$$
(1+i)^i = e^{i \log(1+i)} = e^{i \left(\log(\sqrt{2}) + \left( \frac{\pi}{4} + 2\pi n \right) i \right)} =
e^{-\frac{\pi}{4} + 2\pi n + i\log(\sqrt{2})}
$$
$$
(-1)^{\frac{1}{\pi}} = e^{\frac{1}{\pi}\log(-1)} = e^{\frac{1}{\pi}(2n+1)\pi i} = e^{(2n+1)i}
$$
for any $n \in \Z$.
\end{proof}
\end{enumerate}






%%%%%%%%%%%%%%%% 24 %%%%%%%%%%%%%%%%







\item Show that $\log(i^{\frac{1}{2}}) = \frac12 \log(i)$, in the sense that each denotes the same infinite set of complex numbers.  However, show that $\log(i^2) \neq 2 \log(i)$.

\begin{proof}

First, note that the square roots of $i$ are $e^{\frac{\pi}{4}i}$ and $e^{\frac{5\pi}{4}i}$.  So, if we consider the values of $\log$ and $\arg$ to be sets, we have
\begin{align*}
\log(i^{\frac12})
&= \log(e^{i\frac{\pi}{4}}) \cup \log(e^{i\frac{5\pi}{4}}) \\
&= (\log |e^{i \frac{\pi}{4}}| + i \arg (e^{i \frac{\pi}{4}})) \cup (\log |e^{i \frac{5\pi}{4}}| + i \arg (e^{i \frac{5\pi}{4}})) \\
&= \{i \left(\frac{\pi}{4} + 2\pi n \right) : n \in \Z\} \cup  \{i \left(\frac{5\pi}{4} + 2\pi n \right) : n \in \Z \} \\
&= \{i \left(\frac{\pi}{4} + 2\pi n \right) : n \in \Z\} \cup  \{i \left(\frac{\pi}{4} + \pi + 2\pi n \right) : n \in \Z \} \\
&= \{i \left(\frac{\pi}{4} + 2\pi n \right) : n \in \Z\} \cup  \{i \left(\frac{\pi}{4} + \pi (2n+1) \right) : n \in \Z \} \\
&= \{i \left(\frac{\pi}{4} + \pi n \right) : n \in \Z\}.
\end{align*}
Also,
\begin{align*}
\frac12 \log(i)
&= \{\frac{1}{2} (\frac{\pi}{2} + 2 \pi n)i : n \in \Z \}
= \{i \left(\frac{\pi}{4} + \pi n \right) : n \in \Z\}.
\end{align*}
Thus, $\log(i^{\frac{1}{2}}) = \frac{1}{2}\log(i)$.

Next, observe that
\begin{align*}
\log(i^2) &= \log(-1) = \{(2n+1)\pi i : n \in \Z \}
\end{align*}
but
\begin{align*}
2 \log(i) &= \left\{2 (\frac{\pi}{2} + 2\pi n)i : n \in \Z \right\} = \left\{(4n+1)\pi i : n \in \Z \right\}.
\end{align*}
Therefore, $-\pi i \in \log(i^2)$, but $-\pi i \not \in 2 \log(i)$.  So $\log(i^2) \neq 2 \log(i)$.
\end{proof}


\end{enumerate}

\end{document}

