\documentclass[10pt]{article}
\usepackage[margin=1in]{geometry}
%\addtolength{\oddsidemargin}{-.1in} 
\usepackage{amsmath,amsthm,amssymb}
\usepackage{bm}
\usepackage{enumitem}
\usepackage{array}
\usepackage{lipsum}
\usepackage[]{units}
\usepackage{relsize}
\usepackage{verbatim}
\usepackage{bbm}
\usepackage{mathtools}
\usepackage{eufrak}
\usepackage[mathscr]{euscript}

\usepackage{tikz}
\usetikzlibrary{positioning}
\usepackage{graphicx}
\usepackage{xfrac}

\setenumerate{listparindent=\parindent}

\newcommand{\Q}{\mathbb{Q}}
\newcommand{\Z}{\mathbb{Z}}
\newcommand{\R}{\mathbb{R}}
\newcommand{\N}{\mathbb{N}}
\newcommand{\Int}{{\displaystyle \int}}
\newcommand{\M}{\mathcal{M}}
\newcommand{\B}{\mathcal{B}}
\newcommand{\U}{\mathcal{U}}
\renewcommand{\L}{\mathcal{L}}

\newcommand{\sd}{\Delta}

\DeclareMathOperator*{\dom}{dom}
\DeclareMathOperator*{\Aut}{Aut}
\DeclareMathOperator*{\Ann}{Ann}
\DeclareMathOperator*{\Tor}{Tor}
\DeclareMathOperator*{\Gal}{Gal}
\DeclareMathOperator*{\Hom}{Hom}
\DeclareMathOperator*{\End}{End}
\DeclareMathOperator*{\im}{Im}
\DeclareMathOperator*{\card}{card}

\renewcommand{\bar}{\overline}
\renewcommand{\P}{\mathcal{P}}

\usepackage{fancyhdr} % Required for custom headers 
%\usepackage{lastpage} % Required to determine the last page for the footer

\pagestyle{fancy}
\lhead{Math 202A (HW 5)}
\chead{Michael Knopf (24457981)}
\rhead{October $10^\text{th}$, 2015}
\lfoot{}
\cfoot{}
\rfoot{}
%\rfoot{Page\ \thepage\ of\ \pageref{LastPage}}
\renewcommand\headrulewidth{0.4pt}
%\renewcommand\footrulewidth{0.4pt}

\begin{document}

\begin{enumerate}
\item (F 2.1, exercise 3) If $\{f_n\}$ is a sequence of measurable functions on $X$, then $\{x : \lim f_n(x) \text{ exists}\}$ is a measurable set.

\begin{proof}
We know that the difference of two measurable functions is measurable, and the maximum of two measurable functions is measurable, therefore
\begin{align*}
\{x : \lim f_n(x) \text{ exists}\} &= \bigcap_{k=1}^\infty \bigcup_{N=1}^\infty \bigcup_{n=N+1}^\infty \bigcup_{m=N+1}^\infty \{x : |f_n(x) - f_m(x)| < \frac{1}{k} \}
\\
&= \bigcap_{k=1}^\infty \bigcup_{N=1}^\infty \bigcup_{n=N+1}^\infty \bigcup_{m=N+1}^\infty \{x : \max(f_n(x) - f_m(x), f_m(x) - f_n(x)) < \frac{1}{k} \}
\end{align*}
is a measurable set because $\max(f_n(x) - f_m(x), f_m(x) - f_n(x))$ is a measurable function.
\end{proof}

\item (F 2.1, exercise 4) If $f:X \rightarrow \bar{\R}$ and $f^{-1}((r, \infty]) \in \M$ for each $r \in \Q$, then $f$ is measurable.

\begin{proof}
It is not entirely clear what the measure space of $\bar{\R}$ is intended to be.  I will assume it is $\B_{\bar{\R}}$, which is the $\sigma$-algebra generated by the open sets of $\bar{\R}$.  I cannot find it anywhere mentioned in Folland what the ``usual" topology on $\bar{\R}$ is.  The metric on $\R$ does not extend to this set in a natural way, since distances must be finite.  One condition it is natural to require is that the topology on $\bar{\R}$ restrict to the usual topology on $\R$.  Also, we know that the entire space $[-\infty, \infty]$ must be open.  But then $\{-\infty, + \infty\} = [-\infty, +\infty] \cap (-\infty, + \infty)^c$ is open, hence $U \cup \{-\infty, + \infty\}$ is open for any open $U \subseteq \R$.  $\{+\infty\}$ is open if and only if $\{-\infty\}$ is open, since $\{- \infty\} = \{-\infty, + \infty\} \cap \{+\infty\}^c$.  Also, $\{+\infty\}$ is open if and only if $U \cup \{+\infty\}$ is open for all (or for any) $U \subseteq \R$.  Continuing this train of logic leads us to the conclusion that the only choice to make is whether or not $\{+\infty\}$ is open.  In every source I can find, this is set is taken to be open.  Therefore, the topology on $\bar{\R}$ should be taken to be $$\{U \cup E : U \subseteq \R \text{ is open and } E \in \{\emptyset, \{+\infty\}, \{- \infty\}, \{-\infty, + \infty\} \} \}.$$

We will show that $B = \{(x,\infty] : x \in \R\}$ generates the Borel algebra on $\bar{\R}$.  Let $a < b$.  Then $(a,b] = (a,\infty] \cap (b,\infty]^c$.  Also, $\{+ \infty\} = \bigcap_1^\infty (n, \infty]$ and $\{-\infty\} = \left( \bigcup_1^\infty (-n, \infty] \right)^c$.  Obviously, $\{-\infty, +\infty\} = \{-\infty\} \cup \{+\infty\}$ and $\emptyset = \{-\infty\} \cap \{+\infty\}$.  Also, given $a<b$ we have $(a,b) = \bigcap_1^\infty (a,b+\frac{1}{n}]$, so $B$ generates every open interval, hence it generates every open set of $\R$, along with $\emptyset, \{+\infty\}, \{- \infty\},$ and $\{-\infty, + \infty\}$.  Thus the topology on $\bar{\R}$ is contained in the $\sigma$-algebra generated by $B$, and hence so is the Borel algebra on $\bar{\R}$.  Also, every set in $B$ is open in $\bar{\R}$, and hence $B$ is contained in the Borel algebra on $\bar{\R}$.  So $B$ generates the Borel algebra on $\bar{\R}$.

Finally, let $x \in \R$.  Taking an increasing sequence $r_n \rightarrow x$ of rationals, we have
$$
f^{-1}((x,\infty]) = f^{-1} \left( \bigcap_1^\infty (r_n, \infty ] \right) = \bigcap_1^\infty f^{-1}((r_n, \infty ])
$$
is a countable intersection of measurable sets, hence measurable.  By Proposition 2.1, $f$ is measurable because $f^{-1}(E)$ is measurable for each $E \in B$, and $B$ generates the Borel algebra on $\bar{\R}$.
\end{proof}

\item Let $F: [0, \infty) \rightarrow \R$ be a Lebesgue measurable function such that
$$
f(x+y) = f(x) + f(y)
$$
for all $x,y \geq 0$.  For $a \in \R$, set $\Lambda_a \subseteq [0,\infty), \Lambda_a = \{x \geq 0 : f(x) \geq ax \}$.
\begin{enumerate}
\item[(1)] Show that if $a \in F$ is such that $\mu(\Lambda_a) > 0$, then $\Lambda_a$ contains an interval of the form $[b, \infty)$ for some $b > 0$.
\item[(2)] Show that, in fact, we may take $b = 0$ in part (1).
\item[(3)] Prove that there exists $\lambda \in \R$ such that $f(x) = \lambda x$ for all $x \geq 0$.
\end{enumerate}

\begin{proof}
We begin by proving $f$ is right continuous.  It suffices simply to show that $f$ is right continuous at $0$, since given a sequence $x_n$ converging to some $x \geq 0$ from above, we know $x_n - x \rightarrow 0$.  Thus, $f(x_n - x) \rightarrow 0$, giving
$$
f(x_n) = f(x + (x-x_n)) = f(x) + f(x - x_n) \rightarrow f(x) + 0 = f(x)
$$
hence $f$ is right continuous at $x$.

Note first that $f(0) = f(0 + 0) = f(0) + f(0)$, therefore $f(0) = 0$.  Let $\epsilon > 0$; we want to find a $\delta$ such that $|f(x)| = |f(x) - f(0)| < \epsilon$ whenever $|x| = |x-0| < 0$ (note this also implies that $f(0) = 0$).  Consider the restriction of $f$ to $[0,1]$.  By Lusin's theorem, there exists a closed interval $F \subseteq [0,1]$ such that $\mu([0,1] \setminus F) < \frac23$, i.e. $\mu(F) > \frac23$, and $f \mid_F$ is uniformly continuous on $F$ (since $F$ is compact).  Thus, for any $x,y \in F$ there is a $\delta > 0$ such that $|x-y| < \delta$ implies $|f(x) - f(y)| < \epsilon$.

Let $0 \leq x < \min(\delta, \frac13)$.  If $F \cap F - x$ were empty, then we would have
$$
x+1 \geq \mu(F \cup F - x) = 2\mu(F) > \frac{4}{3}
$$
contradicting that $x < \frac13$.  Therefore, there is some $y \in F \cap F - x$.  So $|(x+y) - y| = |x| < \delta$ by assumption, giving us
$$
|f(x)| = |f(x+y) - f(y)| < \epsilon.
$$
So $f$ is continuous at $0$, and hence continuous on all of $[0,\infty)$.

We have shown already that $f(0 \cdot x) = f(0) = 0 = 0\cdot f(x)$; if $f(nx) = nf(x)$ for some integer $n \geq 0$ then $$f((n+1)x) = f(nx + x) = nf(x) + f(x) = (n+1)f(x).$$  Thus, $f(nx) = nf(x)$ for all integers $n \geq 0$.  Also, for integers $a,b \geq 0$ we have $af(x) = b f(\frac{a}{b} x)$, so dividing by $b$ gives us $\frac{a}{b}f(x) = f(\frac{a}{b}x)$.  Therefore, for nonnegative $q \in \Q$, we have $f(q) = qf(1)$.  So let $\lambda = f(1)$.  Using the denseness of $\Q$ in $\R$, we can construct a sequence $q_n$ of rationals converging to an arbitrary nonnegative $x \in \R$ from above.  By right continuity, we have $f(q_n) \rightarrow f(x)$; but we know $f(q_n) = \lambda q_n \rightarrow \lambda x$.  So $f(x) = \lambda x$ for all nonnegative $x \in \R$.

Parts 1 and 2 are now obvious.  If $\mu(\Lambda_a)$ has positive measure, then it is nonempty.  So there is some $x \geq 0$ such that $\lambda x = f(x) \geq ax$, meaning $\lambda \geq a$.  So $f(x) = \lambda x \geq ax$ for all $x \geq 0$, thus $[0,\infty) \subseteq \Lambda_a$.
\end{proof}

\item Let $A \subset [0,1]$ be a Lebesgue measurable set with $\mu(A) > 0$.  Show that, for any $\epsilon > 0$, there exists an interval $O \subset [0,1]$ such that the relative density of $A$ in $O$ - i.e., the ratio $\mu(A \cap O) / \mu(O)$ - exceeds $1-\epsilon$.

\item Let $r \in [0,1)$.  Consider the map $\tau = \tau_r: [0,1) \rightarrow [0,1)$ that sends $x \in [0,1)$ to $(x+r) \pmod{1}$, the fractional part of $x + r$.  For any $A \subset [0,1)$, consider the union $A^* = \bigcup_{n=0}^\infty \tau^n(A)$, where $\tau^0(A) = A$ and $\tau^n(A)$ is the image of $A$ under the $n$th iterate of the function $\tau$, for $n \geq 1$.
\begin{enumerate}
\item[(1)] If $r \in \Q$, find an example of a Lebesgue measurable set $A \subset [0,1)$ of positive Lebesgue measure for which $A^*$ has Lebesgue measure strictly between zero and one.

\begin{proof}
There are integers $a$ and $b$ such that $r = \frac{a}{b}$.  Let
$$
A = \bigcup_{n=0}^{b-1} \left[\frac{2n}{2b}, \frac{2n+1}{2b} \right).
$$
If $x \in A$, then $\frac{2n}{2b} \leq x < \frac{2n+1}{2b}$ for some $n$.  Thus, $\tau(x) = x + \frac{a}{b} \pmod{1}$ satisfies $\frac{2(n+a)}{2b} \pmod{1} \leq \tau(x) < \frac{2(n+a)+1}{2b} \pmod{1}$.  If $n < b-1$, then the $\pmod{1}$s can simply be ignored, so $x \in A$.  Otherwise, $b = n-1$ and we have $0 \leq \tau(x) < \frac{1}{2b}$, so again $\tau(x) \in A$.  Thus, $\tau(A) \subseteq A$.  By the same argument in reverse, if $x \in \tau(A)$ then $x \in A$.  So we must have $\tau(A) = A$.

This means that $A^* = A$, since the union is simply the union of $A$ with itself an infinite number of times.  We know that $A$ has measure strictly between $0$ and $1$, since $A \subseteq [0,1)$, but $\left[\frac{1}{2b}, \frac{2}{2b} \right) \cap A = \emptyset$ and this interval has positive measure.  Hence, the same applies to $A^*$.

\end{proof}

\item[(2)] If $r \not \in \Q$, prove that, for any such set $A$, the set $A^*$ has Lebesgue measure one.  You may wish to consider the interval $O$ discussed in Q4 and its iterates under $\tau_r$.

\begin{proof}

First, note that $\tau$ is measure-preserving.  If $X \subseteq [0,1)$, then we can decompose $X$ as a disjoint union of $Y$ and $Z$ where $\tau$ is a left translation on $Y$ and a right translation on $Z$ ($Y$ is the part that is ``pushed over" the point 1).  The images of $Y$ and $Z$ are still disjoint, since $\tau$ is a bijection, thus $\mu(X) = \mu(Y) + \mu(Z) = \mu(\tau(Y)) + \mu(\tau(Z)) = \mu(\tau(X))$.

Denote $x \pmod{1}$ by $\bar{x}$.  Let $x \in [0,1)$.  If $\tau^n(x) = x$ for some $n$, then $\bar{x + nr} = x$, hence $nr \in \Z$, contradicting that $r$ is irrational.  Similarly, if there are some $m<n$ such that $\tau^n(x) = \tau^m(x)$, then $\tau^n(x) = \tau^{m + (n-m)}(x) = \tau^{n-m}(\tau^{m}(x)) = \tau^m(x)$, contradicting what we have just proven.  Thus, the images of $x$ under the iterates of $\tau$ form an infinite set within $[0,1)$.

Partition $[0,1)$ into intervals of size $\frac{1}{k}$.  Consider the iterates of $0$, i.e. $\{nr : n \geq 0\}$.  Two iterates must fall into the same interval, by the pigeonhole principle.  So there are some $m\neq n$ such that $0 < \bar{nr} - \bar{mr} < \frac{1}{k}$.  So we can write $nr = t + s$ and $mr = u + v$ where $t,u \in \Z$, $s,v < 1$, and $s-v < \frac1k$.  Thus, $(n-m)r = (t-u) + (s-v)$ and $t-u \in \Z$, so $\bar{n-m} < \frac1k$.  If $n-m > 0$, then we have produced a positive iterate of $0$ that is less than $\frac1k$.

Suppose this is not the case.  Since $0 < s-v < \frac1k$, there is some positive integer $N$ such that $\frac{k-1}{k} < N(s-v) < 1$.  Therefore, $$N(m-n) = [N(u-t)-1] + [N(v-s)+1]$$
and $0 < N(v-s)+1 < \frac1k$.  Therefore, $N(m-n)$ is a positive iterate of $0$ that is less than $\frac1k$.

For any $x \in [0,1)$ and any positive $y < \frac1k$, there is some $N$ such that $|x - Ny| < \frac{1}{k}$.  But if $y$ is an iterate of $0$, then so is $Ny$.  Hence, the iterates of $0$ are dense in $[0,1)$.  If we shift a dense set and take the image $\pmod{1}$, we still have a dense set.  The iterates of a given $x \in [0,1)$ are simply the iterates of $0$ shifted by $x$, thus they are also dense.

Now, let $O$ be a nonempty interval in $[0,1)$.  Let $x$ be the center of $O$.  The iterates of $O$ cover the iterates of $x$, and the iterates of $x$ are dense in $[0,1)$.  If $2l$ is the length of $O$, then any point within $l$ of an iterate of $x$ is covered by an iterate of $O$.  But \emph{every} point of $[0,1)$ is within $l$ of an iterate of $x$, thus the iterates of $O$ cover $[0,1)$.

Let $B = (A^*)^c$, and suppose for a contradiction that $\mu(B) > 0$.  For any $\epsilon$ such that $0 < \epsilon < 1$, we can find an interval $O$ such that $\frac{\mu(A \cap O)}{\mu(O)} > 1 - \epsilon$.  Let $A_i$ and $O_i$ be the $i$th iterates of $A$ and $O$, respectively.  Then $\frac{\mu(A_i \cap O_i)}{\mu(O_i)} > 1 - \epsilon$ as well, since $\tau$ is measure-preserving.  Since $B \cap A_i = \emptyset$, for all $i$ we have
\begin{align*}
\mu(O_i) &= \mu((A^* \cup B) \cap O_i) \\
&= \mu((A^* \cap O_i) \cup (B \cap O_i)) \\
&= \mu(A^* \cap O_i) + \mu(B \cap O_i) \\
&\geq \mu(A_i \cap O_i) + \mu(B \cap O_i) \\
&> (1-\epsilon)\mu(O_i) + \mu(B \cap O_i)
\end{align*}
thus $\mu(B \cap O_i) < \epsilon \mu(O_i)$.  Since $\epsilon$ is arbitrary, this means $\mu(B \cap O_i) = 0$ for all $i$.  But the $O_i$ cover $[0,1)$, thus $\bigcup_0^\infty B \cap O_i = B$.  So
$$
\mu(B) = \mu(\bigcup_0^\infty B \cap O_i) \leq \sum_0^\infty \mu(B \cap O_i) = 0.
$$
Since $B = (A^*)^c$ has measure 0, $A^*$ must have measure $1$.
\end{proof}

\end{enumerate}
\end{enumerate}
\end{document}







