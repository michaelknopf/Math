\documentclass[10pt]{article}
\usepackage[margin=1in]{geometry}
%\addtolength{\oddsidemargin}{-.1in} 
\usepackage{amsmath,amsthm,amssymb}
\usepackage{bm}
\usepackage{enumitem}
\usepackage{array}
\usepackage{lipsum}
\usepackage[]{units}
\usepackage{relsize}
\usepackage{verbatim}
\usepackage{bbm}

\usepackage{tikz}
\usetikzlibrary{positioning}
\usepackage{graphicx}
\usepackage{xfrac}
\usetikzlibrary{cd}


\setenumerate{listparindent=\parindent}

\newcommand{\Q}{\mathbf{Q}}
\newcommand{\Z}{\mathbf{Z}}
\newcommand{\R}{\mathbf{R}}
\newcommand{\gen}[1]{\langle #1 \rangle}
\DeclareMathOperator*{\dom}{dom}
\DeclareMathOperator*{\Aut}{Aut}
\DeclareMathOperator*{\Ann}{Ann}
\DeclareMathOperator*{\Tor}{Tor}
\DeclareMathOperator*{\Gal}{Gal}
\DeclareMathOperator*{\Hom}{Hom}
\DeclareMathOperator*{\End}{End}
\DeclareMathOperator*{\im}{Im}
\DeclareMathOperator*{\Perm}{Perm}
\DeclareMathOperator*{\card}{card}
\DeclareMathOperator*{\Alt}{Alt}
\renewcommand{\bar}{\overline}

\newtheorem*{lem}{Lemma}

\usepackage{fancyhdr} % Required for custom headers 
%\usepackage{lastpage} % Required to determine the last page for the footer

\pagestyle{fancy}
\lhead{Math 250A (HW 3)}
\chead{Michael Knopf (24457981)}
\rhead{September $17^\text{th}$, 2015}
\lfoot{}
\cfoot{}
\rfoot{}
%\rfoot{Page\ \thepage\ of\ \pageref{LastPage}}
\renewcommand\headrulewidth{0.4pt}
%\renewcommand\footrulewidth{0.4pt}

\begin{document}
\begin{enumerate}
\item[28.] Let $p, q$ be distinct primes.  Prove that a group of order $p^2 q$ is solvable, and that one of its Sylow subgroups is normal.

\begin{proof}
Suppose $|G| = p^2q$ for some primes $p \neq q$, and let $n_p$ and $n_q$ be the numbers of $p$- and $q$-Sylows, respectively.  We know that $n_p \mid q$ and $n_q \mid p^2$.  If either $n_p = 1$, then the $p$-Sylow subgroup $P$ is stabilized by conjugation, hence is normal (and similarly if $n_q = 1$).  So we may assume $n_p = q$ and $n_q = p$ or $p^2$.  If $n_q = p^2$, we have a combinatorial issue regarding the size of $G$.  Since the $q$-Sylows are cyclic, their pairwise intersections must be trivial.  Similarly, they must have trivial intersection with each $p$-Sylow as well, else their intersection would generate an order $q$ subgroup of a $p$-Sylow, contradicting that $q \nmid p$.  So these $q$-Sylows, along with just one of the $p$-Sylows, account for $p^2(q-1) + (p^2 - 1) + 1 + p^2q = |G|$ elements of the group.  This leaves no room for any more $p$-Sylows, contradicting that $n_p \neq 1$.  Therefore, $G$ contains a normal subgroup (either a $p$-Sylow or a $q$-Sylow).

We showed in exercise 27 that if some normal subgroup $N$ of $G$ is solvable, and $G / N$ is solvable, then $G$ is solvable.  If there is a normal $p$-Sylow $P$, then $G / P \cong \Z_p$, thus is solvable.  We showed in exercise 24 that a group of order $p^2$ is abelian, so $P$ is also solvable.  So in this case, $G$ is solvable.  The only other possibility is that there is a normal $q$-Sylow $Q$, which is abelian and hence solvable.  $G/Q$ has order $p^2$, so again by exercise 24 is solvable.  So $G$ is solvable.
\end{proof}

\item[29.] Let $p,q$ be odd primes.  Prove that a group of order $2pq$ is solvable.

\begin{proof}
Let $G$ be a group of order $2pq$.  First, suppose $p = q$.  Then $G$ has order $2p^2$, and we are reduced to the situation solved in exercise 28.  So we may assume $p \neq q$.

We will first show that either a $2$-, $p$-, or $q$-Sylow is normal.  Assume this is false, meaning that $n_2, n_p, n_q > 1$.  We may also assume, without loss of generality, that $p < q$.  Since $n_q \mid 2p$, we know $n_q \in \{2,p,2p\}$.  However, since $n_q \equiv 1 \pmod{q}$, and $2 < q$, we cannot have $n_q = 2$, else $2 \equiv 1 \pmod {q}$, a contradiction.  If $n_q \equiv p \pmod{q}$, then since $p < q$ we have the same problem in that $p \equiv 1 \pmod{q}$, a contradiction.  So the only possibility is $n_q = 2p$.

Next, we know that $n_p \in \{2,q,2q\}$.  Since $q > 2$, we cannot have $n_p = 2$ by the same reasoning (that it would imply $2 \equiv 1 \pmod{p}$).  So we can assume $n_p \geq q$.  Finally, we know $n_2 \in \{p,q,pq\}$, so that $n_2 \geq p$.

Each Sylow subgroup is cyclic, and therefore any two Sylows must have trivial intersection.  This means the number of elements accounted for by these Sylows is at least $2p(q-1) + q(p-1) + p(2-1) + 1$.  This number cannot exceed the order of the group, thus
$$
2p(q-1) + q(p-1) + p(2-1) + 1 \leq 2pq
$$
or, equivalently, $(p-1)(q-1) \leq 0$.  The only solutions to this have $p \leq 1$ or $q \leq 1$, a contradiction.  Therefore, some Sylow subgroup is normal.

Let $H$ be this normal Sylow subgroup.  $H$ has order $2$, $p$, or $q$, so is cyclic and hence solvable.  All that remains to show is that $G/H$ is solvable.  $G/H$ has order $pq$, $2q$, or $2p$.  Any group whose order is a product of two primes is solvable, so $G/H$ is solvable.  Thus $G$ is solvable.
\end{proof}

\item[32.] Let $S_n$ be the permutation group on $n$ elements.  Determine the $p$-Sylow subgroups of $S_3, S_4, S_5$ for $p = 2$ and $p=3$.

\begin{proof}
Suppose $\gen{\sigma}$ is an order $k$ cyclic subgroup of $S_n$.  We can decompose $\sigma$ into (nonempty) disjoint cycles as $\sigma = \sigma_1 \cdots \sigma_j$, and these cycles commute with each other.  We have
$$
\sigma^k = (\sigma_1 \cdots \sigma_j)^k = \sigma_1^k \cdots \sigma_j^k = e
$$
thus $\sigma_1^k = \cdots = \sigma_j^k = e$.  If $k = 2$ or $k=3$, then this means that $\sigma$ is a product of disjoint $2$- or $3$-cycles, respectively.  Conversely, it is clear that if $\sigma$ is a product of disjoint $2$- or $3$-cycles, then it generates a subgroup of order $2$ or $3$, respectively.  Therefore, the subgroups of order $2$ or $3$ are precisely those generated by a product of disjoint $2$-cycles, resp. $3$-cycles.

It is easy to list the $3$-Sylows for these groups.  The highest power of $3$ that divides $3!$, $4!$, or $5!$ is $3^1$, so the $3$-Sylow subgroups are just the cyclic subgroups of order $3$, which we have just identified.  No two $3$-cycles in $S_n$ are disjoint when $n \leq 5$, so the $3$-Sylows are simply the subgroups generated by any three-cycle.  Note also that, given any three objects, there are only two ways they can be cycled, and these two permutations generate each other.  Therefore, there are $\binom{n}{3}$ $3$-Sylow subgroups of $S_n$ (when $n \leq 5$).  For $S_3$, there is only $\gen{(123)}$.  For $S_4$, they are $\gen{(123)}, \gen{(124)},$ and $\gen{(134)}$.  For $S_5$, they are $\gen{(123)}, \gen{(124)}, \gen{(125)}, \gen{(134)}, \gen{(135)}, \gen{(145)}, \gen{(234)}, \gen{(235)}, \gen{(245)},$ and $\gen{(345)}$.

The highest power of $2$ that divides $3!$ is $2^1$, so we are looking for cyclic $2$-groups.  In $S_3$, no two $2$-cycles are disjoint, and no two $2$-cycles generate the same subgroup.  So there are $\binom{3}{2} = 3$ $2$-Sylows, specifically $\gen{(12)}, \gen{(13)},$ and $\gen{(23)}$.

The $2$-Sylows of $S_4$ and $S_5$ have order $2^3 = 8$.  Observe first that $\gen{(13),(1234)}$ is an order $8$ subgroup of both $S_4$ and $S_5$ which is isomorphic to $D_8$ under the mapping $(13) \mapsto s$, $(1234) \mapsto r$.  All $2$-Sylows are conjugate, and hence isomorphic.  Therefore, the $2$-Sylows are precisely the copies of $D_8$ within $S_4$ and $S_5$.

In $S_n$, for $n = 4$ or $n=5$, there are $\binom{n}{4} \cdot 3$ copies of $S_4$.  This can be confirmed simply by taking the conjugates of the subgroup $\gen{(13),(1234)}$ in each group.  However, a more intuitive reasoning is that each copy of $D_8$ is the symmetry group of a square labeled with some choice of four numbers from $\{1, \dots , n \}$.  After choosing these four numbers, the only remaining decision is which labels will be positioned on corners opposite each other.  There are 3 ways to decide this.

The three $2$-Sylows in $S_4$ are $\gen{(12),(1324)}, \gen{(13),(1234)},$ and $\gen{(14),(1243)}$.  In $S_5$, there are $15$:
$$
\begin{array}{ccc}
\gen{(12),(1324)} &  \gen{(13),(1234)} & \gen{(14),(1243)}  \\ 
\gen{(12),(1325)} & \gen{(13),(1235)} & \gen{(15),(1253)} \\ 
\gen{(12),(1425)} & \gen{(14),(1245)} & \gen{(15),(1254)} \\
\gen{(13),(1435)} & \gen{(14),(1345)} & \gen{(15),(1354)} \\ 
\gen{(23),(2435)} & \gen{(24),(2345)} & \gen{(25),(2354)} \\
\end{array} 
$$

\end{proof}

\item[33.] Let $\sigma$ be a permutation of a finite set $I$ having $n$ elements.  Define $e(\sigma)$ to be $(-1)^n$ where
$$
m = n - \text{number of orbits of } \sigma
$$

If $I_1, \dots, I_r$ are the orbits of $\sigma$, then $m$ is also equal to the sum
$$
m = \sum_{v=1}^r [\card(I_v) - 1].
$$
If $\tau$ is a transposition, show that $e(\sigma \tau ) = -e(\sigma)$ by considering the two cases when $i,j$ lie in the same orbit of $\sigma$, or lie in different orbits.  in the first case, $\sigma \tau$ has one more orbit and in the second case one less orbit than $\sigma$.  In particular, the sign of a transposition is $-1$.  Prove that $e(\sigma) = \varepsilon(\sigma)$ is the sign of the permutation.

\begin{proof}
Let $\tau$ be the transposition swapping $i$ and $j$.  $\sigma$ decomposes as a product of disjoint cycles: $\sigma = \sigma_1 \cdots \sigma_r$.  The $v$th orbit of $\sigma$ is precisely the set $I_v = \{x \mid \sigma_v(x) \neq x \}$.  First, suppose $i$ and $j$ lie in the same orbit, which we can assume without loss of generality is orbit $1$.

$\sigma_1$ takes the form $\sigma_1 = (i \ s_1 \cdots s_l \ j \ s_{l+1} \cdots s_k)$ for some $s_1, \dots s_k \in I_1$.  So
$$
\sigma_1 \tau = (i \ s_1 \cdots s_l \ j \ s_{l+1} \cdots s_k) (i \ j) = (i \ s_{l+1} \cdots s_k) (s_1 \cdots s_l \ j)
$$
$\tau$ commutes with $\sigma_u$ for all $u \neq 1$, so
$$\sigma \tau = \sigma_1 \cdots \sigma_r \tau = \sigma_1 \tau \sigma_2 \cdots \sigma_r = (i \ s_{l+1} \cdots s_k) (s_1 \cdots s_l \ j) \sigma_2 \cdots \sigma_r
$$
has one more orbit than $\sigma$ did.  Therefore, $e(\sigma \tau) = (-1)^{m - 1} = - (-1)^m = - e(\sigma)$.

Next, suppose $i$ and $j$ lie in different orbits.  Again, since disjoint cycles commute, we may assume WLOG that $i$ lies in orbit 1 and $j$ lies in orbit 2.  So $\sigma_1 = (i \ s_1 \cdots s_k)$ and $\sigma_2 = (j \ t_1 \cdots t_l)$.  Now,
$$
\sigma_1 \sigma_2 \tau = (i \ s_1 \cdots s_k) (j \ t_1 \cdots t_l) (i \ j) = (i \ t_1 \cdots t_l \ j \ s_1 \cdots s_k ).
$$
So
$$
\sigma \tau = (i \ t_1 \cdots t_l \ j \ s_1 \cdots s_k ) \sigma_3 \cdots \sigma_r
$$
has one less orbit than $\sigma$, thus $e(\sigma \tau) = (-1)^{m + 1} = - (-1)^m = - e(\sigma)$.  So, in all cases, $e(\sigma \tau) = - e(\sigma)$.

The sign of the identity permutation $()$ is $1$, and $e(()) = (-1)^{n-n} = 1$ since the identity permutation has $n$ orbits.  So for the identity permutation, $e$ is its sign.  We also have in particular that, if $\tau$ is a transposition, then $e( () \tau) = - e(()) = -1$ is the sign of $\tau$.

Any permutation $\sigma$ can also be decomposed into transpositions as $\sigma = \tau_1 \cdots \tau_k$.  Its sign is $(-1)^k$.  By induction, $e(\sigma) = e(\tau_1 \cdots \tau_{k-1} \tau_k) = -e(\tau_1 \cdots \tau_{k-1}) = -(-1)^{k-1} = (-1)^k$ as well.  Thus, $e(\sigma)$ is the sign of $\sigma$, for any permutation $\sigma$.
\end{proof}

\item[39.] Show that the action of $A_n$ on $\{1, \dots , n \}$ is $(n-2)$-transitive.

\begin{proof}
Suppose we are given two tuples $(s_1, \dots , s_{n-2})$ and $(t_1, \dots , t_{n-2})$ of $n-2$ distinct elements from $\{1,\dots,n\}$.  Then $\{1,\dots,n\} \setminus \{s_1, \dots , s_{n-2}\}$ is some set of two elements $s_{n-1}, s_n$, and $\{1,\dots,n\} \setminus \{t_1, \dots , t_{n-2}\}$ is some set of two elements $t_{n-1}, t_n$.  We can form the permutation
$ \sigma = 
\begin{pmatrix}
s_1 & \cdots & s_{n} \\
t_1 & \cdots & t_{n}
\end{pmatrix}.
$
If $\sigma$ is an even permutation, then we are done.  Otherwise, $\sigma \circ (s_{n-1} \ s_n)$ is an even permutation which has the same effect on the elements we are interested in.
\end{proof}

\item[40.] Let $A_n$ be the alternating group of even permutations of $\{1,\dots , n\}$.  For $j = 1, \dots , n$ let $H_j$ be the subgroup of $A_n$ fixing $j$, so $H_j \cong A_{n-1}$, and $(A_n : H_j) = n$ for $n \geq 3$.  Let $n \geq 3$ and let $H$ be a subgroup of index $n$ in $A_n$.
\begin{enumerate}
\item Show that the action of $A_n$ on cosets of $H$ by left translation gives an isomorphism of $A_n$ with the alternating group of permutations of $A_n / H$.

\begin{proof}
The described action gives a homomorphism $A_n \rightarrow \Perm (A_n / H) \cong S_n$.  Let $K$ be the kernel.  First, recognize that $K \subseteq H$, because $g \in K$ implies in particular that $gH = H$.  So $K \neq A_n$.

We have shown that $A_n$ is simple whenever $n \geq 5$, and $A_3 \cong \Z_3$ is also simple.  So in both of these cases, we must have $K = \{e\}$.  When $n = 4$, the index of $H$ must be 4, so $|H| = 3$.  Since $3^1$ is the highest power of $3$ dividing $|A_n| = 12$, all groups of order $3$ are conjugate.  One of these groups is $\{(), (1 \ 2 \ 3), (1 \ 3 \ 2)\}$.  In order for $H$ to be normal, $H$ must equal this group and it can have no other conjugates.  However, $(1 \ 4) (1 \ 2 \ 3)(1 \ 4) = (2 \ 3 \ 4) \not \in H$.  So $H$ is not normal, therefore $K \neq H$.

So, in all cases, the action has a trivial kernel, so the action gives an embedding of $A_n$ into $\Perm(A_n / H)$, whose image must then be the alternating subgroup of this permutation group.
\end{proof}

\item Show that there exists an automorphism of $A_n$ mapping $H_1$ on $H$, and that such an automorphism is induced by an inner automorphism of $S_n$ if and only if $H = H_i$ for some $i$.

\begin{proof}
We have just seen that the action of $A_n$ by left translation on $A_n / H$ gives an isomorphism of $A_n$ into $\Alt(A_n / H)$.  If $h \in H$, then translation by $h$ fixes the coset $H$.  So, labeling the coset $H$ as element $1$, the restriction of this isomorphism to $H$ yields an embedding of $H$ into $H_1$.  Since these two subgroups have equal size, this embedding is actually an isomorphism.

Now, let $i,j \in \{1, \dots , n\}$ and let $\sigma \in S_n$ be such that $\sigma(i) = j$.  Clearly, $\sigma H_i \sigma^{-1} \subseteq H_j$.  However, since $|\sigma H_i \sigma^{-1}| = |H_j|$, we actually have an isomorphism.  Now, suppose $\sigma H \sigma^{-1} = H_j$.  Then $H = \sigma^{-1}H_j \sigma$.  Since $\sigma^{-1}(j) = i$, we must have $H = H_i$.  Thus this automorphism is induced by an inner automorphism of $S_n$ if and only if $H = H_i$ for some $i$.
\end{proof}

\end{enumerate}

\item[44.] Let $f: A \rightarrow A'$ be a homomorphism of abelian groups.  Let $B$ be a subgroup of $A$.  Denote by $A^f$ and $A_f$ the image and kernel of $f$ in $A$ respectively, and similarly for $B^f$ and $B_f$.  Show that $(A:B) = (A^f : B^f)(A_f : B_f)$, in the sense that if two of these three indices are finite, so is the third, and the stated equality holds.

\begin{proof}
We have a morphism of short exact sequences
$$
\begin{tikzcd}
0 \arrow[r] & A_f \arrow[r, hook] & A \arrow[r, twoheadrightarrow, "f"] & A^f \arrow[r] & 0 \\
0 \arrow[r] & B_f \arrow[u, hook] \arrow[r, hook] & B \arrow[u, hook] \arrow[r, twoheadrightarrow, "f"] & B^f \arrow[u, hook] \arrow[r] & 0 \\
\end{tikzcd}
$$
which yields a new short exact sequence
$$
\begin{tikzcd}
0 \arrow[r] & A_f / B_f \arrow[r, hook, "\iota"] & A/B \arrow[r, twoheadrightarrow, "\bar{f}"] & A^f / B^f \arrow[r] & 0
\end{tikzcd}
$$
(We actually have a special case of this where the inclusions are literally containments, but this is not in general necessary for the result.)
It is clear that both sequences in the top diagram are exact and that the diagram commutes.  In this new diagram, we have taken $\iota$ to be $a+B_f \mapsto a+B$ and $\bar{f}$ to be $a+B \mapsto f(a)+B^f$.  These maps are obviously well-defined (if $a \in B_f$ then $a \in B$; if $a \in B$ then $f(a) \in B^f$).  $\iota$ is injective because $A_f \cap B = B_f$, and $\bar{f}$ is clearly surjective.


The image of $\iota$ is $\{a+B \mid a \in A_f\}$.  The sequence is exact because the kernel of $\bar{f}$ is
\begin{align*}
\ker(\bar{f}) &= \{a+B \mid f(a) \in B^f \} \\
&= \{a+B \mid f(a) = f(b) \text{ for some } b \in B \} \\
&= \{a+B \mid a - b \in A_f \text{ for some } b \in B \} \\
&= \{a + b + B \mid a \in A_f, b \in B \} \\
&= \{a + B \mid a \in A_f \}
\end{align*}
as well.  Therefore, we have an isomorphism $(A/B) / \iota(A_f / B_f) \cong A^f / B^f$ (this occurs with any short exact sequence of abelian groups:  if $0 \rightarrow K \xrightarrow{\varphi} G \xrightarrow{\psi} H \rightarrow 0$ is exact, then $x + \varphi(K) \mapsto \psi(x)$ is an isomorphism of $G / \varphi(K)$ to $H$).

Since $|A_f / B_f| = |\iota(A_f / B_f)|$, it is clear now that if any two of the mentioned indices are finite, then so must be the third, and they must satisfy the asserted equality.
\end{proof}

\item[45.] Let $G$ be a finite cyclic group of order $n$, generated by an element $\sigma$.  Assume that $G$ operates on an abelian group $A$, and let $f,g:A \rightarrow A$ be the endomorphisms of $A$ given by
$$
f(x) = \sigma x - x \text{ and } g(x) = x + \sigma x + \cdots + \sigma^{n-1} x.
$$
Define the \textbf{Herbrand quotient} by the expression $q(A) = (A_f:A^g) / (A_g : A^f)$, provided both indices are finite.  Assume now that $B$ is a subgroup of $A$ such that $GB \subseteq B$.
\begin{enumerate}
\item Define in a natural way an operation of $G$ on $A / B$.
\begin{proof}
The natural operation is $\sigma^k(a+B) = \sigma^k(a)+B$.  This is well-defined because of the assumption $GB \subseteq B$, since if $a \in B$ then $\sigma^k(a)+B = B$.
\end{proof}
\item Prove that
$$
q(A) = q(B)q(A/B)
$$
in the sense that if two of these quotients are finite, so is the third, and the stated equality holds.
\begin{proof}
An online collection of notes at \\ \verb|http://www.math.harvard.edu/~chaoli/doc/ClassFieldTheory2.html#cor:CFT2hexagon| on class field theory suggest using an ``exact hexagon" to show this result.  It does not give details, only the diagram itself.  I was nowhere close to coming up with this idea myself, but this hint gave me what I needed to solve the problem.

First, check that we actually have $A^g \subseteq A_f$ and $A^f \subseteq A_g$.  Notice that every element of $A^g$ is fixed by $\sigma$, since applying $\sigma$ to $x + \sigma x + \cdots + \sigma^{n-1}x$ simply reorders the terms.  This gives the first inclusion.  For the second, suppose $x = \sigma y - y$ for some $y \in A$.  Then $\sigma^{k-1} = \sigma^k y - \sigma^{k-1}y$ for all integers $k$, so
\begin{align*}
g(x) &= x + \sigma x + \cdots + \sigma^{n-1} x \\
&= (\sigma y - y) + (\sigma^2 y - \sigma y) + \cdots + (\sigma^{n-1} y - \sigma^{n-2}y) + (y - \sigma^{n-1} y) \\
&= 0.
\end{align*}

First, we will show that there are two exact sequences
$$
\begin{tikzcd}
B_f / B^g \arrow[r, "\alpha"] & A_f / A^g \arrow[r, "\beta"] & (A/B)_f / (A/B)^g
\end{tikzcd}
$$
$$
\begin{tikzcd}
B_g / B^f \arrow[r, "\alpha'"] & A_g / A^f \arrow[r, "\beta'"] & (A/B)_g / (A/B)^f.
\end{tikzcd}
$$
$\alpha$ is the map $b+B^g \mapsto b+A^g$, which is well-defined because $B^g \subseteq A^g$.  $\beta$ is the map $a+A^g \mapsto (a+B) + (A/B)^g$.  This is well-defined because if $a \in A^g$, then there is some $x \in A$ such that $a+B = g(x) + B = g(x+B) \in (A/B)^g.$  The image of $\alpha$ is $\{b + A^g \mid b \in B_f\}$.  We will explain why this equals the kernel of $\beta$.

The inclusion $\im(\alpha) \subseteq \ker(\beta)$ is clear, so for the other direction suppose $a+A^g \in \ker(\beta)$, that is $a \in A_f$ and $a+B \in (A/B)^g$.  Then $a - g(x) \in B$ for some $x \in A$.  So $a - g(x) = b$ for some $b \in B$.  But $a \in A_f$ by assumption, and $g(x) \in A^g \subseteq A_f$, therefore $b \in B_f$.  Therefore, $a - b  = g(x) \in A^g$, so $a + A^g = b + A^g$ for some $b \in B_f$, meaning $a + B \in \ker(\beta)$.

For the next sequence, we define $\alpha'$ by $b + B^f \mapsto b + A^f$, which is well-defined because $B^f \subseteq A^f$.  We define $\beta'$ to be $a + A^f \mapsto (a+B) + (A/B)^f$.  If $a \in A^f$, then $f(x) = a$ for some $x \in A$, thus $a+B = f(x) + B \in (A/B)^f$.  So $\beta'$ is well-defined.

The image of $\alpha'$ is $\{b+A^f \mid b \in B_g\}$.  Again, $\im(\alpha') \subseteq \ker(\beta')$ is trivial.  For the other direction, suppose $a + A^f \in \ker(\beta')$.  Then $a \in A_g$ and $a+B \in (A/B)^f$.  So $a+B = f(x) + B$, i.e. $a - f(x) \in B$, for some $x \in A$.  But $f(x) \in A^f \subseteq A_g$, and $a \in A_g$ by assumption, thus $a - f(x) \in B \cap A_g = B_g$.  Let $b = a - f(x)$.  Now, $a - b = f(x) \in A^f$, thus $a + A^f = b + A^f$ for some $b \in B_g$, therefore $a + A^f \in \im(\alpha')$.  This proves that the second sequence is exact.

Now, consider the maps $\varphi:(A/B)_f / (A/B)^g \rightarrow B_g / B^f$ and $\psi : (A/B)_g / (A/B)^f$ given by
$$
(a+B) + (A/B)^g \mapsto f(a) + B^f
$$
$$
(a+B) + (A/B)^f \mapsto g(a) + B^g
$$
respectively.  We will show that these are well-defined, beginning with $\varphi$.  If $a \in (A/B)_f$, then $f(a + B) = B$, so $f(a) \in B$.  Also, $f(a) \in A^f \subseteq A_g$, thus $f(a) \in B \cap A_g = B_g$, as desired.  Also, if $a \in B$, then $f(a) \in B^f$, so each element is assigned a unique image.

A similar argument shows that $\psi$ is well-defined.  If $a \in (A/B)_g$, then $g(a) \in B$.  But $g(a) \in A^g \subseteq A_f$ as well, so $g(a) \in B_f$, as desired.  Also, if $a \in B$ then $g(a) \in B^g$, so each element is assigned a unique image.

These six maps, when strung together, yield the following exact hexagon:

$$
\begin{tikzcd}
\ & B_f / B^g \arrow[r, "\alpha"] & A_f / A^g \arrow[dr, "\beta"] &  \ \\
(A/B)_g / (A/B)^f \arrow[ur, "\psi"] & \ & \ & (A/B)_f / (A/B)^g \arrow[dl, "\varphi"] \\
\ & A_g / A^f \arrow[ul, "\beta'"] & B_g / B^f \arrow[l, "\alpha'"] & \ \\
\end{tikzcd}
$$

\begin{comment}
$$
\begin{tikzcd}
B_f / B^g \arrow[r, "\alpha"] & A_f / A^g \arrow[r, "\beta"] & (A/B)_f / (A/B)^g \arrow[dd, "\varphi"] \\ \ & \ & \phantom{a} \\
(A/B)_g / (A/B)^f \arrow[uu, "\psi"] & A_g / A^f \arrow[l, "\beta'"] & B_g / B^f \arrow[l, "\alpha'"]
\end{tikzcd}
$$
\end{comment}

All we must show is that the proper images and kernels coincide.  First, we compute
\begin{align*}
\im(\varphi) &= \{f(a) + B^f \mid (a+B) \in (A/B)_f \} \\
&= \{f(a) + B^f \mid f(a) \in B , a \in A\} \\
&= \ker(\alpha') \\
\im(\psi) &= \{g(a) + B^g \mid (a+B) \in (A/B)_g \} \\
&= \{g(a) + B^g \mid g(a) \in B, a \in A \} \\
&= \{b + B^g \mid b \in B \cap A^g \} \\
&= \ker(\alpha)
\end{align*}
Next, note that $\im(\beta) = \{(a+B) + (A/B)^g \mid a \in A_f\}$ and $\ker \varphi = \{(a+B) + (A/B)^g \mid f(a) \in B^f \}$.  The inclusion $\im(\beta) \subseteq \ker(\varphi)$ is easy, since if $a \in A_f$ then $f(a) = 0 \in B^f$.  For the reverse inclusion, suppose $a \in A$ and $f(a) \in B^f$.  Then $f(a) = f(b)$ for some $b \in B$, i.e. $a-b \in A_f$.  Then $(a+B) + (A/B)^g = \beta(a-b + A^g) \in \im(\beta)$.  The same exact arguments, replacing $f$ with $g$, prove that $\im(\beta ' ) = \ker(\psi)$.  Therefore, the above hexagon is exact.

With this, the result immediately follows, since the quotient of any group in this chain by the previous is isomorphic to the following, which gives us
\begin{align*}
(A_f : A^g) &= (B_f : B^g)((A/B)_f : (A/G)^g) \\
(A_g : A^f) &= (B_g : B^f)((A/B)_g : (A/B)^f) \\
\implies
\frac{(A_f : A^g)}{(A_g : A^f)} &= \frac{(B_f : B^g)}{(B_g : B^f)} \frac{((A/B)_f : (A/G)^g)}{((A/B)_g : (A/B)^f)} \\
\implies \ \ \ \ \ \ \
q(A) &= q(B)q(A/B).
\end{align*}
\end{proof}
\item If $A$ is finite, show that $q(A) = 1$.

\begin{proof}
\begin{comment}
First, we will show the result for cyclic groups of prime order $p$, so let $A = \Z_p$.  In this case, the kernels of $f$ and $g$ completely determine their images, and conversely.  For instance, if $A_f$ is trivial, then $A^f = f(A) = \{0\}$.  This gives us the following table of possibilities for these indices, and the resulting value of $q(A)$:
\begin{center}
\renewcommand{\arraystretch}{1.2}
\begin{tabular}{|c|c|c|c|c|c|c|}
\hline 
$|A_f|$ & $|A^g|$ & $|A_g|$ & $|A^f|$ & $(A_f:A^g)$ & $(A_g:A^f)$ & $q(A)$ \\ 
\hline 
$p$ & $p$ & $0$ & $0$ & $1$ & $1$ & $1$ \\ 
\hline 
$p$ & $0$ & $p$ & $0$ & $p$ & $p$ & $1$ \\ 
\hline 
$0$ & $0$ & $p$ & $p$ & $1$ & $1$ & $1$ \\ 
\hline 
\end{tabular} 
\end{center}
So the result holds when $A$ is cyclic of prime order.

The general claim now follows by induction.  The result is trivial when $A = \{0\}$, so suppose it holds for all groups of order less than some $k$.  Then some prime $p$ divides the order of $A$, hence $A$ has a subgroup $B$ of order $p$.
\end{comment}

Since $A / A_f \cong A^f$ and $A / A_g \cong A^g$, we have $(A:A_f) = |A^f|$ and $(A:A_g) = |A^g|$.  Also, we can tower the indices in the subgroup chains
$$
0 \subseteq A^g \subseteq A_f \subseteq A
\hspace{1cm}
0 \subseteq A^f \subseteq A_g \subseteq A
$$
to obtain
\begin{align*}
|A^g|(A_f : A^g)(A:A_f) &= |A| = |A^f|(A_g : A^f)(A:A_g)
\\
\implies |A^g|(A_f : A^g)|A^f| &= |A^f|(A_g : A^f)|A^g|
\\
\implies
q(A) \cdot \frac{|A^g|}{|A^f|}\frac{|A^f|}{|A^g|} &= 1
\\
\implies q(A) &= 1
\end{align*}
\end{proof}

\end{enumerate}
\end{enumerate}
\end{document}































