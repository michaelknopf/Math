\documentclass[10pt]{article}
\usepackage[margin=1in]{geometry}
\addtolength{\oddsidemargin}{-.1in} 
\usepackage{amsmath,amsthm,amssymb}
\usepackage{bm}
\usepackage{enumitem}
\usepackage{array}
\usepackage{lipsum}
\usepackage[]{units}
\usepackage{relsize}
\usepackage{verbatim}
\usepackage{bbm}

\usepackage{tikz}
\usetikzlibrary  {positioning}
\usepackage{graphicx}
\usepackage{xfrac}
%\usetikzlibrary{cd}
\usepackage{tikz-cd}


\setenumerate{listparindent=\parindent}

\newcommand{\Q}{\mathbf{Q}}
\newcommand{\Z}{\mathbf{Z}}
\newcommand{\R}{\mathbf{R}}
\newcommand{\C}{\mathbf{C}}
\newcommand{\p}{\mathfrak{p}}
\newcommand{\q}{\mathfrak{q}}
\renewcommand{\a}{\mathfrak{a}}
\renewcommand{\b}{\mathfrak{b}}
\renewcommand{\c}{\mathfrak{c}}
\renewcommand{\o}{\mathfrak{o}}
\newcommand{\m}{\mathfrak{m}}
\newcommand{\gen}[1]{\langle #1 \rangle}
\DeclareMathOperator*{\dom}{dom}
\DeclareMathOperator*{\Aut}{Aut}
\DeclareMathOperator*{\Ann}{Ann}
\DeclareMathOperator*{\Tor}{Tor}
\DeclareMathOperator*{\Gal}{Gal}
\DeclareMathOperator*{\Hom}{Hom}
\DeclareMathOperator*{\End}{End}
\DeclareMathOperator*{\im}{Im}
\let\ker\relax
\DeclareMathOperator*{\ker}{Ker}
\DeclareMathOperator*{\spn}{span}
\DeclareMathOperator*{\Perm}{Perm}
\DeclareMathOperator*{\card}{card}
\DeclareMathOperator*{\Alt}{Alt}
\DeclareMathOperator*{\id}{id}
\DeclareMathOperator*{\Pic}{Pic}
\DeclareMathOperator*{\Maps}{Maps}
\renewcommand{\bar}{\overline}

\newtheorem*{lem}{Lemma}

\usepackage{fancyhdr} % Required for custom headers 
%\usepackage{lastpage} % Required to determine the last page for the footer

\pagestyle{fancy}
\lhead{Math 250A (HW 10)}
\chead{Michael Knopf (24457981)}
\rhead{November $12^\text{th}$, 2015}
\lfoot{}
\cfoot{}
\rfoot{}
%\rfoot{Page\ \thepage\ of\ \pageref{LastPage}}
\renewcommand\headrulewidth{0.4pt}
%\renewcommand\footrulewidth{0.4pt}

\begin{document}
\begin{enumerate}
\item[18.] Let $P(X) \in \Q[X]$ be a polynomial in one variable with rational coefficients.  It may happen that $P(n) \in \Z$ for all sufficiently large integers $n$ without necessarily $P$ having integer coefficients.
\begin{enumerate}
\item Give an example of this.
\begin{proof}
The value of $\binom{X}{2} = \frac12 X^2 + \frac12 X$ at $X = n \in \Z$ is the number of ways to choose $2$ elements from a collection of $n$, which must be integral.
\end{proof}
\item Assume that $P$ has the above property.  Prove that there are integers $c_0, c_1, \dots , c_r$ such that
$$
P(X) = c_r \binom{X}{r} + c_{r-1} \binom{X}{r-1} + \cdots + c_0
$$
where
$$
\binom{X}{r} = \frac{1}{r!}X(X-1) \cdots (X-r+1)
$$
is the binomial coefficient function.  In particular, $P(n) \in \Z$ for all $n$.  Thus we may call $P$ integral valued.

\begin{lem}
For a sequence $a_n$, let $\Delta$ be the difference operator $\Delta a_n = a_{n+1} - a_n$.  Then
$$
\Delta^k a_n = \sum_{i=0}^k \binom{k}{i} (-1)^i a_{n+k-i} \hspace{20pt} \text{and} \hspace{20pt} a_{n+k} = \sum_{i=0}^k \binom{k}{i}\Delta^i a_n.
$$
\end{lem}

\begin{proof}[Proof of lemma]

By induction.  Both clearly hold for $k=0$.  Suppose they hold for some $k$.
\begin{align*}
\Delta^{k+1} a_n &= \Delta^k a_{n+1} - \Delta^k a_n
\\
&= \sum_{i=0}^k \binom{k}{i}(-1)^i a_{n+k-i+1} + \sum_{i=0}^k \binom{k}{i}(-1)^{i+1} a_{n+k-i}
\\
&= \binom{k}{0} a_{n+k+1} + \sum_{i=0}^{k-1} \binom{k}{i+1}(-1)^{i+1} a_{n+k-i} + \sum_{i=0}^{k-1} \binom{k}{i}(-1)^{i+1} a_{n+k-i} + \binom{k}{k} (-1)^{k+1} a_n \\
&= \binom{k+1}{0} a_{n+k+1} + \sum_{i=0}^{k-1} \binom{k+1}{i+1}(-1)^{i+1} a_{n+k-i} + \binom{k+1}{k+1} (-1)^{k+1} a_n
\\
&= \binom{k+1}{0} a_{n+k+1} + \sum_{i=1}^{k} \binom{k+1}{i}(-1)^{i} a_{n+(k+1)-i} + \binom{k+1}{k+1} (-1)^{k+1} a_n
\\
&= \sum_{i=0}^{k+1} \binom{k+1}{i}(-1)^{i} a_{n+(k+1)-i}
\\
%%%%%%%%%%%%%%%%%%%%%%%%%%%%%%%%%%%%%%%%%%%%%%
a_{n+k+1} &= a_{n+k} + \Delta a_{n+k}
\\
&= \sum_{i=0}^k \binom{k}{i} \Delta^{i} a_n + \sum_{i=0}^k \binom{k}{i} \Delta^{i+1} a_n
\\
&= \binom{k}{0} \Delta^0 a_n + \sum_{i=0}^{k-1} \binom{k}{i+1} \Delta^{i+1} a_n + \sum_{i=0}^{k-1} \binom{k}{i} \Delta^{i+1} a_n + \binom{k}{k}\Delta^{k+1}a_n
\\
&= \binom{k+1}{0} \Delta^0 a_n + \sum_{i=0}^{k-1} \binom{k+1}{i+1} \Delta^{i+1} a_n + \binom{k+1}{k+1}\Delta^{k+1}a_n
\\
&= \sum_{i=0}^{k+1} \binom{k+1}{i} \Delta^{i} a_n
\end{align*}

\end{proof}

\begin{proof}
Suppose there is some $N$ such that $P(n) \in \Z$ for all $n > N$.  Define a sequence $a_n = p(N+d-n+1)$, where $d-1$ is the degree of the polynomial.  We want to show that $a_{d+1} = P(N) \in \Z$, since this will inductively imply that $P(n) \in \Z$ for all $n \in \Z$.  For any polynomial $f$ of degree $d-1$, the degree of $\Delta f(X)$ is strictly less than that of $f(X)$, and so $\Delta^{d} f(X)$ is the zero polynomial.  Therefore, by the lemma, we know that $\Delta^k a_1 \in \Z$ for all $k \in \{1, \dots , d\}$, since $\Delta^k a_1$ is a $\Z$-linear combination of $a_1, \dots , a_{k+1}$.  Since $a_{d+1}$ is a $\Z$-linear combination of $a_1, \dots , a_d$, we have $P(N) = a_{d+1} \in \Z$, as desired.  So we have $P(\Z) \subseteq \Z$.

Now, take the sequence $a_n = P(n)$.  We know the elements are integral and hence so are the $k$th differences.  For any $n \geq d$, we have by the lemma
$$
P(n) = a_n = \sum_{k=0}^n \binom{n}{k} \Delta^k P(0)  = \sum_{k=0}^{d-1} \binom{n}{k} \Delta^k P(0)
$$
because $\Delta^{k} P(0) = 0$ for $k \geq d$.  Taking any $d$ integers greater than $d$, we see that $P$ agrees at these points with the polynomial
$$
\sum_{k=0}^{d-1} \binom{X}{k} \Delta^k P(0).
$$
Since $d$ points define a degree $d-1$ polynomial, this must be $P(X)$, which thus takes the stated form.
\begin{comment}
Now, recognize that the binomial coefficients $\binom{X}{r}$, for nonnegative integers $r$, form a basis for $\Q[X]$ over $\Q$.  We can define a linear operator on $\Q[X]$ by $X^r \mapsto \binom{X}{r}$.  This is clearly injective because the degrees of the images of $X^n$ and $X^m$ differ whenever $n \neq m$.  Since $\binom{X}{r}$ has degree $r$, the matrix of this transformation is upper-triangular and has nonzero diagonal elements, thus the mapping is surjective (since if we restrict to the span of a finite subset of basis vectors, we obtain a surjective mapping).  So this is an invertible map, meaning this matrix represents a change of basis, hence the binomial coefficients are indeed a basis.

Next, let $\Delta$ be the first difference operator on $\Q[X]$, i.e. $\Delta(f(X)) = f(X+1) - f(X)$.  $\Delta$ is easily seen to be a linear map, and also $\deg(\Delta(f(X)) < \deg(f(X))$.  Therefore, if $P$ has degree $d$, then $\Delta^d(P(X))$ is constant.
\end{comment}
\end{proof}

\item Let $f:\Z \rightarrow \Z$ be a function.  Assume that there exists an integral valued polynomial $Q$ such that the difference function $\Delta f$ defined by
$$
(\Delta f)(n) = f(n) - f(n-1)
$$
is equal to $Q(n)$ for all $n$ sufficiently large positive.  Show that there exists an integral-valued polynomial $P$ such that $f(n) = P(n)$ for all $n$ sufficiently large.

\begin{proof}
There is some $N$ such that $\Delta f(n) = Q(n)$ for $n > N$.  Define a sequence $a_n = f(N+n)$.  Then $\Delta^{k+1} a_1 = \Delta^k Q(N+1)$ for all $n \in \mathbb{N}$.  By the lemma, we have $P(N+k) = a_{k} = \sum_{i=0}^k \binom{k}{i}\Delta^i a_1 = a_1 + \sum_{i=0}^k \binom{k}{i+1}\Delta^i Q(N+1)$ for all $k \in \mathbb{N}$.  Again, taking $n$ greater than the degree $d-1$ of $Q$, we get
$$
f(N+n) = a_1 + \sum_{i=0}^{d-1} \binom{n}{i+1}\Delta^i Q(N+1).
$$
Therefore, for large $n$, $f$ is the polynomial
$$
f(X) = a_1 + \sum_{i=0}^{d-1} \binom{X-N}{i+1}\Delta^i Q(N+1).
$$

\end{proof}

\end{enumerate}
\setcounter{enumi}{0}
\item Let $E = \Q(\alpha)$, where $\alpha$ is a root of the equation
$$
\alpha^3 + \alpha^2 + \alpha + 2 = 0.
$$
Express $(\alpha^2 + \alpha + 1)(\alpha^2+\alpha)$ and $(\alpha-1)^{-1}$ in the form
$$
a\alpha^2 + b\alpha + c
$$
with $a,b,c \in \Q$.

\begin{proof}
We employ the division algorithm:
$$
(\alpha^2 + \alpha + 1)(\alpha^2+\alpha) = (\alpha + 1)(\alpha^3 + \alpha^2 + \alpha + 2) + (-2x-2) = -2x-2.
$$
Next, we know that $\alpha - 1$ and $\alpha^3 + \alpha^2 + \alpha + 2$ are relatively prime, we can express $1$ as a linear combination of them:
$$
0 = \alpha^3 + \alpha^2 + \alpha + 2 = (\alpha^2 + 2\alpha + 3)(\alpha - 1) + 5
$$
$$
\implies (\alpha - 1)^{-1} = -\frac15 \alpha^2 - \frac25 \alpha - \frac35.
$$
\end{proof}

\item Let $E = F(\alpha)$ where $\alpha$ is algebraic over $F$, of odd degree.  Show that $E = F(\alpha^2)$.

\begin{proof}
Let $2n+1$ be the degree of $F(\alpha)$.  Then $\{\alpha, \alpha^2, \dots , \alpha^{2n+1}\}$ is a basis for $F(\alpha)$ over $F$.  $\{\alpha^2, \alpha^4, \dots , \alpha^{2n}\}$ is linearly independent, but cannot span $F(\alpha^2)$ since the degree of this extension must divide $2n+1$.  So there is some $\alpha^{2k+1} \in F(\alpha^2)$.  But then $\frac{\alpha^{2k+1]}}{\alpha^{2k}} = \alpha \in F(\alpha^2)$, so $F(\alpha^2) = F(\alpha)$ (since the other inclusion is trivial).
\end{proof}

\item Let $\alpha$ and $\beta$ be two elements which are algebraic over $F$.  Let $f(X) = \min_F(\alpha)$ and $g(X) = \min_F(\beta)$.  Suppose that $\deg f$ and $\deg g$ are relatively prime.  Show that $g$ is irreducible in the polynomial ring $F(\alpha)[X]$.

\begin{proof}

Note $\deg(g) = [F(\beta):F]$ divides $[F(\alpha,\beta):F(\beta)]F[(\beta):F] = [F(\alpha,\beta):F(\alpha)]F[(\alpha):F]$.  Since $\deg(g)$ and $F[(\alpha):F] = \deg(f)$ are relatively prime, $\deg(g)$ divides $[F(\alpha,\beta):F(\alpha)]$, which is the degree of $\min_{F(\alpha)}(\beta)$.  However, this polynomial divides $g$, and so $[F(\alpha,\beta):F(\alpha)] \leq \deg (g)$.  Therefore, we must have the equality $\deg(g) = \deg \min_{F(\alpha)}(\beta)$, and so $g = \min_{F(\alpha)}$.  Thus $g$ is irreducible over $F(\alpha)$.

\end{proof}

\item Let $\alpha$ be the real positive fourth root of $2$.  Find all the intermediate fields in the extension $\Q(\alpha)$ of $\Q$.

\begin{proof}
The only intermediate field is $\Q(\sqrt{2})$.  The minimal polynomial for $\alpha$ is $X^4 - 2$, and $\Q(\alpha)$ is an extension of degree $4$.  An intermediate extension $E$ must then have degree 2, and the minimal polynomial $g(X)$ of $\alpha$ over $E$ must divide $X^4 - 2$.  Clearly $g$ has real coefficients, and so the only possibility is $g(X) = X^2 - \sqrt{2}$.  So any intermediate extension must contain $\Q(\sqrt{2})$.  Since this has degree 2, it \emph{is} the only intermediate extension.
\end{proof}

\item If $\alpha$ is a complex root of $X^6 + X^3 + 1$, find all the homomorphisms $\sigma: \Q(\alpha) \rightarrow \C$.

\begin{proof}
Because $\alpha$ generates the extension, its image will determine $\sigma$.  Since $\sigma(\alpha)^6 + \sigma(\alpha)^3 + 1 = 0$, the only possible images for $\alpha$ are the roots of this polynomial.  Furthermore, and such mapping gives a homomorphism, since they can be viewed as homomorphisms induced by those from $\Q(X)$: $\Q(\alpha) \cong \Q[X] / (X^6 + X^3 + 1)$, and such a $\sigma$ contains $(X^6 + X^3 + 1)$ in its kernel, hence it factors through $\Q(\alpha)$.

Suppose $\beta$ is a root.  $\beta^3$ is a root of $X^2 + X + 1$ and so is a primitive cube root of unity, thus $\beta$ must be a primitive $9$th root of unity.  So the possible maps are $\alpha \mapsto e^{k\frac{2\pi i}{9}}$ where $k \in \{1,2,4,5,7,8\}$.
\end{proof}

\item Show that $\sqrt{2} + \sqrt{3}$ is algebraic over $\Q$, of degree $4$.

\begin{proof}
The polynomial $f(X) = X^4 - 10X^2 + 1$ has $\sqrt{2} + \sqrt{3}$ as a root.  In fact, its four roots are $\pm \sqrt{3} \pm \sqrt{2}$.  If $f$ were reducible over $\Q$, then some product of two of these roots would have to be rational, which is obviously false.  So $f$ is the minimal polynomial of $\sqrt{2} + \sqrt{3}$, meaning this number has degree $4$ over $\Q$.
\end{proof}

\item Let $E,F$ be two finite extensions of a field $k$, contained in a larger field $K$.  Show that
$$
[EF:k] \leq [E:k][F:k].
$$
If $[E:k]$ and $[F:k]$ are relatively prime, show that one has an equality sign in the above relation.

\begin{proof}
Let $\{\alpha_1, \dots, \alpha_m\}$ and $\{\beta_1, \dots , \beta_n\}$ be bases for $E$ and $F$ over $K$, respectively.  Then $EF = E(\beta_1, \dots , \beta_m)$, so $[EF:E] \leq m = [F:K]$.  Therefore, the inequality holds by the tower property.  Now, since $[EF:E][E:k] = [EF:F][F:k]$, if $[E:k]$ and $[F:k]$ are relatively prime then $[F:k]$ divides $[EF:E]$.  Therefore, the inequality $[EF:E] \leq [F:K]$ must be an equality, and so we have $[EF:k] = [E:k][F:k]$.
\end{proof}

\end{enumerate}
\end{document}


















