\documentclass[10pt]{article}
\usepackage[margin=1in]{geometry}
\addtolength{\oddsidemargin}{-.1in} 
\usepackage{amsmath,amsthm,amssymb}
\usepackage{bm}
\usepackage{enumitem}
\usepackage{array}
\usepackage{lipsum}
\usepackage[]{units}
\usepackage{relsize}
\usepackage{verbatim}
\usepackage{bbm}

\usepackage{tikz}
\usetikzlibrary  {positioning}
\usepackage{graphicx}
\usepackage{xfrac}
\usetikzlibrary{cd}


\setenumerate{listparindent=\parindent}

\newcommand{\Q}{\mathbf{Q}}
\newcommand{\Z}{\mathbf{Z}}
\newcommand{\R}{\mathbf{R}}
\newcommand{\C}{\mathbf{C}}
\newcommand{\p}{\mathfrak{p}}
\newcommand{\q}{\mathfrak{q}}
\renewcommand{\a}{\mathfrak{a}}
\renewcommand{\b}{\mathfrak{b}}
\renewcommand{\c}{\mathfrak{c}}
\renewcommand{\o}{\mathfrak{o}}
\newcommand{\m}{\mathfrak{m}}
\newcommand{\gen}[1]{\langle #1 \rangle}
\DeclareMathOperator*{\dom}{dom}
\DeclareMathOperator*{\Aut}{Aut}
\DeclareMathOperator*{\Ann}{Ann}
\DeclareMathOperator*{\Tor}{Tor}
\DeclareMathOperator*{\Gal}{Gal}
\DeclareMathOperator*{\Hom}{Hom}
\DeclareMathOperator*{\End}{End}
\DeclareMathOperator*{\im}{Im}
\let\ker\relax
\DeclareMathOperator*{\ker}{Ker}
\DeclareMathOperator*{\spn}{span}
\DeclareMathOperator*{\Perm}{Perm}
\DeclareMathOperator*{\card}{card}
\DeclareMathOperator*{\Alt}{Alt}
\DeclareMathOperator*{\id}{id}
\DeclareMathOperator*{\Pic}{Pic}
\DeclareMathOperator*{\Maps}{Maps}
\renewcommand{\bar}{\overline}

\newtheorem*{lem}{Lemma}

\usepackage{fancyhdr} % Required for custom headers 
%\usepackage{lastpage} % Required to determine the last page for the footer

\pagestyle{fancy}
\lhead{Math 250A (HW 9)}
\chead{Michael Knopf (24457981)}
\rhead{November $5^\text{th}$, 2015}
\lfoot{}
\cfoot{}
\rfoot{}
%\rfoot{Page\ \thepage\ of\ \pageref{LastPage}}
\renewcommand\headrulewidth{0.4pt}
%\renewcommand\footrulewidth{0.4pt}

\begin{document}
\begin{enumerate}
\item[5.]
\begin{enumerate}
\item[(a)] Show that the polynomials $X^4 + 1$ and $X^6 + X^3 + 1$ are irreducible over the rational numbers.

\begin{proof}
It suffices to show that $f(X+1)$ is irreducible.   For if $f(X) = g(X)h(X)$ with $\deg g \geq 1$ or $\deg h \geq 1$, then we would have $f(X+1) = g(X+1)h(X+1)$, and this transformation has preserved the degrees of $g$ and $h$.  Since $f(X+1) = X^4 + 4X^3 + 6X^2 + 4X + 2$, applying Eisenstein's Criterion with $p = 2$ suffices.

Applying the same logic to $f(X) = X^6 + X^3 + 1$, we have $f(X+1) = X^6+6 X^5+15 X^4+21 X^3+18 X^2+9 X+3$, so $p=3$ satisfies Eisenstein's Criterion.
\end{proof}

\item[(c)] Show that the polynomial in two variables $X^2 + Y^2 - 1$ is irreducible over the rational numbers.  Is it irreducible over the complex numbers?

\begin{proof}
Consider the polynomial as an element of $\Z[X][Y]$.  Its image under the homomorphism $Y \mapsto 2$ is $X^2 + 3$, which is nonzero and of the same degree.  By Eisenstein's Criterion, $X^2 + 3$ is irreducible over $\Q$, thus the original polynomial was irreducible over $\Z[X]$.  So it is also irreducible over $\Q$.

Suppose that $f(X,Y) = X^2 + Y^2 - 1$ were reducible over $\C$.  Then $f = gh$ for some $g,h \in \C[X,Y]$, where neither of $g$ or $h$ is a unit.  The units of $\C[X,Y]$ are just the constant polynomials, so neither $g$ nor $h$ is constant.  It must then be that $g$ and $h$ both have degree $1$ when considered as polynomials WLOG in $Y$ over $\C[X]$.  So we have a factorization
$$
Y^2 + (X^2 - 1) = (Y + p(X))(Y - p(X))
$$
for some $p(X) \in \C[X]$ such that $p(X)^2 = X^2 - 1$.  (Write $Y^2 + (X^2 - 1) = (Y + a(X))(Y + b(X))$, since we may assume the coefficients of $Y$ are $1$.  Expanding shows the factorization must take this form.)  $\C$ is factorial, so the only factorization of $X^2 - 1$ is $(X+1)(X-1)$ (modulo units and permutation), hence no such $p(X)$ can exist.  So $f$ must be irreducible over $\C$.
\end{proof}

\end{enumerate}
\item[7.]
\begin{enumerate}
\item[(a)] Let $k$ be a finite field with $q = p^m$ elements.  let $f(X_1, \dots , X_n)$ be a polynomial in $k[X]$ of degree $d$ and assume $f(0,\dots,0) = 0$.  An element $(a_1, \dots , a_n) \in k^{(n)}$ such that $f(a) = 0$ is called a zero of $f$.  If $n>d$, show that $f$ has at least one other zero in $k^{(n)}$.

\begin{proof}
Consider the polynomials $F(X) = 1 - f(X)^{q-1}$ and $G(X) = \prod_i (1-X_i^{q-1})$, which have degrees $d(q-1)$ and $n(q-1)$, respectively.  These both induce the indicator function that is $1$ at $x = 0$ and $0$ elsewhere.  Let $\bar{F}(X)$ be the reduced polynomial belonging to $F(X)$.  Then the degree of $\bar{F}(X) - G(X)$ in each variable is $<q$ and this polynomial induces the $0$ function on $k^{(n)}$.  Thus, $\bar{F}(X) = G(X)$.  Since $\bar{F}(X)$ is the reduced version of $F(X)$, we have $n(q-1) = \deg G = \deg \bar{F} \leq \deg F = d(q-1)$.  Since $q \geq 2$, this contradicts that $n > d$.
\end{proof}

\item[(b)] Refine the above results by proving that the number $N$ of zeros of $f$ in $k^{(n)}$ is $\equiv 0 \pmod{p}$.

\begin{proof}
Define a function on the nonnegative integers by $\psi(i) = \sum\limits_{x \in k} x^i$.  Clearly, $\psi(0) = 0$, so assume $i > 0$.  If $q-1 \mid i$ then we have $$\psi(i) = 0 + \sum\limits_{x \in k, x \neq 0} x^{i\pmod{q-1}} = 0 + \sum\limits_{x \in k, x \neq 0} 1 = q - 1 = -1.$$  Otherwise, $i > 0$.  So some $g$ is a generator of $k^{\times}$, hence $g^i \neq 1$ and multiplication by $g$ is an isomorphism on $k$, so
$$
\psi(i) = \sum_{x \in k} x^i = \sum_{x \in k}(gx)^i = g^i \psi(i)
$$
therefore $\psi(i) = 0$ (since $g^i \neq 1$).  Now, define $\Psi(i_1, \dots , i_n) = \sum\limits_{x \in k^{(n)}} x_1^{i_1} \cdots x_n^{i_n}$.  Then by induction we have
$$
\Psi(i_1, \dots , i_n) = \sum\limits_{x_n \in k} x_n^{i_n} \left(\sum_{x_1 \in k} \cdots \sum_{x_{n-1} \in k} x_1^{i_1} \cdots x_{n-1}^{i_{n-1}} \right)
$$
$$=\sum\limits_{x_n \in k} x_n^{i_n} \psi(i_1) \cdots \psi(i_{n-1}) = \psi(i_1) \cdots \psi(i_n).
$$

The number $N$ of zeros (mod $p$) that $f(x)$ has in $k^{(n)}$ can be expressed as $N = \sum\limits_{x \in k^{(n)}}(1 - f(x)^{q-1})$.  This is because a given term is $1$ if $f(x) = 0$ and $0$ otherwise, and adding $1$s and $0$s in $k$ amounts to adding them in its prime subfield $\Z_p$.  $1-f(x)^{q-1}$ is a sum of terms of the form $a_ix_1^{i_1} \cdots x_n^{i_n}$ with $\sum\limits_j i_j \leq d(q-1) < n(q-1)$.  Thus, each term in $\sum\limits_{x \in k^{(n)}} 1 - f(x)^{q-1}$ contains a factor of $\Psi(i_1, \dots , i_n)$, and furthermore some $i_j < q-1$.  Therefore, since either $i_j = 0$ or $q-1 \nmid i_j$ for this $j$, we know $\psi(i_j) = 0$ and thus, by the result of the last paragraph, the whole term is $0$.  So this entire sum is $0$, meaning $p \mid N$.
\end{proof}

\item[(c)] Extend Chevalley's theorem to $r$ polynomials $f_1, \dots , f_r$ of degrees $d_1 , \dots , d_r$ respectively, in $n$ variables.  If they have no constant term and $n > \sum d_i$, show that they have a non-trivial common zero.

\begin{proof}
Suppose the sum of the degrees is less than $n$.  Then the number $N$ of common zeros is congruent to $\prod (1-f_i(x)^{q-1}) \pmod{p}$, since $x$ is a common zero if and only if every factor equals $1$.  We also have
$$
\deg \prod (1-f_i(X)^{q-1}) = \sum_1^r \deg f_i(X)^{q-1} = (q-1)\sum_1^r d_i < n(q-1)
$$
therefore every term contains some variable to a power less than $q-1$, so the entire sum is $0$.  Thus $p \mid N$.

If every polynomial has no constant term, then $0$ is a common zero.  But $1$ is not divisible by any prime, so there must be another zero, which is then non-trivial.
\end{proof}

\item[(d)] Show that an arbitrary function $f:k^{(n)} \rightarrow k$ can be represented by a polynomial. (As before, $k$ is a finite field.)

\begin{proof}
Recall that every polynomial over $k$ has a unique reduced polynomial which gives the same function.  Since the multiplicative group of nonzero elements of $k$ is cyclic of order $q-1$, we know that $X^v$ agrees, as a function, with $X^{v \pmod{q-1}}$ as long as $v \neq q-1$.  If $v = q-1$, then the two functions agree everywhere except for at $0$, however this single point distinguishes them.  So a set of representatives for the distinct polynomial functions on $k^{(n)}$ is the set of polynomials whose degree $d$ in each variable satisfies $0 \leq d \leq q-1$ (by Corollary 1.8, we know that no two of these polynomials give the same function).

This means that there are $q^{q^n}$ distinct polynomial functions $k^{(n)} \rightarrow k$.  This is also the number of functions $k^{(n)} \rightarrow k$.  So each of these functions must be given by exactly one of these polynomials.
\end{proof}

\end{enumerate}

\item[8.] Let $A$ be a commutative entire ring and $X$ a variable over $A$.  Let $a,b \in A$ and assume that $a$ is a unit in $A$.  Show that the map $X \mapsto aX + b$ extends to a unique automorphism of $A[X]$ inducing the identity on $A$.  What is the inverse automorphism?

\begin{proof}
If $\varphi$ is constrained to fix $A$, then it obviously extends to the unique homomorphism $$c_nX^n + \cdots + c_1X + c_0 \mapsto c_n(aX + b)^n + \cdots + c_1(aX + b) + c_0.$$
This map has an inverse, which is $X \mapsto a^{-1}X - a^{-1}b$ (being of the same form, this also extends to a unique homomorphism fixing $A$).  The composition of these maps clearly gives the identity on $X$ and on $A$, and so the composition is the unique extension of $X \mapsto X$ fixing $A$, which is the identity.  So this is an automorphism of $A[X]$.
\end{proof}

\item[9.] Show that every automorphism of $A[X]$ inducing the identity on $A$ is of the type described in Exercise 8.

\begin{proof}
Let $\varphi$ be an automorphism of $A[X]$ inducing the identity on $A$.  Then for any polynomial we have
$$
c_nX^n + \cdots + c_1X + c_0 \mapsto c_np(X)^n + \cdots + c_1p(X) + c_0
$$
where $p(X)$ is the image of $X$.  If $\deg p > 1$ then for all nonconstant polynomials $f$ we will have $\deg \varphi f > \deg f$, hence $X$ is not in the image of $\varphi$.  If $\deg p < 1$ then obviously $\varphi$ is not injective, since it fixes $A$.  So $p(X) = aX + b$ for some $a,b \in A$.

$\varphi^{-1}$ satisfies the same hypothesis, and thus must be of the form $cX + d$.  Since $c(aX+b)+d = acX + cb + d = X$, we must have $ac = 1$ and $d = -cb$.  Thus the inverse map is given by $X \mapsto a^{-1}X + a^{-1}b$, as claimed.
\end{proof}

\item[10.] Let $K$ be a field, and $K(X)$ the quotient field of $K[X]$.  Show that every automorphism of $K(X)$ which induces the identity on $K$ is of type
$$
X \mapsto \frac{aX+b}{cX+d}
$$
with $a,b,c,d \in K$ such that $(aX+b)/(cX+d)$ is not an element of $K$, or equivalently, $ad-bc \neq 0$.

\begin{proof}
Let $\varphi$ be an automorphism fixing $K$.  Let $f(X) = \frac{p(X)}{q(X)}$ be the image of $X$, where $p(X)$ and $q(X)$ are relatively prime.  Then $X$ is a root of $g(Y) = q(Y)f(X) - p(Y) \in K(f(X))[Y]$, hence $X$ is algebraic over $K(f(X))$.  This polynomial is also contained in the subring $K[f(X)][Y] = K[Y][f(X)]$, and it is irreducible in $K[Y][f(X)]$ since it is linear in $f(X)$.  Thus it is irreducible over $K[f(X)][Y]$.  But a polynomial is irreducible over a UFD if and only if it is irreducible over its field of fractions, thus $g$ is irreducible in $K(f(X)[Y]$.  Hence it is the minimal polynomial for $X$ over $K(f(X))$, and so its degree (in $Y$) is the degree of the extension $K(f(X))(X)$ over $K(f(X))$, which is thus $\max\{\deg(p),\deg(q)\}$ (the degrees taken in $Y$).  But $K(f(X))(X) = K(X)$ because $f(X) \in K(X)$, therefore
$$
[K(X):K(f(X))] = \max\{\deg(p),\deg(q)\}.
$$

Since $\varphi$ is surjective, and its image $K(\varphi(X)) = K(f(X))$ must equal $K(X)$.  Therefore, the degree of this extension is $1$, and so both $p$ and $q$ are either linear or constant.  Finally, if $ad - bc = 0$ then we have
$$
\frac{c}{a}\varphi(X) = \varphi(\frac{c}{a}X) = \frac{c}{a}\frac{aX+b}{cX+d} = \frac{acX+bc}{acX+ad} = 1
$$
and so $\varphi(X) = \frac{a}{c}$, contradicting the injectivity of $\varphi$.  So $ad-bc \neq 0$.

Next, we will show that all maps of this form are automorphisms.  Let $\varphi$ be the unique extension of $X \mapsto \frac{aX+b}{cX+d}$, fixing $K$, to a homomorphism $K[X] \rightarrow K(X)$.  It takes $p(X)$ to $p(\frac{aX+b}{cX+d})$.  If $\varphi(p(X)) = p(\frac{aX+b}{cX+d}) = 0$, then $\frac{aX+b}{cX+d}$ is algebraic over $K$.  Now note that
$$
\frac{c}{bc-ad}\left(\frac{aX+b}{cX+d} - \frac{a}{c}\right) = \frac{1}{cX+d}
$$
because $ad - bc \neq 0$, therefore $K(\frac{aX+b}{cX+d}) = K(\frac{1}{cX+d}) = K(cX+d) = K(X)$ is algebraic, a contradiction.  So $\varphi$ is injective.

Since $\varphi$ is injective, it extends to a unique endomorphism $K(X) \rightarrow K(X)$ (since no denominator can map to $0$).  Since a field homomorphism is either injective or trivial, this map must be injective.  By the discussion in the proof of the other direction, we have
$$
[K(X):\im] = [K(X):K(\frac{aX+b}{cX+d})] = \max\{\deg(aX+b),\deg(cX+d)\} = 1
$$
since if both polynomials were constant we would have $ad = bd = 0$.  Therefore, the image of the map is all of $K(X)$, hence it is surjective.


\end{proof}
\end{enumerate}
\end{document}


















