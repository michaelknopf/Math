\documentclass[10pt]{article}
\usepackage[margin=1in]{geometry}
%\addtolength{\oddsidemargin}{-.1in} 
\usepackage{amsmath,amsthm,amssymb}
\usepackage{bm}
\usepackage{enumitem}
\usepackage{array}
\usepackage{lipsum}
\usepackage[]{units}
\usepackage{relsize}
\usepackage{verbatim}
\usepackage{bbm}

\usepackage{tikz}
\usetikzlibrary{positioning}
\usepackage{graphicx}
\usepackage{xfrac}

\setenumerate{listparindent=\parindent}

\newcommand{\Q}{\mathbb{Q}}
\newcommand{\Z}{\mathbb{Z}}
\newcommand{\R}{\mathbb{R}}
\DeclareMathOperator*{\dom}{dom}
\DeclareMathOperator*{\Aut}{Aut}
\DeclareMathOperator*{\Ann}{Ann}
\DeclareMathOperator*{\Tor}{Tor}
\DeclareMathOperator*{\Gal}{Gal}
\DeclareMathOperator*{\Hom}{Hom}
\DeclareMathOperator*{\End}{End}
\DeclareMathOperator*{\im}{Im}
\renewcommand{\bar}{\overline}

\usepackage{fancyhdr} % Required for custom headers 
%\usepackage{lastpage} % Required to determine the last page for the footer

\pagestyle{fancy}
\lhead{Math 250A (HW 1)}
\chead{Michael Knopf (24457981)}
\rhead{September $3^\text{rd}$, 2015}
\lfoot{}
\cfoot{}
\rfoot{}
%\rfoot{Page\ \thepage\ of\ \pageref{LastPage}}
\renewcommand\headrulewidth{0.4pt}
%\renewcommand\footrulewidth{0.4pt}

\begin{document}

\begin{enumerate}
\item[4.] Let $H,K$ be subgroups of a finite group $G$ with $K \subset N_H$.  Show that $\#(HK) = \dfrac{\#(H)\#(K)}{\#(H\cap K)}$.

\begin{proof}
First, note that $HK$ is a subgroup of $G$.  For $h_1,h_2 \in H, k_1,k_2 \in K$, we have $$(h_1k_1)(h_2k_2) = ((h_1k_1h_1^{-1})(h_1h_2))k_2 \in HK$$ because $h_1 k_1 h_1^{-1} \in H$.  So $HK$ is closed under multiplication.  Also,
$$
(hk)^{-1} = k^{-1}h^{-1} = (k^{-1}h^{-1})(kk^{-1}) = (k^{-1}h^{-1}k)k^{-1} \in HK
$$
since $k^{-1}h^{-1}k \in H$.  So $HK$ is closed under inverses.  Clearly, $1 = 1 \cdot 1 \in HK$, thus $HK$ is a subgroup.

Next, observe that $H$ is normal in $HK$ and $H \cap K$ is normal in $K$ (the intersection of subgroups is always a subgroup): for $h, h' \in H$, $k \in k$, and $g \in H \cap K$, we have
$$
(hk)h'(hk)^{-1} = h(kh'k^{-1})h^{-1} \in H
$$
since $k h' k^{-1} \in H$; $kgk^{-1} \in K$ because $g \in K$, and $kgk^{-1} \in H$ because $g \in H$ (and $K$ normalizes $H$).

Now, we can define a map $\varphi: K / H \cap K \rightarrow HK / H$ by $k H\cap K \mapsto kH$.  $\varphi$ is a well-defined because if $k' \in k H \cap K$ then $k' = kh$ for some $h$, thus
$$
\varphi(k' H \cap K) = k'H = khH = kH = \varphi(k H \cap K).
$$
$\varphi$ is also a homomorphism, since for $k \in K$ we have $$\varphi((k_1H\cap K)(k_2H \cap K)) = \varphi(k_1k_2 H \cap K) = k_1k_2 H = k_1 H k_2 H = \varphi(k_1H\cap K) \varphi(k_2H \cap K).$$

If $k \in K$ is such that $k H\cap K \in \ker (\varphi)$, then $\varphi(k H\cap K) = kH = H$; so $k$ is in $H$, and therefore also in $H \cap K$.  So $\varphi$ is injective.

Finally, given any $hk \in HK$, we have
$$
(hk)H = (hH)(kH) = H(kH) = kH = \varphi(k H\cap K)
$$
thus $\varphi$ is surjective.

Therefore, $\varphi$ is an isomorphism, and so $K / H \cap K \cong HK / H$.  Since $G$ is finite, Lagrange's Theorem gives that $\#(HK) = \dfrac{\#(H)\#(K)}{\#(H\cap K)}$.
\end{proof}

\item[7.] Let $G$ be a group such that $\Aut (G)$ is cyclic.  Prove that $G$ is abelian.

\newcommand{\cc}{$\textbf{c}$}

\begin{proof}

Denote by $\textbf{c}_x$ the inner automorphism $y \mapsto xyx^{-1}$.  Lang mentions that the association $x \mapsto \textbf{c}_x$ is a homomorphism of $G$ into its automorphism group (since $\cc_x \circ \cc_y (g) = (xy)g(xy)^{-1} = \cc_{xy}(g)$).  Clearly, the kernel of this map is the center $Z$ of $G$, and its image is the subgroup $I$ of inner automorphisms of $G$.  So $G / Z \cong I$.

Subgroups of cyclic groups are cyclic, so $I$ is cyclic and thus so is $G / Z$.  Let $g$ be such that $gZ$ generates $G / Z$.  Then for any $x,y \in G$ there exist $m,n \in \Z$ and $z,w \in Z$ such that $x = g^mz$ and $y = g^n w$.  So
$$
xy = (g^m z)(g^n w) = (g^m g^n) (z w) = (g^n g^m) (w z) = (g^n w) (g^m z) = yx
$$
thus $G$ is abelian.

%Then $g$ and $Z$ together generate all of $G$.  Since the centralizer of $g$ in $G$ is a subgroup, and it obviously must contain both $g$ and all of $Z$, it must equal all of $G$, and so $g$ is actually contained in the center.  But then $g^n Z = Z$ for all $n \in \Z$, and so $G / Z = \{g^n Z \mid n \in \Z \} = \{Z\}$ is trivial.  Thus $Z = G$, hence $G$ is abelian.

\end{proof}

\let\cc\undefined

\item[9.]\begin{enumerate}
\item Let $G$ be a group and $H$ a subgroup of finite index.  Show that there exists a normal subgroup $N$ of $G$ contained in $H$ and also of finite index.

\begin{proof}

For each $g \in G$, let $T_g: G / H \rightarrow G/H$ denote the translation $S \mapsto gS$ for any coset $S$ of $H$ in $G$.  First, we will show that $T_g$ is a permutation of $G / H$.  Let $x,y \in G$.  If $T_g(xH) = T_g(yH)$, then $gxH = gyH$.  So for all $h \in H$, there exists some $h' \in H$ such that $gxh = gyh'$, hence $xh = xh'$.  So $xH \subseteq yH$.  The symmetric argument shows that $yH \subseteq xH$, thus $xH = yH$ and so $T_g$ is injective.  Also, given any $x \in G$, we have $xH = g(g^{-1}x)H = T_g(g^{-1}x)$, thus $T_g$ is surjective as well.  So $T_g \in S_{G/H}$ (the group of permutations of $G/H$).

Define $\varphi: G \rightarrow S_{G/H}$ by $g \mapsto T_g$.  $\varphi$ is a homomorphism, since for any $x,y,z \in G$ we have $$\varphi(xy)(zH) = T_{xy}(zH) = xyzH = x(yzH) = T_x \circ T_y (zH) = (\varphi(x)\circ \varphi(y))(zH).$$
If $g \in \ker \varphi$, then $T_g(H) = gH = H$, thus $g \in H$.  So $\ker \varphi \subseteq H$.  Therefore, $N = \ker \varphi$ is a normal subgroup of $G$ contained in $H$.  Also, $G / N \cong \im \varphi \subseteq S_{G/H}$, and $\#(S_{G/H}) = \#(G:H)!$.  Thus $\#(G:N) \leq \#(G:H)!$, hence $N$ has finite index in $G$.

\end{proof}

\item Let $G$ be a group and let $H_1, H_2$ be subgroups of finite index.  Prove that $H_1 \cap H_2$ has finite index.

\begin{proof}

Consider the map (of sets) $G / H_1 \cap H_2 \hookrightarrow G / H_1 \times G / H_2$ given by $g H_1 \cap H_2 \mapsto (g H_1, g H_2)$.  If $g H_1 \cap H_2 = h H_1 \cap H_2$, then $g^{-1}h \in H_1 \cap H_2$, so $gH_1 = hH_1$ and $gH_2 = hH_2$, thus the map is well-defined.  Similarly, if $(gH_1, g H_2) = (hH_1, hH_2)$ for some $g,h \in G$, then $g^{-1}h \in H_1$ and $g^{-1}h \in H_2$, hence $g H_1 \cap H_2 = h H_1 \cap H_2$.  So the map is injective.  Therefore, $(G : H_1 \cap H_2)$ must be finite, since the cardinality of the codomain is finite by assumption.

\begin{comment}
By the previous result, there exist normal subgroups $N_1 \subseteq H_1$ and $N_2 \subseteq H_2$ of finite index in $G$.  Consider the map $G / N_1 \cap N_2 \hookrightarrow G / N_1 \times G / N_2$ given by $g N_1 \cap N_2 \mapsto (g N_1, g N_2)$.

If $g N_1 \cap N_2 = h N_1 \cap N_2$, then $g^{-1}h \in N_1 \cap N_2$, so $gN_1 = hN_1$ and $gN_2 = hN_2$, thus the map is well-defined.  Similarly, if $(gN_1, g N_2) = (hN_1, hN_2)$ for some $g,h \in G$, then $g^{-1}h \in N_1$ and $g^{-1}h \in N_2$, hence $g N_1 \cap N_2 = h N_1 \cap N_2$.  So the map is injective.%  The map is clearly a homomorphism, since $(ghN_1, ghN_2) = (gN_1,gN_2)(hN_1,hN_2)$.

Therefore, $(G : N_1 \cap N_2)$ must be finite, since the cardinality of the codomain is finite by assumption.  Since $N_1 \cap N_2 \subseteq H_1 \cap H_2$, $(G: H_1 \cap H_2) \leq (G: N_1 \cap N_2)$.  So $H_1 \cap H_2$ has finite index, as well.
\end{comment}

\end{proof}

\end{enumerate}

\item[15.] Let $G$ be a finite group operating on a finite set $S$ with $\#(S) \geq 2$.  Assume that there is only one orbit.  Prove that there exists an element $x \in G$ which has no fixed point, i.e. $xs \neq s$ for all $s \in S$.

\begin{proof}
For any $s \in S$, we have $\dfrac{\# G}{\# G_s} = (G : G_s) = \# (Gs) = \# S$, so $\# G_s = \dfrac{\# G}{\# S}$.  Suppose, for a contradiction, that each element $x \in G$ has a fixed point.  Each stabilizer contains the identity, so by ``inclusion exclusion" we have
$$
\# G = \# S \dfrac{\# G}{\# S	} = \sum_{s \in S} \dfrac{\#G}{\#S} = \sum_{s \in S} \# G_s \geq \# S + (\#G - 1).
$$
The inequality is justified by the fact that, in summing the sizes of all the stabilizers, we count the identity element $\# S$ times and count every other element of $G$ at least once, since each element stabilizes some element.  This implies that $\# S \leq 1$, a contradiction.
\end{proof}

\item[16.] Let $H$ be a proper subgroup of a finite group $G$.  Show that $G$ is not the union of all the conjugates of $H$.

\begin{proof}
Let $g \in G$.  For any $gh \in gH$, we have $(gh)H(gh)^{-1} = g(hHh^{-1})g^{-1} = gHg^{-1}$.  Therefore, we have a surjection from the cosets of $H$ in $G$ onto the set of distinct conjugates of $H$ in $G$, hence there are at most $(G:H) = \dfrac{\# G}{\# H}$ distinct conjugates of $H$.  Each of these conjugates has size $\# H$ (since conjugation is a bijection) and contains the identity, so by ``inclusion-exclusion" we have
$$
\# \left( \bigcup_{g \in G} gHg^{-1} \right) \leq \dfrac{\# G}{\# H}(\# H - 1) + 1 = \# G - (G : H) + 1 \leq \# G - 2 + 1 = \# G - 1.
$$
The second inequality follows from the fact that $H$ is a proper subgroup, so it has index $\geq 2$.  Therefore, the union is strictly smaller than $G$.
\end{proof}

\item[17.] Let $X,Y$ be finite sets and let $C$ be a subset of $X \times Y$.  For $x \in X$, let $\varphi(x)$ be the number of elements $y \in Y$ such that $(x,y) \in C$.  Verify that $\#(C) = \sum\limits_{x \in X} \varphi(x)$.

\begin{proof}

Let $\mathbbm{1}:X \times Y \rightarrow \{0,1\}$ be given by $\mathbbm{1}(x,y) = \begin{cases} 1 & (x,y) \in C \\ 0 & \text{else} \end{cases}$.  Then
$$
\#(C) = \sum_{(x,y) \in X \times Y} \mathbbm{1}(x,y) = \sum_{x \in X} \left(\sum_{y \in Y} \mathbbm{1}(x,y) \right) = \sum_{x \in X} \varphi(x).
$$

\end{proof}

\item[18.] Let $S,T$ be finite sets.  Show that $\#(T^S) = (\# T) ^{\#(S)}$.

\begin{proof}
A function $S \rightarrow T$ is defined uniquely by the image of each point in $S$.  For each of the $\#(S)$ points in $S$, there are $\#(T)$ possible images.  So the number of functions is $\prod\limits_{s \in S} \#(T) = \#(T)^{\#(S)}$.
\end{proof}

\item[19.]
Let $G$ be a finite group operating on a finite set $S$.  
\begin{enumerate}
\item For each $s \in S$ show that $\sum\limits_{t \in Gs} \dfrac{1}{\#(Gt)} = 1.$
\begin{proof}
For all $t \in Gs$, $Gt = Gs$, since the orbits partition $S$.  So $\sum\limits_{t \in Gs} \dfrac{1}{\#(Gt)} = \sum\limits_{t \in Gs} \dfrac{1}{\#(Gs)} = \#(Gs) \dfrac{1}{\#(Gs)} = 1.$
\end{proof}
\item For each $x \in G$ define $f(x)$ to be the number of elements $s \in S$ such that $xs = s$.  Prove that the number of orbits of $G$ in $S$ is equal to $ \dfrac{1}{\#(G)} \sum\limits_{x \in G} f(x).$

\begin{proof}
Let $\mathbbm{1}: G \times S \rightarrow \{0,1\}$ be given by $\mathbbm{1}(x,s) = \begin{cases} 1 & xs = s \\ 0 & \text{else} \end{cases}$.  Since both $S$ and $G$ are finite,
$$\sum\limits_{x \in G} f(x)
= \sum\limits_{x \in G} \sum\limits_{s \in S} \mathbbm{1}(x,s)
= \sum\limits_{s \in S} \sum\limits_{x \in G} \mathbbm{1}(x,s)
= \sum\limits_{s \in S} \#(G_s).$$
Since $G$ is finite, $\#(Gs) = (G : G_s) = \frac{\#(G)}{\#(G_s)}$, so $\#(G_s) = \frac{\#(G)}{\#(Gs)}$.  This gives
$$
\dfrac{1}{\#(G)} \sum\limits_{x \in G} f(x) = \dfrac{1}{\#(G)} \sum\limits_{s \in S} \#(G_s) = \dfrac{1}{\#(G)} \sum\limits_{s \in S} \frac{\#(G)}{\#(Gs)} = \sum\limits_{s \in S} \frac{1}{\#(Gs)}
$$
$$
= \sum\limits_{\mathcal{O} \in S / G} \sum\limits_{s \in \mathcal{O}} \frac{1}{\#(Gs)} = \sum\limits_{\mathcal{O} \in S / G} 1 = \#(S / G)
$$
where $S / G$ is the set of orbits of $S$ under the action of $G$.
\end{proof}

\end{enumerate}

\end{enumerate}
\end{document}







