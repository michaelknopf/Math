\documentclass[10pt]{article}
\usepackage[margin=1in]{geometry}
\addtolength{\oddsidemargin}{-.1in} 
\usepackage{amsmath,amsthm,amssymb}
\usepackage{bm}
\usepackage{enumitem}
\usepackage{array}
\usepackage{lipsum}
\usepackage[]{units}
\usepackage{relsize}
\usepackage{verbatim}
\usepackage{bbm}

\usepackage{tikz}
\usetikzlibrary  {positioning}
\usepackage{graphicx}
\usepackage{xfrac}
%\usetikzlibrary{cd}
\usepackage{tikz-cd}


\setenumerate{listparindent=\parindent}

\newcommand{\Q}{\mathbf{Q}}
\newcommand{\Z}{\mathbf{Z}}
\newcommand{\R}{\mathbf{R}}
\newcommand{\C}{\mathbf{C}}
\newcommand{\F}{\mathbb{F}}
\newcommand{\p}{\mathfrak{p}}
\newcommand{\q}{\mathfrak{q}}
\renewcommand{\a}{\mathfrak{a}}
\renewcommand{\b}{\mathfrak{b}}
\renewcommand{\c}{\mathfrak{c}}
\renewcommand{\o}{\mathfrak{o}}
\newcommand{\m}{\mathfrak{m}}
\newcommand{\gen}[1]{\langle #1 \rangle}
\DeclareMathOperator*{\dom}{dom}
\DeclareMathOperator*{\Aut}{Aut}
\DeclareMathOperator*{\Ann}{Ann}
\DeclareMathOperator*{\Tor}{Tor}
\DeclareMathOperator*{\Gal}{Gal}
\DeclareMathOperator*{\Hom}{Hom}
\DeclareMathOperator*{\End}{End}
\DeclareMathOperator*{\im}{Im}
\let\ker\relax
\DeclareMathOperator*{\ker}{Ker}
\DeclareMathOperator*{\spn}{span}
\DeclareMathOperator*{\Perm}{Perm}
\DeclareMathOperator*{\card}{card}
\DeclareMathOperator*{\Alt}{Alt}
\DeclareMathOperator*{\id}{id}
\DeclareMathOperator*{\Pic}{Pic}
\DeclareMathOperator*{\Maps}{Maps}
\renewcommand{\bar}{\overline}

\newtheorem*{lem}{Lemma}

\usepackage{fancyhdr} % Required for custom headers 
%\usepackage{lastpage} % Required to determine the last page for the footer

\pagestyle{fancy}
\lhead{Math 250A (HW 11)}
\chead{Michael Knopf (24457981)}
\rhead{November $19^\text{th}$, 2015}
\lfoot{}
\cfoot{}
\rfoot{}
%\rfoot{Page\ \thepage\ of\ \pageref{LastPage}}
\renewcommand\headrulewidth{0.4pt}
%\renewcommand\footrulewidth{0.4pt}

\begin{document}
\begin{enumerate}
\item[8.] Let $f(X) \in k[X]$ be a polynomial of degree $n$.  Let $K$ be its splitting field.  Show that $[K:k]$ divides $n!$.

\begin{proof}
Induct on $n$.  This is trivial if $n=0$, since the splitting field of the constant polynomial is $k$, which has degree 1, and 1 divides $0! = 1$.  So suppose $n > 0$ and the proposition holds for all polynomials of degree at most $n$.  Let $f$ have degree $n+1$.

If $f$ is irreducible, then $E = k[X] / (f(X))$ has degree $n+1$ over $k$, and $f$ has at least one linear factor $X - \alpha$ over $E$, where $\alpha = \bar{X}$.  By induction, the splitting field of $\frac{f(X)}{X-\alpha}$ over $E$ (which equals that of $f$ over $k$) is an extension of degree dividing $n!$, since $\frac{f(X)}{X-\alpha}$ has degree at most $n$.  Thus, the degree of $K$ over $k$ divides $(n+1)!$.

Suppose now that $f$ is reducible, meaning $f(X) = h(X)g(X)$ where $\deg(h) = r$ and $\deg(g) = s$.  By induction, the splitting field $K_h$ of $h$ over $k$ has degree dividing $r!$, and the splitting field of $g$ over $K_h$ has degree dividing the degree of $g$ over $K_h$, which is less than $s$, so the degree of $K_g$ over $K_h$ divides $s!$.  This latter extension gives the field $K$, however.  So the degree of $K$ over $k$ divides $s!r!$.  The degree of the original polynomial $f$ was $n=s+r$, and it is always true that $s!r! \mid (s+r)!$, since this quotient counts the number of ways to choose $r$ items from a set of $s+r$.  Thus, $[K:k] \mid s!r! \mid (s+r)! = n$.
\end{proof}

\item[9.] Find the splitting field of $X^{p^8} - 1$ over the field $\Z / p\Z$.

\begin{proof}
This is simply $\mathbb{F}_{p}$, since $X^{p^8} - 1 = (X-1)^{p^8}$ splits completely over this field.  In the case where $p$ is odd, this factorization holds because $(-1)^{p^8} = -1$.  If $p = 2$, then $-1 = 1$ in $\mathbb{F}_p$.  So this factorization holds for all $p$.
\end{proof}

\item[10.]Let $\alpha$ be a real number such that $\alpha^4 = 5$.
\begin{enumerate}
\item Show that $\Q(i\alpha^2)$ is normal over $\Q$.

\begin{proof}
The minimal polynomial of $i\alpha^2$ over $\Q$ is $X^2 + 5$, which splits completely as $(X + i\alpha^2)(X-i\alpha^2)$ over this extension.  Since this polynomial is irreducible, it does not split over any smaller extension (the only other is $\Q$), so this is the splitting field of $X^2 + 5$, hence is normal.
\end{proof}

\item Show that $\Q(\alpha + i\alpha)$ is normal over $\Q(i\alpha^2)$.

\begin{proof}
$\alpha + i\alpha$ satisfies $X^4 + 20$, since $(\alpha+i\alpha)^4 = \alpha^4(1+i)^4 = -20$.  However, over $\Q(i\alpha^2)$ this has a factor of $X^2+ 2i\alpha^2$, which is the minimal polynomial of $\alpha+i\alpha$ over $\Q(i\alpha^2)$.  However, this polynomial is irreducible, so $\Q(\alpha+i\alpha)$ is the splitting field of $X^2 + 2i\alpha^2$ over $\Q(i\alpha^2)$.
\end{proof}

\item Show that $\Q(\alpha + i\alpha)$ is not normal over $\Q$.

\begin{proof}
The minimum polynomial of $\alpha + i\alpha$ over $\Q$ is $X^4 + 20$, whose roots are $\pm \alpha \pm i\alpha$.  However, $\Q(\alpha + i\alpha)$ does not contain $\alpha - i\alpha$.  If it did, then it would also contain $\alpha$, and thus $i$ as well.  Since $\alpha$ has degree 4 over $\Q$, $\alpha \in \Q(\alpha + i\alpha)$ would mean $\Q(\alpha) = \Q(\alpha + i\alpha)$, and so $i \in \Q(\alpha) \subseteq \R$, a contradiction.  So this extension is not normal.
\end{proof}
\end{enumerate}

\item[11.] Describe the splitting fields of the following polynomials over $\Q$, and find the degree of each such splitting field.

\noindent I will give the splitting fields as subfields of $\C$.

\begin{enumerate}
\item $X^2 - 2$

\noindent $\Q(\sqrt{2})$, degree 2.

\item $X^2 - 1$

\noindent $\Q$, degree 1.

\item $X^3 - 2$

\noindent $\Q(\sqrt[3]{2}, \omega)$ where $\omega = \frac{-1 + \sqrt{-3}}{2}$, degree 6.

\begin{proof}
The roots are $\sqrt[3]{2}, \sqrt[3]{2}\omega,$ and $\sqrt[3]{2}\omega^2$.  $X^3 - 2$ is irreducible over $\Q$, so $\Q(\sqrt[3]{2})$ has degree $3$.  The minimum polynomial for $\omega$ is $X^2 + X + 1$, which is also irreducible over $\Q(\sqrt[3]{2})$, and so the total extension has degree $2 \cdot 3 = 6$.
\end{proof}

\item $(X^3 - 2)(X^2 - 2)$

\noindent $\Q(\sqrt{2}, \sqrt[3]{2}, \omega)$, degree 12.

\begin{proof}
This is simply the compositum of the fields from (a) and (c).  Since $\sqrt{2} \not \in \Q(\sqrt[3]{2},\omega)$, the total degree must be $2 \cdot 6 = 12$.
\end{proof}

\item $X^2 + X + 1$

\noindent $\Q(\omega)$, degree 2.

\begin{proof}
The only roots are $\pm \omega$, which are not in $\Q$.
\end{proof}

\item $X^6 + X^3 + 1$

\noindent $\Q(\zeta_9)$ where $\zeta_9 = e^{\frac{2\pi i}{9}}$, degree 6.

\begin{proof}
The roots of this polynomial are the primitive $9$th roots of unity, which are $\zeta_9^k$ for $k$ relatively prime to $9$.  This is because each of these cubes to a primitive cube root of unity, and $X^6 + X^3 + 1 = (X^3)^2 + (X^3) + 1$.
\end{proof}

\item $X^5 - 7$

\noindent $\Q(\sqrt[5]{7}, \zeta_5)$, $\zeta_5 = e^{\frac{2\pi i}{5}}$, degree 20.

\begin{proof}
The roots are $\zeta_5^k \sqrt[5]{7}$ for $0 \leq k \leq 4$.  $\sqrt[5]{7}$ has degree $5$ over $\Q$ and $\zeta_5$ has degree 4.  For $1 \leq k \leq 4$, $\zeta_5^k \not \in \R$, so this element has degree 4 over $\Q(\sqrt[5]{7})$ as well.  Thus, the extension has degree $4 \cdot 5 = 20$.
\end{proof}
\end{enumerate}

\item[12.] Let $K$ be a finite field with $p^n$ elements.  Show that every element of $K$ has a unique $p$-th root in $K$.

\begin{proof}
This is simply a restatement of the fact that the Frobenius endomorphism $\varphi$ is an automorphism.  Recall that $(\alpha + \beta)^p = \alpha^p + \beta^p$ in $K$ (prove using the binomial theorem, $p$ divides $\binom{p}{m}$ if $1 \leq m \leq p-1$).  Obviously $(\alpha\beta)^p = \alpha^p\beta^p$.  Since $1 \mapsto 1$, the kernel is nonzero.  So this map is an embedding.  Since $K$ is finite, it is an isomorphism.
\end{proof}

\item[13.] If the roots of a monic polynomial $f(X) \in k[X]$ in some splitting field are distinct, and form a field, then char$(k) = p$ and $f(X) = X^{p^n} - X$ for some $n \geq 1$.

\begin{proof}
Let $K$ be the field formed by these roots.  $K$ must be finite, since $f$ has finitely many roots.  By the uniqueness of finite fields, $K = \F_{p^n}$ for some $n \geq 1$ and some $p$, which is its characteristic.  We know then that
$$
f(X) = \prod_{\alpha \in K} (X-\alpha) = \prod_{\alpha \in \F_{p^n}} (X-\alpha) = X^{p^n} - X.
$$
\end{proof}

\end{enumerate}
\end{document}


















