\documentclass[10pt]{article}
\usepackage[margin=1in]{geometry}
\addtolength{\oddsidemargin}{-.1in} 
\usepackage{amsmath,amsthm,amssymb}
\usepackage{bm}
\usepackage{enumitem}
\usepackage{array}
\usepackage{lipsum}
\usepackage[]{units}
\usepackage{relsize}
\usepackage{verbatim}
\usepackage{bbm}

\usepackage{tikz}
\usetikzlibrary  {positioning}
\usepackage{graphicx}
\usepackage{xfrac}
\usetikzlibrary{cd}


\setenumerate{listparindent=\parindent}

\newcommand{\Q}{\mathbf{Q}}
\newcommand{\Z}{\mathbf{Z}}
\newcommand{\R}{\mathbf{R}}
\newcommand{\C}{\mathbf{C}}
\newcommand{\p}{\mathfrak{p}}
\newcommand{\q}{\mathfrak{q}}
\renewcommand{\a}{\mathfrak{a}}
\renewcommand{\b}{\mathfrak{b}}
\renewcommand{\c}{\mathfrak{c}}
\renewcommand{\o}{\mathfrak{o}}
\newcommand{\m}{\mathfrak{m}}
\newcommand{\gen}[1]{\langle #1 \rangle}
\DeclareMathOperator*{\dom}{dom}
\DeclareMathOperator*{\Aut}{Aut}
\DeclareMathOperator*{\Ann}{Ann}
\DeclareMathOperator*{\Tor}{Tor}
\DeclareMathOperator*{\Gal}{Gal}
\DeclareMathOperator*{\Hom}{Hom}
\DeclareMathOperator*{\End}{End}
\DeclareMathOperator*{\im}{Im}
\let\ker\relax
\DeclareMathOperator*{\ker}{Ker}
\DeclareMathOperator*{\spn}{span}
\DeclareMathOperator*{\Perm}{Perm}
\DeclareMathOperator*{\card}{card}
\DeclareMathOperator*{\Alt}{Alt}
\DeclareMathOperator*{\id}{id}
\DeclareMathOperator*{\Pic}{Pic}
\DeclareMathOperator*{\Maps}{Maps}
\renewcommand{\bar}{\overline}

\newtheorem*{lem}{Lemma}

\usepackage{fancyhdr} % Required for custom headers 
%\usepackage{lastpage} % Required to determine the last page for the footer

\pagestyle{fancy}
\lhead{Math 250A (HW 9)}
\chead{Michael Knopf (24457981)}
\rhead{October $22^\text{nd}$, 2015}
\lfoot{}
\cfoot{}
\rfoot{}
%\rfoot{Page\ \thepage\ of\ \pageref{LastPage}}
\renewcommand\headrulewidth{0.4pt}
%\renewcommand\footrulewidth{0.4pt}

\begin{document}
\begin{enumerate}
\item[III.11.] Let $M$ be a finitely generated torsion-free module over $\o$.  Prove that $M$ is projective.

\begin{proof}

Given a prime ideal $\p$, the localized module $M_\p$ is generated over $\o_\p$ by any generating set of $M$ over $\o$, hence is finitely generated.  If $\frac{a}{s}m = 0$ for some $m \in M_\p$, then multiplication by $s$ gives that $am = 0$.  Thus, since $M$ is torsion free, $a=0$ or $m=0$, so $M_\p$ is also torsion free.  $\o_\p$ is a principal ideal domain by the combination of exercises 15 and 16: there is a unique prime ideal $\q$ of $\o_\p$, so every ideal factors as $\q^k$ for some $k$; but there is some $t \in \q \setminus \q^2$, hence $\q = (t)$ is principal, and so every ideal is principal of the form $(t^k)$.  By Theorem 7.3, $M_\p$ is free and therefore projective.

Let $F$ be finite free over $\o$ and $f:F \rightarrow M$ surjective.  $f$ extends naturally to a homomorphism $f_\p:F_\p \rightarrow M_\p$ by $f_\p(\frac{a}{s}x) = \frac{a}{s}f(x)$ for $s \not \in \p$, which is surjective because every element of $M_\p$ is of the form $\frac{a}{s}m$ for some $m \in M$.  So the sequence
$$
\begin{tikzcd}
0 \arrow[r]& \ker f_\p \arrow[r] & F_\p \arrow[r, "f_\p"'] & M_\p \arrow[r] \arrow[l, bend right, "g_\p"'] & 0
\end{tikzcd}
$$
is exact, hence it yields a splitting homomorphism $g_\p$.

As a homomorphism of $\o_\p$-modules, $g_\p$ is naturally a homomorphism of $\o$-modules as well.  Since it is a right inverse for $f_\p$, it is injective, hence gives an embedding $g_\p(M)$ of $M$ into $F_\p$ (as an $\o$-module).  This embedding has a finite generating set $\{m_1, \dots , m_n\}$, and for each $i$ there is some $c_i \not \in \p$ for which $c_i m_i \in F$.  Let $c_\p$ be the product of all the $c_i$.  We know $c_\p \not \in \p$ since $\p^c$ is multiplicative, and that $c_\p m_i \in F$ for all $i$.  Thus $c_\p g_p(M) \subseteq F$.

Consider the ideal $\a$ generated by $\{c_\p : \p \subseteq \o \text{ is prime}\}$.  If $\a \neq \o$, then $\a$ is contained in some maximal (hence prime) ideal $\q$.  But $c_\q \not \in \a$ contradicts that $c_\q \in \a \subseteq \q$.  So we must have $\a = \o$.  So there are some finite collections $\{c_{\p_i}\}$ and $\{x_i\} \subseteq \o$ for which $\sum x_i c_{\p_i} = 1$.  Letting $g = \sum x_i c_{\p_i} g_{\p_i}$, we have $g: M \rightarrow F$ and
$$
f \circ g(m) = \sum x_i f (c_{\p_i} g_{\p_i}(m)) = \sum x_i f_{\p_i} (c_{\p_i} g_{\p_i}(m)) = \sum x_i c_{\p_i} f_{\p_i} \circ g_{\p_i}(m) = m\sum x_i c_{\p_i} = m
$$
for any $m \in M$.  Thus $f \circ g = \id_M$.  This means that the sequence
$$
\begin{tikzcd}
0 \arrow[r] & \ker f \arrow[r] & F \arrow[r] & M \arrow[r] & 0
\end{tikzcd}
$$
splits, thus $F = \ker f \oplus M$.  So $M$ is a direct summand of a free module, hence projective.
\end{proof}

\item[III.12.]
\begin{enumerate}
\item Let $\a,\b$ be ideals.  Show that there is an isomorphism of $\o$-modules
$$
\a \oplus \b \cong \o \oplus \a\b.
$$

\begin{proof}
We may choose a $c \in K$ such that $\a + c\b = \o$.  Consider the surjective $\o$-linear map $\a \oplus \b \rightarrow \a + c\b$ given by $a+b \mapsto a + cb$.  Its kernel is the set of pairs $(-cb,b)$ for which $cb \in \a$, which is isomorphic simply to $c\b \cap \a$.  Since these are relatively prime, we know this equals $c \a\b$, which as an $\o$-module is isomorphic to $\a\b$.  Thus we have an exact sequence
$$
\begin{tikzcd}
0 \arrow[r] & \a\b \arrow[r] & \a \oplus \b \arrow[r] & \o \arrow[r] & 0.
\end{tikzcd}
$$
Clearly, $\o$ is finitely generated and torsion-free over itself (it is an integral domain, generated over itself by $1$).  By the previous exercise, $\o$ is projective.  Hence, the above sequence splits, giving us $\a \oplus \b \cong \o \oplus \a\b$.
\end{proof}

\item Let $\a,\b$ be fractional ideals, and let $f:\a \rightarrow \b$ be an isomorphism (of $\o$-modules).  Then $f$ has an extension to a $K$-linear map $f_K:K \rightarrow K$.  Let $c = f_K(1)$.  Show that $\b = c\a$ and that $f$ is given by the mapping $m_c: x \rightarrow cx$.

\begin{proof}

If $f = 0$, we must have $\a=\b=0$.  In this case, $f_K(1)$ is not defined, but we can take $c=0$ and still have $\b = c\a$ with $f = m_c$.  So assume $f\neq 0$, and let $a \in \a$ such that $f(a) \neq 0$.  Then $c=f_K(1) = a^{-1}f(a)$, and for any $x \in \a$ we have $f(x) = xf_K(1) = cx$, thus $f = m_c$.  Since $\b$ is the image of $f$, $\b = c\a$.

\end{proof}

\item Let $\a$ be a fractional ideal.  For each $b \in \a^{-1}$ the map $m_b:\a \rightarrow \o$ is an element of the dual $\a^{\vee}$.  Show that $\a^{-1} = \a^\vee = \Hom_\o (\a,\o)$ under this map, and so $\a^{\vee \vee} = \a$.

\begin{proof}

Consider $\a^{-1} \rightarrow a^{\vee}$ given by $b \mapsto m_b$, which is clearly injective.  If $f \in \a^{\vee}$, then by the argument in (b) we know $f = m_c$ where $c = a^{-1}f(a)$.  Since $\a^{-1}$ is an $\o$-module, and $f(a) \in \o$, we have $c \in \a^{-1}$.  So we have an isomorphism $\a^{-1} \cong a^{\vee}$.  Therefore, $\a = (\a^{-1})^{-1} \cong \a^{\vee \vee}$.

\end{proof}

\end{enumerate}
\item[III.13.]
\begin{enumerate}
\item Let $M$ be a projective finite module over the Dedekind ring $\o$.  Show that there exist free modules $F$ and $F'$ such that $F \supseteq M \supseteq F'$, and $F,F'$ have the same rank, which is called the rank of $M$.

\begin{proof}

Let $S$ be a generating set for $M$ that is as small as possible, and let $|S| = n$.  The free $\o$-module $F$ on $S$ surjects onto $M$; since $M$ is projective, then, $M$ is a direct summand of $F$.  Thus $M \subseteq F$.

Let $\a$ be an ideal of $\o$.  We showed in the previous homework that $\a$ is finitely generated.  Since $\o$ is an integral domain, $\a$ is torsion free.  Thus, by exercise 11, $\a$ is projective.  Also, if $\a \neq 0$ and $a$ is one of the generators of $\a$, then $\a$ contains the free $\o$-module on $a$.  Thus, $\a$ contains a free module of rank 1.  Clearly, if $\a = 0$ then it contains the free module on 0 generators.

We will now induct on $n$ (the least possible size of a generating set for $M$) to show that if $M \subseteq \o^{n}$ then $M = \a_1 \oplus \cdots \oplus \a_n$ for some nonzero ideals $\a_i$ of $\o$.  This is clear if $n=0$, since then $M \cong (0)$ is the empty sum.  If $n=1$, then $M$ is, by definition, a nonzero ideal of $\o$.  Supposing this holds for $n-1$, if $M \subseteq \o^n$ we can take the projection map $\pi$ of $\o^n$ onto its first coordinate, and consider its restriction to $M$.  The image of $M$ is a submodule of the image of $\o^n$, which is $\o$.  Therefore, the image of $M$ is an ideal $\a_n$, which is also projective.  The kernel lies within the last $n-1$ summands of $\o^n$, hence by induction equals $\a_1 \oplus \cdots \oplus \a_{n-1}$.  Since $\a_n$ is projective, the exact sequence
$$
\begin{tikzcd}
0 \arrow[r] & \ker \pi \mid_M \arrow[r] & M \arrow[r] & \im M \arrow[r] & 0
\end{tikzcd}
$$
which is the same as
$$
\begin{tikzcd}
0 \arrow[r] & \a_1 \oplus \cdots \oplus \a_{n-1}  \arrow[r] & M \arrow[r] & \a_n \arrow[r] & 0
\end{tikzcd}
$$
splits, giving us $M = \a_1 \oplus \cdots \oplus\a_n$.  We have already argued that each $\a_i$ contains a rank $1$ submodule $M_i$.  Therefore, $F' = M_1 \oplus \cdots \oplus M_n$ is a free module of rank $n$ contained in $M$.

\end{proof}

\item Prove that there exists a basis $\{e_1, \dots , e_n\}$ of $F$ and ideals $\a_1, \dots , \a_n$ such that $M = \a_1 e_1 + \cdots + \a_ne_n$, or in other words, $M \cong \oplus \a_i$.

\begin{proof}
We have just shown this in part (a).  We know $M \cong \oplus \a_i$, so $M$ has a generating set $e_1, \dots , e_n$ such that $M = \oplus \a_i e_i$.  Thus, $\o\gen{\{e_1, \dots , e_n\}}$ is a free module $F$ of rank $n$ that contains $M$ (since $M$ is projective).
\end{proof}

\item Prove that $M \cong  \o^{n-1} \oplus \a$ for some ideal $\a$, and that the association $M \mapsto \a$ induces an isomorphism of $K_0(\o)$ with the group of ideal classes $\Pic(\o)$.

\begin{proof}
Using associativity of the direct sum, part (a) of exercise 12 extends to $\a_1 \oplus \cdots \oplus \a_n \cong \o^{n-1} \oplus (\a_1 \cdots \a_n)$ for any $n$.  By the previous result, we have $M \cong \a_1 \oplus \cdots \oplus \a_n \cong \o^{n-1} \oplus (\a_1 \cdots \a_n)$, hence $\a = \a_1 \cdots \a_n$.  Note that if $M \neq 0$ can be written as $\o^{n-1} \oplus 0$, then $n \geq 2$ and so it can also be written as $\o^{n-2} \oplus \o$.  For the remainder of this proof, we will always take the latter form if this ambiguous case arises, so that the righthand summand is never $0$ unless $M = 0$.

Since any finite projective module $M$ over $\o$ can be written as $M \cong \o^{n-1} \oplus \a$ for some nonzero ideal $\a$, we would like to define a map $K_0(\o) \rightarrow \Pic(\o)$ by $[\o^{n-1} \oplus \a] \mapsto [\a]$; however, it is not immediately evident that this is well-defined.  Specifically, we need to verify that if $[\o^{n-1} \oplus \a] = [\o^{m-1} \oplus \b]$ in $K_0(\o)$ (for some $\a$ and $\b$ nonzero), then $[\a] = [\b]$ in $\Pic(\o)$.

To verify this, suppose that $[\o^{n-1} \oplus \a] = \o^{m-1} \oplus \b]$.  Because the equivalence relation defining the classes of $K_0(\o)$ equates two ideals that are isomorphic up to adding a free module, we can simply assume that $\o^{n-1} \oplus \a \cong \o^{m-1} \oplus \b$ (since adding free modules does nothing but increase the values of $n$ and $m$, which are already arbitrary).  Tensoring over $\o$ with its field of fractions $K$ gives us
$$
(\o^{m-1} \oplus \b) \otimes K = (\o \otimes K)^{m-1} \oplus (\b \otimes K)
$$
We have ``extended the base" of $\o$ and of $\b$ to produce vector spaces over $K$.  Lang's Proposition 4.1 in the chapter on the tensor product states that $\o \otimes K$ is a 1-dimensional vector space over $K$.

For the righthand term, we know $\b \otimes K \subseteq \o \otimes K$, thus $\b \otimes K$ has dimension at most 1.  If it had dimension $0$, this would mean that the only $\o$-bilinear map from $\b \times K$ to a given $\o$-module is $0$.  However, $(b,k) \mapsto bk$ is bilinear, and is nonzero unless $\b = 0$, which we know is not the case.  Therefore, $\b \otimes K$ has dimension 1 as well, proving that $m = n$.

We will now show by induction on $n$ that if $\o^{n-1} \oplus \a \cong \o^{n-1} \oplus \b$, then $\a \cong \b$.  This is clear for $n = 1$, but consider $n=2$, i.e. $f: \o \oplus \a \rightarrow \o \oplus \b$ is an isomorphism.  Suppose first that $f^{-1}((1,0)) = (0,\alpha) \in 0 \oplus \a$.  Then $f^{-1}(\o \oplus 0) \subseteq 0 \oplus \gen{\alpha}$.  If $\a \neq \gen{\alpha}$, then there is some $\beta \in \a \setminus \gen{\alpha}$.  Since $f$ is injective, we must have $f((0,\beta)) \in 0 \oplus \b$, hence $f(0 \oplus \gen{\beta}) \subseteq 0 \oplus \b$.  However, $\alpha\beta \in \gen{\alpha} \cap \gen{\beta}$, thus $f((0,\alpha\beta)) \in (\o \oplus 0) \cap (0 \oplus \a) = 0$.  This means $\alpha\beta = 0$, a contradiction ($\alpha \neq 0$ because $f((0,\alpha)) \neq 0$, $\beta \neq 0$ because $\beta \not \in \gen{\alpha}$, and $\o$ is an integral domain so $\alpha \beta \neq 0$).

We now know that either $\a$ is principal and $f(0 \oplus \a) = \o \oplus 0$, or $f^{-1}((1,0)) \in \o \oplus 0$.  The first case implies that $f(\o \oplus 0) = 0 \oplus \b$, meaning that $\b \cong \o \cong \a$, as desired.  So assume the second case, meaning $f^{-1}((1,0)) = (x,0)$ for some nonzero $x$.  If $f((1,0)) = (a,b)$, then $(1,0) = f((x,0)) = xf((1,0)) = (xa,xb)$, so $a$ is a unit and $b = 0$.  Thus $f((1,0))$ generates $\o \oplus 0$, so $f(\o \oplus 0) = \o \oplus 0$.  Thus the restriction of $f$ to $0 \oplus \a$ gives an isomorphism $\a \cong \b$, as desired.

Finally, consider the general case $\o^{n-1} \oplus \a \cong \o^{n-1} \oplus \b$.  We have $\o^{n-2} \oplus (\o \oplus \a) \cong \o^{n-2} \oplus (\o \oplus \b)$, thus $\o \oplus \a \cong \o \oplus \b$ by induction.  Having reduced the problem to the $n=2$ case, we know $\a \cong \b$.  By part (b) of the previous exercise, $\a = c\b$ for some $c \in K$, therefore $\a\b^{-1} = (c)$ is principal; thus $[\a] = [\b]$ in $\Pic(\o)$.  This proves that the map $K_0(\o) \rightarrow \Pic(\o)$ given by $[\o^{n-1} \oplus \a] \mapsto [\a]$ is well-defined.

Now, suppose $[\o^{n-1} \oplus \a]$ is in the kernel.  Then $\a$ is principal, meaning that $\a \cong \o$.  But then $\o^{n-1} \oplus \a \cong \o^n \sim 0$, so $[\o^{n-1} \oplus \a] = [0]$.  So the map is injective.  Next, consider any ideal $\a$ of $\o$.  Since $\a$ is projective, so is $\o^{n-1} \oplus \a$.  Thus $[\o^{n-1} \oplus \a] \mapsto [\a]$, so this map is surjective as well.  Every class contains some ideal as a representative, so to see that the map is a homomorphism we can just check the property on ideals:
$$
[\a][\b] = [\a \oplus \b] = [\o \oplus \a\b] \mapsto [\a\b] = [\a][\b].
$$
The explanation for this is that the operation in $K_0(\o)$ is the direct sum, but we can rewrite $\a \oplus \b$ as $\o \oplus \a\b$ by the previous exercise.  On the right side of the arrow, the classes are elements of $\Pic(\o)$.  But $\Pic(\o)$ is a quotient of the group of nonzero fractional ideals of $\o$ under ideal multiplication, therefore the $[\a\b] = [\a][\b]$.  So this map is an isomorphism $K_0(\o) \cong \Pic(\o)$.
\end{proof}
\end{enumerate}

\item[Exercise.] Show that the functor that takes each set to its power set is not representable.

\begin{proof}
Suppose the functor is representable by some set $S$.  Then $\Maps(X,\emptyset) \cong \mathcal{P}(\emptyset) = \{\emptyset\}$.  If $X$ is nonempty, then $\Maps(X,\emptyset) = \emptyset$, a contradiction because this has cardinality less than $\{\emptyset\}$.  Thus $X$ is the empty set (so $\Maps(X,\emptyset) = \{\emptyset\}$ is satisfied).  But then $\Maps(X,\{\emptyset\}) = \emptyset \not \cong \{\emptyset, \{\emptyset\} \} = \mathcal{P}(\{\emptyset\})$.
\end{proof}

\item[III.15.]\textbf{The five lemma.} Consider a commutative diagram of $R$-modules and homomorphisms such that each row is exact:
$$
\begin{tikzcd}
M_1 \arrow[r] \arrow[dd,"f_1"'] & M_2 \arrow[r] \arrow[dd,"f_2"'] & M_3 \arrow[r] \arrow[dd,"f_3"'] & M_4 \arrow[r] \arrow[dd,"f_4"'] & M_5 \arrow[dd,"f_5"'] \\ \\
N_1 \arrow[r] & N_2 \arrow[r] & N_3 \arrow[r] & N_4 \arrow[r] & N_5
\end{tikzcd}
$$
Prove
\begin{enumerate}
\item If $f_1$ is surjective and $f_2,f_4$ are monomorphisms, then $f_3$ is a monomorphism.

\begin{proof}

Let $m_3 \in \ker f_3$.  The image of $m_3$ in $N_3$, and hence also in $N_4$, is 0.  Since $f_4$ is injective, this means the image of $m_3$ in $M_4$ is $0$.  Since the top row is exact, $m_3$ has a preimage $m_2$ in $M_2$.  $m_2$ maps to $0$ in $N_3$ (since it shares the image of $m_3$), hence $f(m_2)$ has a preimage $n_1$ in $N_1$ by the exactness of the bottom row.  Since $f_1$ is surjective, $n_1$ has a preimage $m_1$ in $M_1$.  The image of $m_1$ in $N_2$ is the same as that of $m_2$, hence $m_1$ maps to $m_2$ by the injectivity of $f_2$.  Thus, by the exactness of the top row, the image of $m_2$ in $M_3$ is $0$.  But this image is $m_3$ by assumption, thus $f_3$ is injective.

\end{proof}

\item If $f_5$ is a monomorphism and $f_2,f_4$ are surjective, then $f_3$ is surjective.

\begin{proof}

Let $n_3 \in N_3$.  Some $m_4$ in $M_4$ shares an image in $N_4$ with $n_3$.  By the exactness of the bottom row, $n_3$ maps to $0$ in $N_5$, hence so does $m_4$.  By the injectivity of $f_5$, $m_4$ goes to $0$ in $M_5$.  So, by the exactness of the top row, $m_4$ has a preimage $m_3$ in $M_3$.  Now, $f_3(m_3)$ and $n_3$ share an image in $N_4$, so $n_3 - f(m_3)$ has a preimage $n_2$ in $N_2$ (by exactness of the bottom row).  Let $x$ be the image of $m_2$ in $M_3$.  Since $m_2$ goes to $n_3 - f_3(m_3)$, we know $f_3(x) = n_3 - f_3(m_3)$.  Thus, $f_3(x + m_3) = n_3$, so $f_3$ is surjective.

\end{proof}

\end{enumerate}

\item[XVI.6.] Let $M,N$ be flat.  Show that $M \otimes N$ is flat.

\begin{proof}
Suppose
$
\begin{tikzcd}
0 \arrow[r] & X \arrow[r] & Y
\end{tikzcd}
$
is exact.  Since $N$ is flat, 
$
\begin{tikzcd}
0 \arrow[r] & N \otimes X \arrow[r] & N \otimes Y
\end{tikzcd}
$
is exact.  Since $M$, is exact,
$
\begin{tikzcd}
0 \arrow[r] & (M \otimes N) \otimes X \arrow[r] & (M \otimes N) \otimes Y
\end{tikzcd}
$
where we have also applied the associativity of the tensor product.  Therefore, $M \otimes N$ is flat.
\end{proof}

\item[XVI.7.] Let $F$ be a flat $R$-module, and let $a \in R$ be an element which is not a zero-divisor.  Show that if $ax = 0$ for some $x \in F$ then $x = 0$.

\begin{proof}
Since $a$ is not a zero divisor, we have an exact sequence $\begin{tikzcd} 0 \arrow[r] & R \arrow[r, "\varphi_a"] & (a) \arrow[r] & 0 \end{tikzcd}$ where $\varphi_a$ is multiplication by $a$.  By Proposition 3.7, $(a) \otimes F \cong (a)F$ by the natural map, so tensoring with $F$ yields the exact sequence $\begin{tikzcd} 0 \arrow[r] & F \arrow[r, "\bar{\varphi}_a"] & (a)F \arrow[r] & 0 \end{tikzcd}$, where the induced map $\bar{\varphi}_a$ is scaling by $a$.  Therefore, the kernel of this homomorphism is $0$, which is the desired result.
\end{proof}

\item[XVI.9.] Prove Proposition 3.2:
\begin{enumerate}
\item[(i)] Let $S$ be a multiplicative subset of $R$.  Then $S^{-1}R$ is flat over $R$.

\begin{proof}
Suppose $f: M \rightarrow N$ is an injection, and that the induced map $\bar{f}:S^{-1}R \otimes M \rightarrow S^{-1}R \otimes N$ takes $\frac{r}{s} \otimes m$ to $0$, meaning $\frac{r}{s} \otimes f(m) = 0$.  Multiplying by $s$ gives $r \otimes f(m) = 0$.  We know that the base ring $R$ is flat, however, so the restriction of $\bar{f}$ to $R \otimes M$ must be injective.  Thus, $r \otimes m = 0$.  Multiplying by $\frac{1}{s}$ yields $\frac{r}{s} \otimes m = 0$, hence $\bar{f}$ is injective.
\end{proof}

\item[(ii)] A module $M$ is flat over $R$ if and only if the localization $M_\p$ is flat over $R_\p$ for each prime ideal $\p$ of $R$.

\begin{proof}
\begin{comment}
If $M$ and $L$ are a $R$-modules and $N$ is an $R_\p$-module, then an $R$-bilinear map $g : M \times N \rightarrow L$ has a natural extension to an $R_\p$-bilinear map $g_\p : M_\p \times N \rightarrow L$ given by $g_\p(\frac{r}{s}m,n) = \frac{r}{s}g(m,n)$ (linearity in the second component comes from the fact that $sg(m,\frac{r}{s}n) = rg(m,n)$ after dividing through by $s$).  Thus, if $m \otimes n = 0$ is nonzero in $M \otimes_R N$, then it must also be nonzero in $M_\p \otimes_{R_\p} N$.  This is because there is a bilinear map $M \times N \rightarrow L$ which does not take $(m,n)$ to 0, hence neither does the extension $g_\p$.

Suppose that $M$ is flat over $R$, and let $f:X \rightarrow Y$ be injective for some $R_\p$ modules $X,Y$.  Suppose that the induced map $\bar{f} : M_\p \otimes_{R_\p} X \rightarrow M_\p \otimes_{R_\p} Y$ takes $\frac{r}{s}m \otimes x$ to $0$ for some $m \in M$.  Then $\frac{r}{s}m \otimes f(x) = 0$, so $rm \otimes f(x) = 0$.  By the previous paragraph, $rm \otimes f(x)$ must also be zero in $M \otimes_R Y$.  Since $M$ is flat, we know $rm \otimes x = 0$, hence so is $\frac{r}{s}m \otimes x$.  So $\bar{f}
$ is injective, thus $M_\p$ is flat over $R_\p$.
\end{comment}

Localization distributes over exact sequences: Let $S$ be a multiplicative subset of $R$, and denote $S^{-1}M$ by $M_S$.  Given a sequence
$
\begin{tikzcd}
Y \arrow[r,"f"] & X \arrow[r,"g"] & Z
\end{tikzcd}
$
there is a unique induced sequence
$
\begin{tikzcd}
Y_S \arrow[r,"f_S"] & X_S \arrow[r,"g_S"] & Z_S
\end{tikzcd}.
$
This is because an $R$-linear map on $M$ has a unique extension to an $R_S$-linear map on $M_S$, since $sf(\frac{r}{s}m) = f(rm)$, so $f(\frac{r}{s}m) = \frac{r}{s}f(m)$.  We will show the induced sequence is exact.  Let $x \in \ker g_S$.  Then $\frac{1}{s}g_S(rx) = 0$ so $rx \in \ker g = \im f \subseteq \im f_S$.  This is an ideal, thus scaling gives $\frac{r}{s}x \in \im f_S$.  Now if $\frac{r}{s}x \in \im f_S$ then it is the image of some $\frac{u}{v}y \in Y$.  Thus $f(suy) = vrx$, so $vrx \in \im f = \ker g \subseteq \ker g_S$.  Scaling by $\frac{v}{s}$ shows that $\frac{r}{s}x \in \ker g_S$.  Therefore, if $M \rightarrow N$ is injective, then
$
\begin{tikzcd}
0 \arrow[r] & M_\p \arrow[r,"f"] & N_\p
\end{tikzcd}
$
is exact, hence $f_\p$ is injective.

Localization also distributes over the tensor product.  $(M \otimes_R N)_S = M_S \otimes_{R_S} N_S$ since the map $(\frac{r}{s}m, \frac{u}{v}n) \mapsto \frac{ru}{sv} m \otimes n$ is obviously bilinear and induces a bijection.  Also, if $M$ is already an $R_S$-module, then $M_S = M$ due to the fact of $S$ being multiplicative.

Finally, $\begin{tikzcd} 0 \arrow[r] & M \arrow[r] & N \end{tikzcd}$ is exact if and only if $\begin{tikzcd} 0 \arrow[r] & M_\p \arrow[r] & N_\p \end{tikzcd}$ is exact for all prime ideals $\p \subseteq R$.  The forward direction is trivial, since the kernel of $M \rightarrow N$ is a subset of the kernel of $M_\p \rightarrow N_\p$.  For the converse, suppose the kernel $K$ contains some nonzero $x$.  Then $1$ is not in the annihilator of $x$, hence this ideal is proper and can be embedded in some maximal (hence prime) ideal $\p$.  But then we cannot have $x = 0$ in $K_\p$, since it would mean $sx = 0$ for some $s \not \in \p$, contradicting that the annihilator is contained in $\p$.  Thus $K_\p \neq 0$.

For the main proof, suppose $M$ is flat over $R$.  If $\begin{tikzcd} 0 \arrow[r] & X \arrow[r] & Y \end{tikzcd}$ is exact (over $R_\p$), then \break $\begin{tikzcd} 0 \arrow[r] & M \otimes_R X \arrow[r] & M \otimes_R Y \end{tikzcd}$ is exact, hence $\begin{tikzcd} 0 \arrow[r] & (M \otimes_R X)_\p \arrow[r] & (M \otimes_R Y)_\p \end{tikzcd}$ is exact for all $\p$.  But this sequence equals $\begin{tikzcd} 0 \arrow[r] & M_\p \otimes_{R_\p} X_\p \arrow[r] & M_\p \otimes_{R_\p} Y_\p \end{tikzcd}$, which equals $\begin{tikzcd} 0 \arrow[r] & M_\p \otimes_{R_\p} X \arrow[r] & M_\p \otimes_{R_\p} Y \end{tikzcd}$ because $X_\p = X$.  Therefore, $M_\p$ is flat over $R_\p$ for all $\p$.

Next, suppose $M_\p$ is flat over $R_\p$ for all $\p$, and that $\begin{tikzcd} 0 \arrow[r] & X \arrow[r] & Y \end{tikzcd}$ is exact (over $R$).  Then $\begin{tikzcd} 0 \arrow[r] & M_\p \otimes_{R_\p} X_\p \arrow[r] & M_\p \otimes_{R_\p} Y_\p \end{tikzcd}$ and hence $\begin{tikzcd} 0 \arrow[r] & (M \otimes_R X)_\p \arrow[r] & (M \otimes_R Y)_\p \end{tikzcd}$ are exact for all $\p$.  By the previous paragraph, $\begin{tikzcd} 0 \arrow[r] & M \otimes_R X \arrow[r] & M \otimes_R Y \end{tikzcd}$ must be exact, so $M$ is flat.
\end{proof}

\item[(iii)] Let $R$ be a principal ring.  A module $F$ is flat if and only if $F$ is torsion free.
\begin{proof}
The forward direction is the result of the previous exercise.  By Proposition 3.7, $F$ is flat if the natural map $(a) \otimes F \rightarrow (a)F$ is an isomorphism.  Clearly, it is surjective.  If $a \otimes x \mapsto ax = 0$, then we must have $a = 0$ or $x = 0$ because $F$ is torsion free.  Thus the map is injective, so $F$ is flat.
\end{proof}
\end{enumerate}

\end{enumerate}
\end{document}


















