\documentclass[10pt]{article}
\usepackage[margin=1in]{geometry}
\addtolength{\oddsidemargin}{-.1in} 
\usepackage{amsmath,amsthm,amssymb}
\usepackage{bm}
\usepackage{enumitem}
\usepackage{array}
\usepackage{lipsum}
\usepackage[]{units}
\usepackage{relsize}
\usepackage{verbatim}
\usepackage{bbm}

\usepackage{tikz}
\usetikzlibrary  {positioning}
\usepackage{graphicx}
\usepackage{xfrac}
%\usetikzlibrary{cd}
\usepackage{tikz-cd}


\setenumerate{listparindent=\parindent}

\newcommand{\Q}{\mathbf{Q}}
\newcommand{\Z}{\mathbf{Z}}
\newcommand{\R}{\mathbf{R}}
\newcommand{\C}{\mathbf{C}}
\newcommand{\F}{\mathbb{F}}
\newcommand{\p}{\mathfrak{p}}
\newcommand{\q}{\mathfrak{q}}
\renewcommand{\a}{\mathfrak{a}}
\renewcommand{\b}{\mathfrak{b}}
\renewcommand{\c}{\mathfrak{c}}
\renewcommand{\o}{\mathfrak{o}}
\newcommand{\m}{\mathfrak{m}}
\newcommand{\gen}[1]{\langle #1 \rangle}
\DeclareMathOperator*{\dom}{dom}
\DeclareMathOperator*{\Aut}{Aut}
\DeclareMathOperator*{\Ann}{Ann}
\DeclareMathOperator*{\Tor}{Tor}
\DeclareMathOperator*{\Gal}{Gal}
\DeclareMathOperator*{\Hom}{Hom}
\DeclareMathOperator*{\End}{End}
\DeclareMathOperator*{\im}{Im}
\let\ker\relax
\DeclareMathOperator*{\ker}{Ker}
\DeclareMathOperator*{\spn}{span}
\DeclareMathOperator*{\Perm}{Perm}
\DeclareMathOperator*{\card}{card}
\DeclareMathOperator*{\Alt}{Alt}
\DeclareMathOperator*{\id}{id}
\DeclareMathOperator*{\Pic}{Pic}
\DeclareMathOperator*{\Maps}{Maps}
\DeclareMathOperator*{\Char}{char}
\renewcommand{\bar}{\overline}

\newtheorem*{lem}{Lemma}

\usepackage{fancyhdr} % Required for custom headers 
%\usepackage{lastpage} % Required to determine the last page for the footer

\pagestyle{fancy}
\lhead{Math 250A (HW 12)}
\chead{Michael Knopf (24457981)}
\rhead{December $1^\text{st}$, 2015}
\lfoot{}
\cfoot{}
\rfoot{}
%\rfoot{Page\ \thepage\ of\ \pageref{LastPage}}
\renewcommand\headrulewidth{0.4pt}
%\renewcommand\footrulewidth{0.4pt}

\begin{document}
\begin{enumerate}
\item[14.] Let $\Char(K) = p$.  Let $L$ be a finite extension of $K$, and suppose $[L:K]$ is prime to $p$.  Show that $L$ is separable over $K$.

\begin{proof}
Let $E$ be an algebraic closure of $K$.  Since $L$ is finite, it is algebraic, and so $L = K[\alpha_1, \dots , \alpha_n]$ for some $\alpha_1, \dots , \alpha_n \in E$.  We will know that $L$ is separable if $K(\alpha_i)$ is separable for each $i$.  Also, $[K(\alpha_i):K]$ divides $[L:K]$ for each $i$, thus the degree of each $\alpha_i$ is also prime to $p$.  So, it suffices to show that $K(\alpha)$ is separable over $K$ for any algebraic $\alpha \in E$ of degree prime to $p$.

Suppose $\alpha \in E$ satisfies this, and let $f(X)$ be the minimal polynomial of $\alpha$ over $K$.  Assume for a contradiction that $f(X)$ is inseparable.  Then $f(X)$ and its derivative $f'(X)$ share a root.  But $f(X)$ is irreducible, and so it must divide $f'(X)$ over $K$.  However, $f'(X)$ has degree strictly less than that of $f(X)$, and so we must have $f'(X) = 0$.

Now, say $f(X) = X^m + a_{m-1}X^{m-1} + \cdots + a_1 X + a_0$.  We know $m > 0$ because $f$ is irreducible and $p \nmid m$ because the degree of $f$ is prime to $p$.  Then $f'(X) = mX^{m-1} + (m-1)a_{m-1}X^{m-2} + \cdots + a_1 = 0$, and so $p$ divides $m$, a contradiction.
\end{proof}

\item[15.] Suppose $\Char(K)  = p$.  Let $a \in K$.  If $a$ has no $p$-th root in $K$, show that $X^{p^n} - a$ is irreducible in $K[X]$ for all positive integers $n$.

\begin{proof}
Suppose $a$ has no $p$-th root in $K$.  Let $E$ be an algebraic closure of $K$, and let $\alpha$ be a root of $f(X) = X^{p^n} - a = (X - \alpha)^{p^n}$ in $E$.  Suppose $f(X) = g(X)h(X)$ with $g(X),h(X) \in K[X]$.  We may assume $g(X)$ is monic, since otherwise we could multiply both factors by units to make it so.  So $g(X) = (X-\alpha)^s$ for some $s \leq p^n$, and $h(X) = (X-\alpha)^{p^n - s}$.

If $k$ is the highest power of $p$ dividing $s$, then we may write $s = p^k t$ where $p \nmid t$ and $k \leq n$.  Therefore,
$$g(X) = (X-\alpha)^{p^k t} = (X^{p^k} - \alpha^{p^k})^t = \sum_{m=0}^t \binom{t}{m} (\alpha^{p^k})^m X^{p^k m}$$
has coefficients in $K$.  In particular, the coefficient of the term where $m = 1$ is in $K$.  This coefficient is $t \alpha^{p^k}$.  Dividing by $t$ gives us that $\alpha^{p^k} \in K$.  If $k < n$, then $(\alpha^{p^k})^{p^{n-k - 1}} = \alpha^{p^{n-1}} \in K$ is a $p$th root of $a$, a contradiction.  So the only possibility is that $n = k$, and so $h(X)$ must be a unit.  So, by definition, $f(X)$ is irreducible in $K[X]$.
\end{proof}

\item[16.] Let $\Char(K)  = p$.  Let $\alpha$ be algebraic over $K$.  Show that $\alpha$ is separable if and only if $K(\alpha) = K(\alpha^{p^n})$ for all positive integers $n$.

\begin{proof}
First, suppose $\alpha$ is separable, and consider $f(X) = X^{p^n} - \alpha^{p^n} \in K(\alpha^{p^n})[X]$.  Since $\alpha$ is a root of this polynomial, the minimal polynomial $g(X)$ of $\alpha$ over $K(\alpha^{p^n})$ divides $f(X)$.  If $g(X)$ is linear, then it must be $X - \alpha$ and so $\alpha \in K(\alpha^{p^n})$, as desired.  Otherwise, $g(X)$ must contain multiple factors of $X - \alpha$.  We know that $g(X)$ divides the minimal polynomial of $\alpha$ over $K$, and so in this case we know that $\alpha$ is a multiple root of its minimal polynomial over $K$, and so cannot be separable over $K$, a contradiction.  So it must be that $g(X) = X - \alpha$, meaning $K(\alpha^{p^n}) = K(\alpha)$.

For the converse, assume $\alpha$ is inseparable over $K$, so that its minimal polynomial $f(X)$ over $K$ has multiple roots.  As discussed in the proof of exercise 14, we must have that the derivative $f'(X) = 0$, and so $p$ divides the exponent of $X$ in every term of $f(X)$.  Thus, $f(X)$ is actually polynomial in $K[X^p]$.  If $\alpha^p$ is also a multiple root of $f(X)$, then by the same reasoning, $f(X)$ is a polynomial in $K[X^{p^2}]$.  This phenomenon can occur only finitely many times, since otherwise we would eventually end up at some $K[X^{p^m}]$ where $p^m$ exceeds the degree of $f(X)$, a contradiction.  So suppose $n$ is the largest integer such that $f(X) \in K[X^{p^n}]$.  Then $\alpha^{p^n}$ is a root of $f(X)$ (which is its minimal polynomial over $K$), but is separable over $K$.  Therefore, $K(\alpha^{p^n})$ is separable, and so cannot equal the inseparable extension $K(\alpha)$.
\end{proof}


\pagebreak
\item[17.] Prove that the following two properties are equivalent:
\begin{enumerate}
\item Every algebraic extension of $K$ is separable.
\item Either $\Char(K)  = 0$, or $\Char(K)  = p$ and every element of $K$ has a $p$-th root in $K$.
\end{enumerate}

\begin{proof}
Suppose $\Char(K) = 0$, and let $f(X)$ be irreducible over $K$.  Assume, for a contradiction, that $f(X)$ is inseparable, so that $f'(X)$ shares a root with $f(X)$.  Since $f(X)$ divides $f'(X)$, but $\deg f' < \deg f$, this means that $f'(X) = 0$.  The only possibility is that $f(X) \in K$, and so is not irreducible in $K[X]$ since it is a unit, a contradiction.

Now, suppose $\Char(K) = p$ and every element of $K$ has a $p$-th root in $K$.  Assume, for a contradiction, that some element $\alpha$ is not separable over $K$, and let $f(X)$ be its minimal polynomial.  Then $f(X) = a_nX^n + \cdots + a_1X + a_0$.  Each $a_i$ has a $p$th root $b_i$, and so
$$
f(X) = a_nX^n + \cdots + a_1X + a_0 = (b_nX^n + \cdots + b_1X + b_0)^p
$$
contradicting that $f(X)$ was irreducible.

For the converse, suppose that every algebraic extension of $K$ is separable but that $\Char(K) \neq 0$, so that $\Char(K) = p$.  Let $a \in K$ and consider the polynomial $f(X) = X^p - a$.  If $\alpha$ is a root of this in some algebraic closure, then the minimal polynomial of $\alpha$ over $K$ divides $f(X) = (X - \alpha)^p$, hence is of the form $(X - \alpha)^q$ for some $q \leq p$.  If $q > 1$ then $\alpha$ is not separable, a contradiction.  So $X - \alpha \in K[X]$, meaning $\alpha \in K$.  So every element of $K$ has a $p$th root in $K$.
\end{proof}

\item[18.] Show that every element of a finite field can be written as a sum of two squares in that field.

\begin{proof}
Let $K$ be the finite field of order $q = p^n$.  The multiplicative group of $K$ is cyclic of order $q-1$.  If $p = 2$, then $q-1$ is odd, and so every element of $K^\times$ is a square.  Since $0 = 0^2$, this means every element of $K$ is a square.  So assume $p \neq 2$.

In this case, $q-1$ is even.  The map $x \mapsto x^2$ is an endomorphism of $K^\times$.  Identifying $K^\times$ with $\Z_{q-1}$, we see that the kernel is $\{0, \frac{q-1}{2}\}$ and so the image of this map has $\dfrac{\# \Z_{q-1}}{\# \ker} = \dfrac{q-1}{2}$ elements.  Since $0$ is a square, there are exactly $\frac{q+1}{2}$ squares in $K$.

Let $x \in K$.  There must be at least one element which is both a square and is also of the form $x - a^2$ for some $a \in K$, since there are more than $\frac{\# K}{2}$ squares and more than $\frac{\# K} {2}$ elements of the form $x - a^2$.  Therefore, $x - a^2$ is a square for some $a \in K$, hence $a^2 + b^2 = x$ for some $b \in K$.
\end{proof}

\item[19.] Let $E$ be an algebraic extension of $F$.  Show that every subring of $E$ which contains $F$ is actually a field.  Is this necessarily true if $E$ is not algebraic over $F$?  Prove or give a counterexample.

\begin{proof}
Recall that if $\alpha$ is algebraic over $F$ with minimal polynomial $f(X)$, then $F[\alpha] = F / (f(X))$ is a field.  Let $F \subseteq R \subseteq E$ for a subring $R$, and let $\alpha \in R$.  $\alpha$ is algebraic, hence $\alpha^{-1} \in F[\alpha] \subseteq R$.  So $R$ is a field.

This is false if $E$ is not algebraic.  Take $F = \Q$ and $E = \Q(e)$.  Since $\Q[e] \cong \Q[X]$, we know that $\Q(e) \cong \Q(X)$.  Clearly, $\Q[X]$ is a subring of $\Q(X)$ that is not a field.
\end{proof}

\end{enumerate}
\end{document}


















