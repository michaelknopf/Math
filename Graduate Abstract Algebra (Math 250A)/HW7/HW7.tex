\documentclass[10pt]{article}
\usepackage[margin=1in]{geometry}
\addtolength{\oddsidemargin}{-.2in} 
\usepackage{amsmath,amsthm,amssymb}
\usepackage{bm}
\usepackage{enumitem}
\usepackage{array}
\usepackage{lipsum}
\usepackage[]{units}
\usepackage{relsize}
\usepackage{verbatim}
\usepackage{bbm}

\usepackage{tikz}
\usetikzlibrary  {positioning}
\usepackage{graphicx}
\usepackage{xfrac}
\usetikzlibrary{cd}


\setenumerate{listparindent=\parindent}

\newcommand{\Q}{\mathbf{Q}}
\newcommand{\Z}{\mathbf{Z}}
\newcommand{\R}{\mathbf{R}}
\newcommand{\C}{\mathbf{C}}
\newcommand{\p}{\mathfrak{p}}
\newcommand{\q}{\mathfrak{q}}
\renewcommand{\a}{\mathfrak{a}}
\renewcommand{\b}{\mathfrak{b}}
\renewcommand{\c}{\mathfrak{c}}
\renewcommand{\o}{\mathfrak{o}}
\newcommand{\m}{\mathfrak{m}}
\newcommand{\gen}[1]{\langle #1 \rangle}
\DeclareMathOperator*{\dom}{dom}
\DeclareMathOperator*{\Aut}{Aut}
\DeclareMathOperator*{\Ann}{Ann}
\DeclareMathOperator*{\Tor}{Tor}
\DeclareMathOperator*{\Gal}{Gal}
\DeclareMathOperator*{\Hom}{Hom}
\DeclareMathOperator*{\End}{End}
\DeclareMathOperator*{\im}{Im}
\let\ker\relax
\DeclareMathOperator*{\ker}{Ker}
\DeclareMathOperator*{\spn}{span}
\DeclareMathOperator*{\Perm}{Perm}
\DeclareMathOperator*{\card}{card}
\DeclareMathOperator*{\Alt}{Alt}
\DeclareMathOperator*{\id}{id}
\DeclareMathOperator*{\Pic}{Pic}
\renewcommand{\bar}{\overline}

\newtheorem*{lem}{Lemma}

\usepackage{fancyhdr} % Required for custom headers 
%\usepackage{lastpage} % Required to determine the last page for the footer

\pagestyle{fancy}
\lhead{Math 250A (HW 7)}
\chead{Michael Knopf (24457981)}
\rhead{October $15^\text{th}$, 2015}
\lfoot{}
\cfoot{}
\rfoot{}
%\rfoot{Page\ \thepage\ of\ \pageref{LastPage}}
\renewcommand\headrulewidth{0.4pt}
%\renewcommand\footrulewidth{0.4pt}

\begin{document}
\noindent A \emph{Dedekind ring} is defined to be a subring $\o$ of a field $K$ such that every element of $K$ is a quotient of elements of $\o$, and the fractional ideals form a multiplicative group.  Since a Dedekind ring is defined as a subring of a field, we know $\o$ is an integral domain.  Let $\o$ be a Dedekind ring and $K$ its quotient field.  Unless otherwise specified, all ideals are nonzero.

\begin{enumerate}
\setcounter{enumi}{12}
\item Every ideal is finitely generated.

\begin{proof}
Let $\a \subseteq \o$ be an ideal.  If $\a = 0$ then clearly $\a$ is finitely generated, so assume otherwise.  $\o$ is a Dedekind domain, so there is a fractional ideal $\b$ such that $\a\b = \o$, so $\sum a_i b_i = 1$ for some $a_i \in \a, b_i \in \b$, $i = 1, \dots , n$.  For any $a \in \a$, we know $a b_i \in \a\b = \o$.  Thus,
$$
a = a \sum a_i b_i = \sum (ab_i) a_i \in (a_1, \dots , a_n)
$$
since each $ab_i \in \o$.  So $\a \subseteq (a_1, \dots , a_n)$.  The reverse inclusion is obvious, since each $a_i$ is in $\a$.
\end{proof}

\item Every ideal has a factorization as a product of prime ideals, uniquely determined up to permutation.

\begin{proof}
First, note that $\o$ is Noetherian.  For, let $\a_1 \subseteq \a_2 \subseteq \cdots$ be a properly increasing chain of ideals in $\o$.  Then the union $\a = \bigcup_1^\infty \a_i$ is an ideal of $\o$ (we have shown this for increasing unions) and is thus generated by a finite set $(a_1, \dots , a_n)$.  For each $i=1,\dots,n$ there is some $k_i$ such that $a_i \in \a_{k_i}$.  Let $N = \max \{k_1, \dots , k_n\}$.  Then for all $m \geq N$, $a_i \in \a_m$ for all $i$, hence $\a \subseteq \a_m$.  But clearly $\a_m \subseteq \a$, hence we have equality.  Thus every properly increasing chain of ideals terminates, so $\o$ is Noetherian.

First consider the case of the zero ideal.  The proposition is technically false in this case: since $\o$ is an integral domain, $(0)$ is prime, thus we have factorizations $(0) = (0)\p_1 \cdots \p_n$ for any prime ideals $\p_1, \dots, \p_n$.  However, if $(0) = \p_1 \cdots \p_n$ were another factorization where none of the factors were $(0)$, then taking a nonzero element $p_i$ from each factor, we would have $p_1 \cdots p_n = 0$, contradicting that $\o$ is entire.  Therefore, any factorization of $(0)$ must contain $(0)$ as a factor.

Let $\a$ be a nonzero proper ideal of $\o$.  $\a$ is contained in a maximal (hence prime) nonzero ideal $\p_1$.  Let $\a_1 = \a\p_1^{-1}$.  Since $\a \subseteq \p_1$, we know $\a_1 = \a\p_1^{-1} \subseteq \p_1\p_1^{-1} = \o$, so $\a_1$ is an ideal of $\o$.  Now, if $\a_1$ is proper, then letting $\a_1$ take the place of $\a$, we find maximal ideal $\p_2$ containing $\a_1$, and again $\a_2 = \a_1\p_2^{-1}$ is an ideal of $\o$.  Continuing in this fashion, we have at the $n$th step produced a chain $\a_1 \subseteq \a_2 \subseteq \cdots \subseteq \a_n$.  If we were able to continue this process forever, it would create an infinite chain which never stabilizes, a contradiction.  So there is some $n$ such that $\a_{n} = \a_{n-1}\p_{n}^{-1}$ is not proper, i.e. $\a_{n-1}\p_{n}^{-1} = \o$.  But multiplication of ideals is associative, thus $\a_{n-1} = \p_{n}$ is prime.  This gives us a factorization
$$
\a = \a_1\p_1 = \a_2\p_2\p_1 = \cdots = \a_{n-1}\p_{n-1} \cdots \p_1 = \p_n \cdots \p_1
$$
of $\a$ into prime ideals.

One direction of the proof of exercise 17(a) is immediate: if $\a \mid \b$, then $\b = \a\c \subseteq \a$.  Also, if $\p$ contains a product $\a\b$, then it must contain one of $\a$ or $\b$.  If this were not the case, then there would be some $a \in \a, b \in \b$ such that $a,b \not \in \p$.  This is a contradiction, since $\p$ is prime and $ab \in \a\b \subseteq \p$.  Obviously, this extends inductively to a product of any number of ideals.

Now, suppose we have two factorizations $\p_1 \cdots \p_n = \a = \q_1 \cdots \q_m$ into prime ideals (say with $n \leq m$).  Assume without loss of generality that $\p_1$ is a minimal element of the set $\{\p_1, \dots , \p_n\}$, meaning that it is not properly contained in any of the others.  $\p_1$ divides the product $\q_1 \cdots \q_m$, hence it contains one of the factors; assume without loss of generality it is $\q_1$.  $\q_1$ divides the product $\p_1 \cdots \p_n$, thus it contains some $\p_k$.  This gives $\p_k \subseteq \q_1 \subseteq \p_1$.  By the minimality of $\p_i$, we must have equalities throughout, thus $\q_1 = \p_1$.  Since $\a$ is nonzero, $\p_1$ and $\q_1$ are nonzero, hence invertible.  Using the associativity of multiplication of fractional ideals, we can cancel them from the product, leaving us with $\p_2 \cdots \p_n = \q_2 \cdots \q_m$.

After repeating this process $n$ times, we will have shown the first $n$ factors to be equal (up to reordering).  If $n \neq m$, we will have $(1) = \q_{n+1} \cdots \q_m$.  This would imply that $\q_m$ divides, and thus contains, $(1)$ - contradicting that $\q_m$ is prime.  So we must have $n = m$, and the factors are equal up to permutation.
\end{proof}

\item Suppose $\o$ has only one prime ideal $\p$.  Let $t \in \p$ and $t \not \in \p^2$.  Then $\p = (t)$ is principal.

\begin{proof}

We cannot have $t = 0$ or else $t \in \p^2$.  Also, $(t) \neq \o$ or else $\o \subseteq \p$, contradicting that $\p$ is prime.  Thus, $(t)$ is a nonzero proper ideal, hence it has a unique factorization into prime ideals.  This must be of the form $\p^k$ for some $k \geq 1$, since $\p$ is the only prime ideal.  If $k \geq 2$, then $\p^2 \mid (t)$ and hence $(t) \subseteq \p^2$, a contradiction.  So $k = 1$, thus $(t) = \p$ is principal.

\end{proof}

\item Let $\o$ be any Dedekind ring.  Let $\p$ be a prime ideal.  Let $\o_\p$ be the local ring at $\p$.  Then $\o_\p$ is Dedekind and has only one prime ideal.

\begin{proof}

%\begin{comment}
First, we will develop some facts about the localization of an arbitrary integral domain $R$, with field of fractions $K$, at a multiplicative subset $S$.  Given an $R$-module $\a \subseteq K$, we define the \emph{extension} of $\a$ to be $S^{-1}\a = \{a/s \mid a \in \a, s \in S\}$, identifying the localization of $R$ at $S$ as a subring of $K$.  Note $S^{-1}\a$ is an $S^{-1}R$-module: if $a/s,b/t \in S^{-1}\a$ for some $a,b \in \a, s,t \in S$ then $\frac{a}{s} + \frac{b}{t} = \frac{as + bt}{st} \in S^{-1}\a$ since $as, bt \in \a$ and $st \in S$; also, if $c/r \in S^{-1}R$ then $\frac{c}{r}\frac{a}{s} = \frac{ca}{rs} \in S^{-1}\a$ since $ca \in \a$ and $rs \in S$.  So clearly if $\a$ is an ideal then $S^{-1}\a$ is as well.  Also, if $c\a \subseteq R$ for some $c \in R$, then $c\a^e \subseteq S^{-1}R$.  So extension preserves both ideals and fractional ideals.

Extension also distributes over multiplication of $R$-modules.  If $I,J \subseteq K$ are $R$-submodules, then
$$
S^{-1}(IJ) = \left\{\frac{\sum_i a_ib_i}{s} : a_i \in I, b_i \in J, s \in S \right\}
$$
$$
(S^{-1}I)(S^{-1}J) = \left\{ \sum_i \frac{a_i}{s_i}\frac{b_i}{t_i} \mid a_i \in I, b_i \in J, s_i,t_i \in S \right\} = \left\{ \sum_i \frac{a_ib_i \prod\limits_{i \neq j} s_j t_j}{\prod\limits_{i} s_i t_i} \mid a_i \in I, b_i \in J, s_i,t_i \in S \right\}.
$$
Given an element in $S^{-1}(IJ)$, we can express it in the form $\sum\limits_i \frac{a_i}{s_i}\frac{b_i}{t_i}$ by taking $s_1 = s, s_i = 1$ for $i > 1$, and $t_i = 1$ for all $i$.  Given an element of the form $\sum\limits_i \frac{a_ib_i \prod\limits_{i \neq j} s_j t_j}{\prod\limits_{i} s_i t_i}$, we know $a_i \prod\limits_{j \neq i} s_j \in I$ and $b_i \prod\limits_{j \neq i} t_j \in J$ since $I$ and $J$ are $R$-modules, and $\prod\limits_{i} s_it_i \in S$, giving the reverse inclusion.  So $S^{-1}(IJ) = (S^{-1}I)(S^{-1}J)$.

Note also that, if $\a$ is an ideal of $S^{-1}R$, then $\a \cap R = \{a \mid a/s \in \a \text{ for some } s \in S\}$ is an ideal of $R$: the intersection of submodules is a submodule, and the $S^{-1}R$-action on $\a$ restricts to an action of $R$ on $\a$; hence both $\a$ and $R$ are $R$-modules.  Also, the extension $S^{-1}(\a \cap R)$ of $\a \cap R$ is $\a$, since if $a/s \in \a$ for some $s \in S$, then $a/s \in \a$ for all $s \in S$ because $\a$ is closed under multiplication by $S^{-1}R$.

Consider an ideal $\a$ of $\o_\p$.  Here, we will denote the extension of an ideal $\b$ by $\b_\p$.  $\a \cap \o$ has an inverse $(\a \cap \o)^{-1}$, which is a fractional ideal of $\o$.  So
$$
\a ((\a \cap \o)^{-1})_\p = ((\a \cap \o)(\a \cap \o)^{-1})_\p = \o_\p
$$
thus $((\a \cap \o)^{-1})_\p$ is the inverse of $\a$.  Now, if $\b$ is a fractional ideal of $\o_\p$, then there is some $c/s \in \o_\p$ such that $\frac{c}{s}\b$ is an ideal of $\o_\p$.  It has an inverse $\a$, which is a fractional ideal.  But then $\o_\p = (\frac{c}{s} \b) \a = \b (\frac{c}{s}\a)$, hence $\frac{c}{s}\a$ is the inverse of $\b$.  So all fractional ideals of $\o_\p$ are invertible, thus $\o_\p$ is Dedekind.

In exercise 18, we show (without using this result) that prime ideals of a Dedekind domain are maximal.  Thus, $\o_\p$ has a unique prime ideal.
%\end{comment}
\end{proof}

\item As for the integers, we say $\a \mid \b$ if there exists and ideal $\c$ such that $\b = \a\c$.  Prove:
\begin{enumerate}
\item $\a \mid \b$ if and only if $\b \subseteq \a$.
\begin{proof}
If $\a \mid \b$, then there is some ideal $\c$ such that $\b = \a\c \subseteq \a$.  Suppose now that $\b \subseteq \a$.  Then $\b \a^{-1} \subseteq \a \a^{-1} = \o$.  There is some nonzero $k \in \o$ such that $k \a^{-1} \subseteq \o$, so $\b(k\a^{-1}) \subseteq k\a\a^{-1} = (k)$.  Thus, every element of $\b(k\a^{-1})$ is divisible by $k$, hence $(k)$ divides $\b (k\a^{-1})$.  So there is some ideal $\c$ such that $(k)\c = \b(k\a^{-1})$.  So $(k)\c\a = \b(k\a^{-1})\a = \b(k)$.  Since $(k) \neq 0$, it is invertible, thus $\c\a = \b$.  So $\a \mid \b$.
\end{proof}
\item Let $\a,\b$ be ideals.  Then $\a + \b$ is their greatest common divisor.  In particular, $\a,\b$ are relatively prime if and only if $\a + \b = \o$.
\begin{proof}
Suppose $\c \mid \a$ and $\c \mid \b$.  Then $\c \supseteq \a$ and $\c \supseteq \b$, thus $\c \supseteq \a + \b$ and so $\c \mid \a + \b$.  By definition, $\a + \b$ is the greatest common divisor of $\a$ and $\b$.

If $\a+\b = \o$, then every common divisor of $\a$ and $\b$ contains $\o$, hence the only one is $\o$.  So $\a$ and $\b$ are relatively prime.  Conversely, if the only common divisor of $\a$ and $\b$ is $\o$, then the only divisor of $\a+\b$ is $\o$.  So the only ideal containing $\a+\b$ is $\o$.  So $\a+\b$ is not contained in a maximal ideal, hence it must be $\o$.
\end{proof}
\end{enumerate}
\item Every prime ideal $\p$ is maximal.  In particular, if $\p_1, \dots , \p_n$ are distinct primes, then the Chinese remainder theorem applies to their powers $\p_1^{r_1}, \dots , \p_n^{r_n}$.

\begin{proof}
Let $\p$ be a prime ideal.  $\p$ is contained in a maximal ideal $\m$, so $\m \mid \p$.  Due to unique factorization, $\m = \p$.  So $\p$ is maximal.  By uniqueness of prime factorizations, the only divisors of $\p_i^{r_i}$ are of the form $\p_i^{s_i}$ for $s_i \leq r_i$, and the only factors of $\p_j^{r_j}$ are of the form $\p_j^{s_j}$ where $s_j \leq r_j$.  The only ideal that is of both these forms has $s_i = s_j = 0$, which means it is $\o$.  Since $\p_i^{r_i} + \p_j^{r_j}$ divides $\p_i^{r_i}$ and $\p_j^{r_j}$, it must be $\o$.  So the Chinese Remainder Theorem applies.
\end{proof}

\item Let $\a,\b$ be ideals.  Show that there exists an element $c \in K$ such that $c\a$ is an ideal relatively prime to $\b$.  In particular, every ideal class in $\Pic(\o)$ contains representative ideals prime to a given ideal.

\begin{proof}
Let the prime factors of $\b$ be $\p_1, \dots , \p_n$, and represent $\a$ as $\a = \p_1^{r_1} \cdots \p_n^{r_n}\p_{n+1}^{r_{n+1}} \cdots \p_{n+m}^{r_{n+m}}$, where the $\p_i$ are distinct primes and each $r_i \geq 0$.  There exists some $a \in \o$ such that $a \equiv x_i \pmod{\p_i^{r_i+1}}$ for each $i$, where $x_i \in \p_i^{r_i} \setminus \p_i^{r_i+1}$.  This guarantees that $(a)$ factors as
$$
(a) = \p_1^{r_1} \cdots \p_n^{r_n}\a_{1}^{s_1} \cdots \a_{k}^{s_{k}}
$$
where the $\a_i$ are primes distinct from each other and from the $\p_i$.  Next, find $b \in \o$ for which $b \equiv 0 \pmod{\a_i^{s_i}}$ for all $i \leq k$, but $b \equiv 1 \pmod{\p_i}$ for all $i \leq n$.  Thus,
$$
(b) = \c\a_{1}^{s_1} \cdots \a_{k}^{s_{k}}
$$
where $\c$ is relatively prime to $\b$ (it is possible that $\c$ has some factors of $\a_i$, but we have guaranteed it has no factors of any $\p_i$).  Letting $c = \frac{b}{a}$, we now have
\begin{align*}
c\a &= (b)(a)^{-1}\a \\
&= (\c\a_{1}^{s_1} \cdots \a_{k}^{s_{k}})(\p_1^{-r_1} \cdots \p_n^{-r_n}\a_{1}^{-s_1} \cdots \a_{k}^{-s_{k}})(\p_1^{r_1} \cdots \p_n^{r_n}\p_{n+1}^{r_{n+1}} \cdots \p_{n+m}^{r_{n+m}}) \\
&= \c\p_{n+1}^{r_{n+1}} \cdots \p_{n+m}^{r_{n+m}}
\end{align*}
which is an ideal of $\o$ relatively prime to $\b$.

For a given ideal $\b$ and a given ideal class $C \in \Pic(\o)$, choose any ideal $\a \in C$ and let $c\a$ be relatively prime to $\b$.  $c\a \in C$ because $(c)$ is principal, therefore $C$ contains a representative relatively prime to any fixed ideal.
\end{proof}

\setcounter{enumi}{10}
\begin{comment}
\item Let $M$ be a finitely generated torsion-free module over $\o$.  Prove that $M$ is projective.

\begin{proof}

Given a prime ideal $\p$, the localized module $M_\p$ is generated over $\o_\p$ by any generating set of $M$ over $\o$, hence is finitely generated.  If $\frac{a}{s}m = 0$ for some $m \in M_\p$, then multiplication by $s$ gives that $am = 0$.  Thus, since $M$ is torsion free, $a=0$ or $m=0$, so $M_\p$ is also torsion free.  $\o_\p$ is a principal ideal domain by the combination of exercises 15 and 16: there is a unique prime ideal $\q$ of $\o_\p$, so every ideal factors as $\q^k$ for some $k$; but there is some $t \in \q \setminus \q^2$, hence $\q = (t)$ is principal, and so every ideal is principal of the form $(t^k)$.  By Theorem 7.3, $M_\p$ is free and therefore projective.

Let $F$ be finite free over $\o$ and $f:F \rightarrow M$ surjective.  $f$ extends naturally to a homomorphism $f_\p:F_\p \rightarrow M_\p$ by $f_\p(\frac{a}{s}x) = \frac{a}{s}f(x)$ for $s \not \in \p$, which is surjective because every element of $M_\p$ is of the form $\frac{a}{s}m$ for some $m \in M$.  So the sequence
$$
\begin{tikzcd}
0 \arrow[r]& \ker f_\p \arrow[r] & F_\p \arrow[r, "f_\p"'] & M_\p \arrow[r] \arrow[l, bend right, "g_\p"'] & 0
\end{tikzcd}
$$
is exact, hence it yields a splitting homomorphism $g_\p$.

As a homomorphism of $\o_\p$-modules, $g_\p$ is naturally a homomorphism of $\o$-modules as well.  Since it is a right inverse for $f_\p$, it is injective, hence gives an embedding $g_\p(M)$ of $M$ into $F_\p$ (as an $\o$-module).  This embedding has a finite generating set $\{m_1, \dots , m_n\}$, and for each $i$ there is some $c_i \not \in \p$ for which $c_i m_i \in F$.  Let $c_\p$ be the product of all the $c_i$.  We know $c_\p \not \in \p$ since $\p^c$ is multiplicative, and that $c_\p m_i \in F$ for all $i$.  Thus $c_\p g_p(M) \subseteq F$.

Consider the ideal $\a$ generated by $\{c_\p : \p \subseteq \o \text{ is prime}\}$.  If $\a \neq \o$, then $\a$ is contained in some maximal (hence prime) ideal $\q$.  But $c_\q \not \in \a$ contradicts that $c_\q \in \a \subseteq \q$.  So we must have $\a = \o$.  So there are some finite collections $\{c_{\p_i}\}$ and $\{x_i\} \subseteq \o$ for which $\sum x_i c_{\p_i} = 1$.  Letting $g = \sum x_i c_{\p_i} g_{\p_i}$, we have $g: M \rightarrow F$ and
$$
f \circ g(m) = \sum x_i f (c_{\p_i} g_{\p_i}(m)) = \sum x_i f_{\p_i} (c_{\p_i} g_{\p_i}(m)) = \sum x_i c_{\p_i} f_{\p_i} \circ g_{\p_i}(m) = m\sum x_i c_{\p_i} = m
$$
for any $m \in M$.  Thus $f \circ g = \id_M$.  This means that the sequence
$$
\begin{tikzcd}
0 \arrow[r] & \ker f \arrow[r] & F \arrow[r] & M \arrow[r] & 0
\end{tikzcd}
$$
splits, thus $F = \ker f \oplus M$.  So $M$ is a direct summand of a free module, hence projective.
\end{proof}

\item
\begin{enumerate}
\item Let $\a,\b$ be ideals.  Show that there is an isomorphism of $\o$-modules
$$
\a \oplus \b \cong \o \oplus \a\b.
$$
\item Let $\a,\b$ be fractional ideals, and let $f:\a \rightarrow \b$ be an isomorphism (of $\o$-modules).  Then $f$ has an extension to a $K$-linear map $f_K:K \rightarrow K$.  Let $c = f_K(1)$.  Show that $\b = c\a$ and that $f$ is given by the mapping $m_c: x \rightarrow cs$.
\item Let $\a$ be a fractional ideal.  For each $b \in \a^{-1}$ the map $m_b:\a \rightarrow \o$ is an element of the dual $\a^{\vee}$.  Show that $\a^{-1} = \a^\vee = \Hom_\o (\a,\o)$ under this map, and so $\a^{\vee \vee} = \a$.
\end{enumerate}
\item
\begin{enumerate}
\item Let $M$ be a projective finite module over the Dedekind ring $\o$.  Show that there exist free modules $F$ and $F'$ such that $F \supseteq M \supseteq F'$, and $F,F'$ have the same rank, which is called the rank of $M$.
\item Prove that there exists a basis $\{e_1, \dots , e_n\}$ of $F$ and ideals $\a_1, \dots , \a_n$ such that $M = \a_1 e_1 + \cdots + \a_ne_n$, or in other words, $M \cong \oplus \a_i$.
\item Prove that $M \cong = \o^{n-1} \oplus \a$ for some ideal $\a$, and that the association $M \mapsto \a$ induces an isomorphism of $K_0(\o)$ with the group of ideal classes $\Pic(\o)$.
\end{enumerate}
\end{comment}
\end{enumerate}
\end{document}


















