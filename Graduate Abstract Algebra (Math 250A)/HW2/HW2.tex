\documentclass[10pt]{article}
\usepackage[margin=1in]{geometry}
%\addtolength{\oddsidemargin}{-.1in} 
\usepackage{amsmath,amsthm,amssymb}
\usepackage{bm}
\usepackage{enumitem}
\usepackage{array}
\usepackage{lipsum}
\usepackage[]{units}
\usepackage{relsize}
\usepackage{verbatim}
\usepackage{bbm}

\usepackage{tikz}
\usetikzlibrary{positioning}
\usepackage{graphicx}
\usepackage{xfrac}

\setenumerate{listparindent=\parindent}

\newcommand{\Q}{\mathbb{Q}}
\newcommand{\Z}{\mathbb{Z}}
\newcommand{\R}{\mathbb{R}}
\newcommand{\gen}[1]{\langle #1 \rangle}
\DeclareMathOperator*{\dom}{dom}
\DeclareMathOperator*{\Aut}{Aut}
\DeclareMathOperator*{\Ann}{Ann}
\DeclareMathOperator*{\Tor}{Tor}
\DeclareMathOperator*{\Gal}{Gal}
\DeclareMathOperator*{\Hom}{Hom}
\DeclareMathOperator*{\End}{End}
\DeclareMathOperator*{\im}{Im}
\DeclareMathOperator*{\Perm}{Perm}
\renewcommand{\bar}{\overline}

\newtheorem*{lem}{Lemma}

\usepackage{fancyhdr} % Required for custom headers 
%\usepackage{lastpage} % Required to determine the last page for the footer

\pagestyle{fancy}
\lhead{Math 250A (HW 2)}
\chead{Michael Knopf (24457981)}
\rhead{September $10^\text{th}$, 2015}
\lfoot{}
\cfoot{}
\rfoot{}
%\rfoot{Page\ \thepage\ of\ \pageref{LastPage}}
\renewcommand\headrulewidth{0.4pt}
%\renewcommand\footrulewidth{0.4pt}

\begin{document}

\begin{lem}
Suppose $H$ and $K$ are subgroups of $G$, with trivial intersection, such that each element of one commutes with each element of the other.  Then $HK \cong H \times K$.  Specifically, this holds if $H$ and $K$ are subgroups of $G$, with trivial intersection, that normalize each other.
\end{lem}
\begin{proof}
Suppose $H$ and $K$ are as described in the first sentence.  By the second isomorphism theorem, $HK$ is a subgroup of $G$ (since clearly $H$ and $K$ normalize each other).  Define $\varphi: H \times K \rightarrow HK$ by $\varphi(h,k) = hk$.  The map is a homomorphism because
$$
\varphi((x,y)(z,w)) = \varphi(xz,yw) = xzyw = xyzw = \varphi(x,y)\varphi(z,w).
$$
If $(h,k) \in \ker \varphi$, then $hk = 1$, so $h,k \in H \cap K$, and thus $h = k = 1$.  $\varphi$ is clearly surjective as well, therefore it is an isomorphism.

For the latter statement, suppose $H$ and $K$ have trivial intersection and normalize each other.  Then, for any $h \in H$ and $k \in K$, we have
$$
K \ni (h^{-1}k^{-1}h)k = h^{-1}(k^{-1}hk) \in H.
$$
so $h^{-1}k^{-1}hk = 1$, thus all elements of $H$ commute with those of $K$.
\end{proof}

\begin{enumerate}
\item[23.] Let $P, P'$ be $p$-Sylow subgroups of a finite group $G$.
\begin{enumerate}
\item If $P' \subseteq N(P)$, then $P' = P$.
\begin{proof}
Let $|P| = p^n$, and suppose $P'$ normalizes $P$.  Then by the second isomorphism theorem, $P'P$ is a subgroup of $G$ with order $\dfrac{|P'||P|}{|P' \cap P|} = \dfrac{p^{2n}}{p^k}$ for some $k \leq n$.  This is because $|P' \cap P|$ is a subgroup of $P$, hence its order divides $p^n$.  If $k \neq n$, then $|P'P|$ has order $p^m$ where $m > n$, contradicting that $p^n$ is the highest power of $p$ dividing $|G|$.  Thus $k = n$, and so $|P' \cap P| = |P|$, hence $P' = P$.
\end{proof}
\item If $N(P') = N(P)$, then $P' = P$.
\begin{proof}
Since $P' \subseteq N(P') = N(P)$, this follows from the previous result.
\end{proof}
\item We have $N(N(P)) = N(P)$.
\begin{proof}
Clearly, $N(P) \subseteq N(N(P))$.  For the reverse inclusion, suppose $g \in N(N(P))$.  Then $gPg^{-1} \subseteq N(P)$, since $gN(P)g^{-1} = N(P)$ and $P \subseteq N(P)$.  But $gPg^{-1}$ is a p-Sylow subgroup, thus by part (a) we know that $gPg^{-1} = P$.  So $g \in N(P)$.
\end{proof}
\end{enumerate}

\item[24.] Let $p$ be a prime number.  Show that a group of order $p^2$ is abelian, and that there are only two such groups up to isomorphism.

\begin{proof}
Since $G$ is nontrivial, it has a nontrivial center $Z$.  If $Z = G$, then $G$ is abelian, so suppose instead that $|Z| = p$.  Then $G / Z \cong Z_p$.  \textbf{We demonstrated in the previous homework (in the course of showing that if $\Aut(G)$ is cyclic then $G$ is abelian) that if the quotient of a group by its center is cyclic, then the group is abelian.}  Thus $G$ is abelian (this case turns out to be vacuous, but still we have $G$ abelian in all cases).

Suppose $G \not \cong Z_{p^2}$.  Then all non-identity elements have order $p$.  Let $x,y \in G$ be non-identity elements such that $y \not \in \gen{x}$.  Then we must have $\gen{x} \cap \gen{y} = \{1\}$, since $\gen{x}$ and $\gen{y}$ are both cyclic of prime order, so if their intersection was nontrivial then any nonidentity element would necessarily be a generator for both of them (a contradiction).  Since $G$ is abelian, $\gen{x}$ and $\gen{y}$ are normal.  Also, $|\gen{x}\gen{y}| = \frac{|\gen{x}||\gen{y}|}{|\gen{x} \cap \gen{y}|} = |\gen{x}||\gen{y}| = |G|$, thus $G = \gen{x}\gen{y}$.  By the lemma, then, $G \cong \gen{x} \times \gen{y} \cong \Z_p^2$.

%Define a map $\varphi: \Z_p^2 \rightarrow G$ by $\varphi(a,b) = x^ay^b$.  This is a homomorphism because $G$ is abelian, so $\varphi(a+c,b+d) = (x^ay^b)(x^cy^d) = x^{a+c}y^{b+d} = \varphi(a+c,b+d)$.  The kernel is the set of $(a,b) \in \Z_p^2$ such that $x^ay^b = 1$.  If $a$ or $b$ are nonzero and solve this equation, then $x^a = y^{-b} \in \gen{y}$, contradicting that $\gen{x} \cap \gen{y} = \{1\}$.  So $\varphi$ is injective, and thus is an isomorphism since $|G| = |\Z_p^2|$.  So $G$ is isomorphic to either $\Z_{p^2}$ or $\Z_p^2$.

%Then $G = \gen{x,y}$ because $\{1\} = \gen{x} \cap \gen{y} \subsetneq \gen{x} \subsetneq \gen{x,y}$, hence $|\gen{x,y}| = p^2$.  But $G$ is abelian, so $G = \gen{x,y} = \gen{x}\gen{y} \cong \gen{x}\times \gen{y} \cong \Z_p^2$.

\end{proof}

\pagebreak
\item[25.] Let $G$ be a group of order $p^3$, where $p$ is prime, and $G$ is not abelian.  Let $Z$ be its center.  Let $C$ be a cyclic group of order $p$.
\begin{enumerate}
\item Show that $Z \cong C$ and $G/Z \cong C \times C$.
\begin{proof}
Since $G$ is a nontrivial $p$-group, it has a nontrivial center.  But $G$ is not abelian, so $Z \neq G$.  This leaves $|Z| = p$ and $|Z| = p^2$ as possibilities.  We cannot have $|Z| = p^2$, or else $G/Z$ has order $p$ and is thus cyclic, contradicting that $G$ is not abelian.  So $Z$ has order $p$, and is thus isomorphic to $C$.

Now, $G / Z$ has order $p^2$.  By the result of the previous exercise, it is isomorphic to either $C$ or $C^2$.  But if $G/Z \cong C$, then again we must have that $G$ is abelian, a contradiction.  So $G/Z \cong C^2$.
\end{proof}
\item Every subgroup of $G$ of order $p^2$ contains $Z$ and is normal.
\begin{proof}
Let $H$ be such a subgroup.  Clearly, $H$ is normal because its index in $G$ is $p$, which is the smallest prime dividing the order of $G$.  Also, by exercise 24, $H$ must be abelian.

Now, suppose that $H$ does not contain $Z$.  Then we must have $H \cap Z = \{1\}$, since $Z$ is generated by any one of its nontrivial elements.  So $G = HZ$ by the second isomorphism theorem (since $H$ is normalized by $Z$ and $|HZ| = \frac{|H||Z|}{|H\cap Z|} = p^3$).  But $H$ is abelian, and all of its elements commute with those of $Z$, so for any $h,k \in H$ and $x,y \in Z$ we have
$$
(hx)(ky) = (hk)(xy) = (kh)(yx) = (ky)(hx)
$$
contradicting that $G$ is not abelian.  So $H$ must contain the center.
\end{proof}
\item Suppose $x^p = 1$ for all $x \in G$.  Show that $G$ contains a normal subgroup $H \cong C \times C$.
\begin{proof}
We know that $G/Z \cong C^2$.  $C^2$ has a subgroup of order $p$ (for instance, $C \times \{0\}$), and this subgroup naturally lifts to a subgroup $H$ of $G$ such that $H / Z \cong C$ (by the third isomorphism theorem).  So $|H| = |Z||C| = p^2$.  By part (b), $H$ is normal in $G$.  By exercise 24, $H$ is isomorphic to either $\Z_{p^2}$ or $C^2$.  However, $\Z_{p^2}$ contains an element of order $p^2$, contradicting that $x^p = 1$ for all $x \in G$.  Thus $H \cong C^2$.
\end{proof}
\end{enumerate}
\item[26.]
\begin{enumerate}
\item Let $G$ be a group of order $pq$, where $p,q$ are primes and $p<q$.  Assume that $q \not \equiv 1 \pmod{p}$.  Prove that $G$ is cyclic.
\begin{proof}
Let $Q$ be a $q$-Sylow subgroup and $P$ a $p$-Sylow subgroup.  %$Q$ has index $p$ in $G$, which is the smallest prime divisor of $pq$, so $Q$ is normal.
Since $p < q$, we again must have $P \cap Q = \{1\}$ since any nonidentity element in the intersection would have to generate both $P$ and $Q$.  The conjugation action of $P$ on $Q$ gives a homomorphism of $P$ into the automorphism group of $Q$.

$Q$ is a cyclic group of order $q$.  For a fixed nonidentity element $x \in Q$, each automorphism is defined by its action on $x$.  Specifically, there are $q-1$ automorphisms, each sending $x$ to a different nonidentity element of $Q$.  The kernel $K$ of $P$'s action on $Q$ must either be $\{1\}$ or $P$, since these are the only subgroups of $P$.  If $K = \{1\}$ then $P \cong P/K \cong \im \varphi \subseteq \Aut Q$, and so $p$ divides $q-1$.  However, this contradicts that $q \not \equiv 1 \pmod{p}$, thus we must have $K = P$.  So the action is trivial, meaning that $pqp^{-1} = q$ for all $p \in P$ and $q \in Q$, hence every element of $P$ commutes with every element of $Q$.

By the lemma, $PQ \cong P \times Q$.  Also, $G = PQ$ because $|PQ| = \frac{|P||Q|}{|P\cap Q|} = pq$.  Since $P$ and $Q$ are cyclic with relatively prime orders, $P \times Q$ is cyclic, therefore $G$ is cyclic.
\end{proof}
\item Show that every group of order 15 is cyclic.
\begin{proof}
$15 = 3 \cdot 5$, and $5 \equiv 2 \not \equiv 1 \pmod{3}$, hence all groups of order 15 are cyclic by the result of part (a).
\end{proof}
\end{enumerate}

\pagebreak
\item[27.] Show that every group of order $< 60$ is solvable.
\begin{proof}
The trivial group is solvable by definition.  Now, let $n < 60$ and consider a group $G$ of order $n$.  Suppose we have shown for all $m < n$ that all groups of order $m$ are solvable.  If $n$ is prime, then $G$ is cyclic and so is obviously solvable.  Otherwise, suppose we can find a proper nontrivial normal subgroup $N \subsetneq G$.  Then $|N|, |G/N| < n$, so by the inductive hypothesis we have abelian towers $1 = N_0 \subseteq N_1 \subseteq \cdots \subseteq N_j = N$ and $N/N = H_0 / N \subseteq H_1 / N \subseteq \cdots \subseteq H_k / N = G / N$ (the lattice isomorphism theorem tells us that the abelian tower for $G/N$ must take this form, where $H_i \trianglelefteq H_{i+1}$ for each $i$).  This yields an abelian tower for $G$:

$$
1 = N_0 \subseteq N_1 \subseteq \cdots \subseteq N_j = N = H_0 \subseteq H_1 \subseteq \cdots \subseteq H_k = G.
$$
Therefore, it suffices to show that $G$ is not simple.

For many $n < 60$, we can easily verify that $G$ is not simple unless it is cyclic.  All nontrivial $p$-groups have nontrivial center, hence they are simple only if they are cyclic.  If $|G| = pq^s$ for some primes $p < q$ and some integer $s \geq 1$, then a $q$-Sylow subgroup has index $p$, which is the smallest prime dividing $|G|$, and thus is normal.

For a more specific case, suppose $n = p^2q$ for some primes $p < q$.  We know that $n_p \mid q$ and $n_q \mid p^2$.  If either $n_p = 1$, then the $p$-Sylow subgroup $P$ is stabilized by conjugation, hence is normal (and similarly if $n_q = 1$).  So we may assume $n_p = q$ and $n_q = p$ or $p^2$.  If $n_q = p^2$, we have a combinatorial issue regarding the size of $G$.  Since the $q$-Sylows are cyclic, their pairwise intersections must be trivial.  Similarly, they must have trivial intersection with each $p$-Sylow as well, else their intersection would generate an order $q$ subgroup of a $p$-Sylow, contradicting that $q \nmid p$.  So these $q$-Sylows, along with just one of the $p$-Sylows, account for $p^2(q-1) + (p^2 - 1) + 1 + p^2q = |G|$ elements of the group.  This leaves no room for any more $p$-Sylows, contradicting that $n_p = 1$.  Therefore, $G$ contains a normal subgroup (either a $p$-Sylow or a $q$-Sylow).

Next, suppose $p$ divides $n$ with multiplicity $s$, and that $p > \frac{n}{p^s}$.  Since $n_p \mid \frac{n}{p^2} < p$ and $n_p \equiv 1 \pmod{p}$, we must have $n_p = 1$.  Therefore, the single $p$-Sylow is stabilized by conjugation, and thus is normal.

Again, suppose $p$ divides $n$ with multiplicity $s$.  Consider the action of $G$ on the set $S$ of cosets of a $p$-Sylow subgroup $P$ by conjugation.  Assume that $G$ is simple.  Then the kernel $K$ of the action is either $\{1\}$ or $G$.  If $K = G$, however, then every element of $G$ stabilizes every coset of $P$.  In particular, they all stabilize $P$ itself, thus $P$ is normal - a contradiction.  Therefore, $K = \{1\}$.  This gives an embedding of $G$ into $\Perm (S)$, which has order $(\frac{n}{p^s})!$.  Therefore, if $(\frac{n}{p^s})! < n$, then $G$ cannot be simple.

After applying each of these results to as many cases as possible, we have eliminated all cases except for $n = 30, 40,$ and $56$ (see the table below).  For $n = 30$, we have $n_5 \mid 6$ and $n_5 \equiv 1 \pmod{5}$, so we may assume $n_5 = 6$.  $n_3 \mid 10$ and $n_3 \equiv 1 \pmod{3}$, so we may assume $n_3 = 10$.  Since $5$ and $3$ each divide $n$ with multiplicity $1$, all pairwise intersections between any two $3$- or $5$-Sylows must be trivial.  So these alone must account for $6(5-1) + 10(3-1) + 1 = 45$ elements of $G$, a contradiction.  For $n = 40$, we know $n_5$ divides $8$ and is congruent to $1 \pmod{5}$, leaving $n_5 = 1$ as the only possibility.  For $n = 56$, we have $n_7 \mid 8$ and $n_7 \equiv 1 \pmod{7}$.  So $n_7 = 8$.  There is also at least one $2$-Sylow of size $8$.  But these together account for $8(7-1) + (8-1) + 1 = 56$ elements of the group, leaving no room for any more $2$-Sylows (since another would have to have at least one element not yet accounted for).

\renewcommand{\arraystretch}{1.2}

\begin{center}
\begin{tabular}{|c|c|c||c|c|c|}
\hline
$n$ & \textbf{Factorization} &  \textbf{Reason} & $30$ & $2 \cdot 3 \cdot 5$ & • \\ 
\hline 
$1$ & • & trivial & $31$ & $31$ & cyclic \\ 
\hline 
$2$ & $2$ & cyclic & $32$ & $2^5$ & $p$-group \\ 
\hline 
$3$ & $3$ & cyclic & $33$ & $3 \cdot 11$ & $pq^s$ \\ 
\hline 
$4$ & $2^2$ & $p$-group & $34$ & $2 \cdot 17$ & $pq^s$ \\ 
\hline 
$5$ & $5$ & cyclic & $35$ & $5 \cdot 7$ & $pq^s$ \\ 
\hline 
$6$ & $2\cdot 3$ & $pq^s$ & $36$ & $2^2 \cdot 3^2$ & $n > \frac{n}{p^s}!$ \\ 
\hline 
$7$ & $7$ & cyclic & $37$ & $37$ & cyclic \\ 
\hline 
$8$ & $2^3$ & $p$-group & $38$ & $2 \cdot 19$ & $pq^s$ \\ 
\hline 
$9$ & $3^2$ & $p$-group & $39$ & $39$ & cyclic \\ 
\hline 
$10$ & $2\cdot 5$ & $pq^s$ & $40$ & $2^3 \cdot 5$ & • \\ 
\hline 
$11$ & $11$ & cyclic & $41$ & $41$ & cyclic \\ 
\hline 
$12$ & $2^2 \cdot 3$ & $p^2q$ & $42$ & $2 \cdot 3 \cdot 7$ & $p > \frac{n}{p^s}$ \\ 
\hline 
$13$ & $13$ & cyclic & $43$ & $43$ & cyclic \\ 
\hline 
$14$ & $2 \cdot 7$ & $pq^s$ & $44$ & $2^2 \cdot 11$ & $p > \frac{n}{p^s}$ \\ 
\hline 
$15$ & $3\cdot 5$ & $pq^s$ & $45$ & $3^2 \cdot 5$ & $p^2q$ \\ 
\hline 
$16$ & $2^4$ & $p$-group & $46$ & $2 \cdot 23$ & $pq^s$ \\ 
\hline 
$17$ & $17$ & cyclic & $47$ & $47$ & cyclic \\ 
\hline 
$18$ & $2 \cdot 3^2$ & $pq^s$ & $48$ & $2^4 \cdot 3$ & $n > \frac{n}{p^s}!$ \\ 
\hline 
$19$ & $19$ & cyclic & $49$ & $7^2$ & $p$-group \\ 
\hline 
$20$ & $2^2 \cdot 5 $ & $p > \frac{n}{p^s}$& $50$ & $2 \cdot 5^2$ & $pq^s$ \\ 
\hline 
$21$ & $3 \cdot 7$ & $pq^s$ & $51$ & $3 \cdot 17$ & $pq^s$ \\ 
\hline 
$22$ & $2 \cdot 11$ & $pq^s$ & $52$ & $2^2 \cdot 13$ & $p > \frac{n}{p^s}$ \\ 
\hline 
$23$ & $23$ & cyclic & $53$ & $53$ & cyclic \\ 
\hline 
$24$ & $2^3 \cdot 3$ & $n > \frac{n}{p^s}!$ & $54$ & $2 \cdot 3^3$ & $pq^s$ \\ 
\hline 
$25$ & $5^2$ & $p$-group & $55$ & $5 \cdot 11$ & $pq^s$ \\ 
\hline 
$26$ & $2 \cdot 13$ & $pq^s$ & $56$ & $2^3 \cdot 7$ & • \\ 
\hline 
$27$ & $3^3$ & $p$-group & $57$ & $3 \cdot 19$ & $pq^s$ \\ 
\hline 
$28$ & $2^2 \cdot 7$ & $p > \frac{n}{p^s}$ & $58$ & $2 \cdot 29$ & $pq^s$ \\ 
\hline 
$29$ & $29$ & cyclic & $59$ & $59$ & cyclic \\ 
\hline 
\end{tabular}
\end{center}

\end{proof}
\end{enumerate}
\end{document}







