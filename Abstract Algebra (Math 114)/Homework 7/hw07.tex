\documentclass[10pt]{article}
\usepackage[margin=.7in]{geometry} 
\usepackage{amsmath,amsthm,amssymb}
\usepackage{bm}
\usepackage{enumitem}
\usepackage{array}
\usepackage{lipsum}
\usepackage[]{units}
\usepackage{relsize}

\usepackage{tikz}
\usetikzlibrary{positioning}
\usepackage{graphicx}
\usepackage{xfrac}

\setenumerate{listparindent=\parindent}

\newcommand{\Q}{\mathbb{Q}}
\newcommand{\Z}{\mathbb{Z}}
\DeclareMathOperator*{\dom}{dom}
\DeclareMathOperator*{\Aut}{Aut}
\DeclareMathOperator*{\Tor}{Tor}
\DeclareMathOperator*{\Gal}{Gal}

\usepackage{fancyhdr} % Required for custom headers 
%\usepackage{lastpage} % Required to determine the last page for the footer

\pagestyle{fancy}
\lhead{Math 114 (HW7)}
\chead{Michael Knopf (24457981)}
\rhead{April $2^\text{nd}$, 2015}
\lfoot{}
\cfoot{}
\rfoot{}
%\rfoot{Page\ \thepage\ of\ \pageref{LastPage}}
\renewcommand\headrulewidth{0.4pt}
%\renewcommand\footrulewidth{0.4pt}

\begin{document}

In class we defined (left) $R$-modules (p. 337); the exercises below will also require the definition of submodules of a left $R$-module (bottom of p. 337).  In all the exercises, $R$ denotes a ring with identity and $M$ denotes a left $R$-module (which satisfies the condition (d) in the definition on page 337: $1_R \cdot m = m$ for every $m \in M$).

\begin{enumerate}

\item (Exercise 1 in DF \S 10.1.) Prove that $0m = 0$ and $(-1)m=-m$ for all $m \in M$.

\begin{proof}
$$0m = (0+0)m = 0m + 0m,$$ therefore $0m$ is the additive identity. (In any group, if $gh = g$ for some $g$ and some $h$, then $h$ is the identity.  Letting $g = h = 0m$ shows that $0m$ must be the identity in the additive group $M$.)
$$
m + (-1)m = 1m + (-1)m = (1-1)m = 0m = 0,
$$
therefore $(-1)m$ is the unique additive inverse $-m$ of $m$ in $M$.
\end{proof}

\item (Exercise 4 in DF \S 10.1.) Let $M$ be the left $R$-module $R^n$ described in Example 3 on page 338.  Let $I_1,\ldots,I_n$ be left ideals of $R$.  Prove that the following are submodules of $M$:

(a) $J = \{(x_1,x_2,\ldots,x_n) \in R^n: x_1 \in I_1, x_2 \in I_2, \ldots, x_n \in I_n\}$;

(b) $Z = \{(x_1,x_2,\ldots,x_n) \in R^n: x_1+x_2+\ldots+x_n = 0\}$.

\begin{proof}
We will show that $J$ and $Z$ satisfy the submodule criterion given on pg. 342.

Let $x = (x_1, \dots , x_n), y = (y_1, \dots , y_n) \in J$, and $r \in R$.  Any left ideal of $R$ is a left $R$-module (this is part of example 1 on pg. 338), so $x_i + ry_i \in I_i$ for each $i$.  Therefore, $x + ry = (x_1 + ry_1, \dots , x_n + ry_n) \in J$.  Clearly, $J$ is nonempty since any ideal contains $0$, so $(0, \dots , 0) \in J$.  Thus $J$ is a submodule of $M$.

$Z$ is nonempty because $0 + \cdots + 0 = 0$, so $(0, \dots , 0) \in Z$.  Let $x = (x_1, \dots , x_n), y = (y_1, \dots , y_n) \in Z$, and $r \in R$.  Then $$x_1 + ry_1 + \cdots x_n + ry_n = (x_1 + \cdots + x_n) + r(y_1 + \cdots + y_n) = 0 + r \cdot 0 = 0$$
so $x+ ry \in Z$.  Thus, $Z$ is a submodule of $M$.

\noindent Note: We have used the fact here that $r \cdot 0 = 0$ for all $r \in R$.  To prove this, assume that $r \cdot 0 = m \in M$.  Then $r \cdot 0 = r \cdot (m-m) = r\cdot m + (-r)\cdot m = r \cdot m - r \cdot m = 0$.
\end{proof}

\item (Exercise 7 in DF \S 10.1.) Let $N_1, N_2, \ldots$ be an infinite sequence of submodules of $M$ with $N_i \subset N_{i+1}$ for each $i \in \mathbf{N}$ (this is called an \emph{ascending chain} of submodules of $M$).  Prove that $\cup_{i=1}^\infty N_i$ is a submodule of $M$.

\begin{proof}
Since $N_1$ is a submodule of $M$, it contains $0$.  Thus $0 \in \cup_{i=1}^\infty N_i$, so this subset is nonempty.

Let $x,y \in \cup_{i=1}^\infty N_i$ and $r \in R$.  Then there must be some $i,j$ such that $x \in N_i$ and $y \in N_j$.  Assume WLOG that $i \geq j$.  Then $N_j \subset N_i$, thus $y \in N_i$ as well.  So $x,y \in N_i$, thus $x + ry \in N_i \subset \cup_{i=1}^\infty N_i$ because $N_i$ is a submodule.  So $\cup_{i=1}^\infty N_i$ satisfies the submodule criterion.
\end{proof}

\item (Exercise 8 in DF \S 10.1.) An element $m$ of the $R$-module $M$ is called a \emph{torsion element} if $rm=0$ for some nonzero element $r \in R$.  The set of torsion elements is denoted \[ \Tor(M) = \{m \in M: rm=0 \text{ for some nonzero } r \in R\}  \text{.}\]

(a) Prove that if $R$ is an integral domain, then $\Tor(M)$ is a submodule of $M$ (called the \emph{torsion submodule} of $M$).  (Note that integral domains are assumed to be commutative.)

\begin{proof}
Suppose that $R$ is an integral domain.  Let $r \in R$ and $x,y \in \Tor(M)$, so that there exist some nonzero $a,b \in R$ such that $ax = by = 0$.  Then $ab$ is nonzero because $R$ is an integral domain, and $$
ab(x+ry) = (ab)x + (abr)y = (ba)x + (arb)y = b(ax) + ar(by) = b(0) + ar(0) = 0 + 0 = 0.
$$
Also, $\Tor(M)$ is nonempty since $1 \cdot 0 = 0$, and we are assuming that $1 \neq 0$ (as part of the definition of an integral domain).  So $\Tor(M)$ satisfies the submodule criterion.
\end{proof}

(b) Give an example of a ring $R$ and an $R$-module $M$ such that $\Tor(M)$ is not a submodule.  [Hint: Consider the torsion elements in the $R$-module $R$.]

\begin{proof}
Let $R = \Z / 6\Z$ and let $M$ be $R$ as a module over itself.  Then $\Tor(M) = \{ 0, 2, 3, 4 \}$ since $0 = 1\cdot 0 = 3 \cdot 2 = 2 \cdot 3 = 3 \cdot 4$, but $1$ and $5$ are relatively prime to $6$ and so are not zero divisors.  However, $2 + 3 = 5$ is not a torsion element, so $\Tor(M)$ is not closed under addition.  Thus, it is not a submodule of $M$.
\end{proof}

(c) If $R$ has zero divisors, show that every nonzero $R$-module has nonzero torsion elements.

\begin{proof}
Assume that $R$ has zero divisors, i.e. that there exist nonzero elements $a,b \in R$ such that $ab = 0$.  Let $M$ be a nonzero $R$-module, and suppose $m \in M$ is nonzero.  If $bm = 0$, then $m$ is a torsion element since $b$ is nonzero.  So assume that $bm \neq 0$.  Then $a(bm) = (ab)m = 0m = 0$, thus $bm$ is a nonzero torsion element because $a$ is nonzero.
\end{proof}

\item (Exercise 11 in DF \S 10.1.) Let $M$ be the abelian group (i.e., $\mathbf{Z}$-module---see the Example on p. 339) $\mathbf{Z}/24\mathbf{Z} \times \mathbf{Z}/15 \mathbf{Z} \times \mathbf{Z}/50\mathbf{Z}$.

(a) Find the annihilator of $M$ in $\mathbf{Z}$.  (The annihilator of $M$ is defined to be the subset $\{a \in \mathbf{Z}: ax=0 \text{ for all } x \in M\}$.  By Exercise 9 in \S 10.1 (which you may take for granted), the annihilator of $M$ is an ideal of $\mathbf{Z}$; find a generator for it.)

\begin{proof}
The annihilator of $M$ in $\Z$ is $600\Z$, which is the ideal generated by $600$ in $\Z$ (note that 600 is the least common multiple of 24, 15, and 50).

First, check that $600\Z$ annihilates $M$.  Any element of $600\Z$ is of the form $600z$ for some integer $z$.  If $(a,b,c)$ is any element of $M$, then $600z(a,b,c) = (24\cdot 25a, 15 \cdot 40b, 50 \cdot 12c) = (0,0,0)$.

Next, check that $600\Z$ contains the annihilator of $M$ in $\Z$.  Suppose that $r \not \in 600\Z$, so that $600 \nmid r$.  We know that $600$ divides any common multiple of $24, 15$, and $50$, so one of these numbers does not divide $r$.  Now, $(1,1,1) \in M$, but $r(1,1,1) = (r,r,r) \neq 0$ (if $24 \nmid r$ then the first component is nonzero, if $15 \nmid r$ then the second component is nonzero, and if $50 \nmid r$ then the third component is nonzero; we know that at least one of these is the case).  So $r$ is not in the anihilator of $M$ in $\Z$, thus $600\Z$ contains the annihilator of $M$ in $\Z$.
\end{proof}

(b) Let $I = 2 \mathbf{Z}$.  Describe the annihilator of $I$ in $M$ as a direct product of cyclic groups.  (The annihilator of an ideal $I \subset \mathbf{Z}$ is defined to be the subset $\{x \in M: ax=0 \text{ for all } a \in I\}$.  By Exercise 10 in \S 10.1 (which you may take for granted), the annihilator of $I$ is a submodule of $M$.)

\begin{proof}
Let $x = (a,b,c) \in M$.  $rx = 0$ for all $r \in I$  if and only if $24 \mid ra$, $15 \mid rb$, and $50 \mid rc$ for all $r \in 2\Z$, or equivalently $24 \mid 2sa$, $15 \mid 2sb$, and $50 \mid 2sc$ for all $s \in \Z$.

$24 \mid 2sa$ for all $s \in \Z$ if and only if $a = 0$ or $a = 12$.
$15 \mid 2sb$ for all $s \in \Z$ if and only if $b = 0$.
$50 \mid 2sc$ if and only if $c = 0$ or $c = 25$.
This means that the annihilator of $I$ in $M$ is the submodule
$$
\{0,12\} \times \{0\} \times \{0,25\}.
$$
Each of these groups are cyclic (each is isomorphic to either $\Z / 2\Z$ or the trivial group), so this is a direct product of cyclic groups.
\end{proof}

\item (Exercise 15 in DF \S 10.1.)  If $M$ is a finite abelian group then $M$ is naturally a $\mathbf{Z}$-module (cf. Example on p. 339).  Can this action of $\mathbf{Z}$ on $M$ be extended to make $M$ into a $\mathbf{Q}$-module?  (Prove your answer.)

\noindent \emph{The action of $\Z$ on $M$ can be extended to make $M$ into a $\Q$-module if and only if $M$ is the trivial group.}

\begin{proof}
Suppose $M = \{0\}$ is the trivial group.  Then its structure as a $\Z$-module is defined by $r \cdot 0 = 0$ for all $r \in \Z$.  This extends to a $\Q$-module by the definition $r\cdot x = 0$ for all $r \in \Q$.  To check that both of these definitions satisfy the axioms of a module, let $r,s \in \Q$:
\begin{itemize}
\item $(r + s)0 = 0 = 0 + 0 = r(0) + s(0)$
\item $(rs)0 = 0 = r(0) = r(s(0))$
\item $r(0+0) = r(0) = 0 = 0+0 = r(0) + s(0)$
\item $1(0) = 0$
\end{itemize}

Now, suppose that $M$ is not the trivial group.  Then $M$ has some finite order $n > 1$, and it contains some non-identity element $x$.  Denote the identity element of $M$ by $0$.

$M$ is naturally a $\Z$-module, defined by some action $r\cdot m = rm$ for all $r \in \Z$, $m \in M$.  Assume, for a contradiction, that this action can be extended to an action of $\Q$ on $M$, defined by $r\cdot m = rm$ for all $r \in \Q$.  This requires that $\dfrac{1}{n}x = y$ for some $y \in M$.  But then we have
$$
x = \left( n \frac1n \right) x = n \left( \frac1n x \right) = ny = 0
$$
(writing $M$ multiplicatively instead, we have that $ny = y^n = 0$ because $M$ has order $n$).  This contradicts our assumption that $x$ was not the identity element, thus such an extension does not exist when $M$ is nontrivial.
\end{proof}

\item (Exercise 21 in DF \S 10.1.) Let $F$ be a field, and let $R = M_n(F)$ be the ring of $n \times n$ matrices with entries in $F$, where $n \in \mathbf{Z}_{\geq 2}$.  Let $M$ denote the set of matrices with arbitrary elements of $F$ in the first column and zeros everywhere else; that is, 
\[
M = \{(c_{i,j}) \in R: c_{i,j}=0 \hspace{.05in} \forall \hspace{.05in} j \neq 1 \} \text{.}
\]
Show that $M$ is a submodule of $R$ when $R$ is considered as a left $R$-module, but $M$ is not a submodule of $R$ when $R$ is considered as a right $R$-module.

(Note: an abelian group $M$ has the structure of a right $R$-module if there is a map $R \times M \rightarrow M$, usually denoted $(r,m) \mapsto mr$, such that (a), (c), and (d) from the definition on page 337 hold, while (b) is replaced by the condition $m(rs)=(mr)s$.  (If the map $R \times M \rightarrow M$ is written as $(r,m) \mapsto rm$, how should condition (b) be written?)  In particular, just as the abelian group $M=R$ forms a left $R$-module via the map $R \times M \rightarrow M$ defined by $(r,m) \mapsto rm$ (where the operation in the expression $rm$ is multiplication in $R$), it forms a right $R$-module via the map $R \times M \rightarrow M$ defined by $(r,m) \mapsto mr$.)

\begin{proof}
Let $B \in M$ and $A \in R$.  The $ij$th entry of $AB$ is the Euclidean inner product of the $i$th row of $A$ with the $j$th column of $B$.  Since the $j$th column of $B$ is $0$ for all $j > 1$, this means that the $ij$th entry of $AB$ is $0$ whenever $j > 1$, meaning that the only nonzero elements of $AB$ are in the first column (where $j=1$).  Thus, $AB \in M$.

It is clear that $M$ is closed under addition.  If $A,B \in M$, then $a_{ij} = 0$ and $b_{ij} = 0$ whenever $j > 1$, so $a_{ij} + b_{ij} = 0$ whenever $j > 1$.  It is also clear that $M$ is nonempty, since it contains the $0$ matrix.  Therefore, $M$ is a left submodule of $R$.

When $R$ is considered as a right $R$-module, $M$ is not a submodule because it is not closed under right multiplication by elements of $R$.  Consider the matrices $A \in M$ and $B \in R$ where $A$ has zeros in every entry except the top-left entry $a_{11}$, and $B$ has zeros in every entry except the top-right entry $b_{1n}$.  Then the first row of $A$ is $\begin{pmatrix}
1 & 0 & \cdots & 0
\end{pmatrix}$
and the $n$th column of $B$ is
$\begin{pmatrix}
1 & 0 & \cdots & 0
\end{pmatrix}^T$.  Therefore, the $1n$ entry of $AB$ is
$$\begin{pmatrix}
1 & 0 & \cdots & 0
\end{pmatrix}
\begin{pmatrix}
1 & 0 & \cdots & 0
\end{pmatrix}^T
= 1,$$
so $AB \not \in M$ (since $n \geq 2$).  Thus $M$ is not a right submodule of $R$.
\end{proof}

\end{enumerate}
\end{document}
