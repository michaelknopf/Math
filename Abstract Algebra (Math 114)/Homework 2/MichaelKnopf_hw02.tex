\documentclass[10pt]{article}

\usepackage[margin=.75in]{geometry} 
\usepackage{amsmath,amsthm,amssymb}
\usepackage{bm}
\usepackage{enumitem}
\usepackage{array}
\usepackage{lipsum}

\setenumerate{listparindent=\parindent}

\newcommand{\Q}{\mathbb{Q}}
\newcommand{\Z}{\mathbb{Z}}

\DeclareMathOperator*{\dom}{dom}
\DeclareMathOperator*{\Aut}{Aut}

\def\slfrac#1#2{{\mathord{\mathchoice   % 
        {\kern.1em\raise.5ex\hbox{$\scriptstyle#1$}\kern-.1em
        /\kern-.15em\lower.25ex\hbox{$\scriptstyle#2$}}
        {\kern.1em\raise.5ex\hbox{$\scriptstyle#1$}\kern-.1em
        /\kern-.15em\lower.25ex\hbox{$\scriptstyle#2$}}
        {\kern.1em\raise.4ex\hbox{$\scriptscriptstyle#1$}\kern-.1em
        /\kern-.14em\lower.25ex\hbox{$\scriptscriptstyle#2$}}
        {\kern.1em\raise.2ex\hbox{$\scriptscriptstyle#1$}\kern-.1em
        /\kern-.1em\lower.25ex\hbox{$\scriptscriptstyle#2$}}}}}



\begin{document}

\begin{center}
\large Math 114 Homework 2

\normalsize Michael Knopf

\normalsize (due Thursday, 5 February)
\end{center}

\begin{enumerate}[leftmargin=0cm,itemindent=.5cm,labelwidth=\itemindent,labelsep=0cm,align=left]

\item (Exercise 2 in DF \S 13.2.) Let $g(x)=x^2+x-1$ and $h(x)=x^3-x+1$.  Obtain fields of $4$, $8$, $9$, and $27$ elements by adjoining a root of $f(x)$ to the field $F$ where $f(x)$ equals $g(x)$ or $h(x)$ and $F$ equals $\mathbf{F}_2$ or $\mathbf{F}_3$.  Write down the multiplication tables for the fields with four and nine elements and show that the nonzero elements form a cyclic group.

\begin{proof}
\ If $g(x)$ or $h(x)$ were reducible over either $\mathbb{F}_2$ or $\mathbb{F}_3$, then they would have a linear factor, thus a root in that field.  However, in $\mathbb{F}_2$, $g(0) = g(1) = h(0) = h(1) = 1$; and in $\mathbb{F}_3$, $g(0) = h(0) = g(1) = h(1) = h(2) = 1$.  So both polynomials are irreducible over both fields.

Let $\alpha$ and $\beta$ be roots of $g(x)$ in 
some extensions of $\mathbb{F}_2$ and $\mathbb{F}_3$, respectively.  The multiplication tables for the extensions $\mathbb{F}_2(\alpha)$ and $\mathbb{F}_2(\beta)$ are given below.  They have $4$ and $9$ elements, respectively.  This is not suprising, since $g(x)$ is irreducible over both $\mathbb{F}_2$ and $\mathbb{F}_3$, so both extensions are of degree $2$.  A vector space of dimension $n$ over a finite field with $p$ elements has cardinality $p^n$ since, for each of the $n$ basis vectors, we have $p$ choices for its coefficient.

\begin{center}

%\begin{tabular}{>{$}l<{$}|*{6}{>{$}l<{$}}}
\begin{tabular}{c|cccc}
$\mathbb{F}_2(\alpha)$  & 0 & 1 & $\alpha$ & $1+\alpha$ \\
\hline\vrule height 12pt width 0pt
0 & 0 & 0 & 0 & 0 \\ 
1 & 0 & 1 & $\alpha$ & $1+\alpha$ \\ 
$\alpha$ & 0 & $\alpha$ & $1+\alpha$ & 1 \\ 
$1+\alpha$ & 0 & $1+\alpha$ & 1 & $\alpha$ \\ 
\end{tabular}

\begin{tabular}{c|ccccccccc}
$\mathbb{F}_3(\beta)$ & 0 & 1 & 2 & $\beta$ & $2\beta$ & $1+\beta$ & $1+2\beta$ & $2+\beta$ & $2+2\beta$ \\ 
\hline\vrule height 12pt width 0pt
0 & 0 & 0 & 0 & 0 & 0 & 0 & 0 & 0 & 0 \\ 
1 & 0 & 1 & 2 & $\beta$ & $2\beta$ & $1+\beta$ & $1+2\beta$ & $2+\beta$ & $2+2\beta$ \\
2 & 0 & 2 & 1 & $2\beta$ & $\beta$ & $2+2\beta$ & $2+\beta$ & $1+2\beta$ & $1+\beta$ \\ 
$\beta$ & 0 & $\beta$ & $2\beta$ & $1+2\beta$ & $2+\beta$ & 1 & $2+2\beta$ & $1+\beta$ & 2 \\ 
$2\beta$ & 0 & $2\beta$ & $\beta$ & $2+\beta$ & $1+2\beta$ & 2 & $1+\beta$ & $2+2\beta$ & 1 \\ 
$1+\beta$ & 0 & $1+\beta$ & $2+2\beta$ & 1 & 2 & $2+\beta$ & $\beta$ & $2\beta$ & $1+2\beta$ \\ 
$1+2\beta$ & 0 & $1+2\beta$ & $2+\beta$ & $2+2\beta$ & $1+\beta$ & $\beta$ & 2 & 1 & $2\beta$ \\ 
$2+2\beta$ & 0 & $2+2\beta$ & $1+\beta$ & 2 & 1 & $1+2\beta$ & 2 & $\beta$ & $2+\beta$ \\ 
\end{tabular} 

\end{center}

The nonzero elements of a field always form a multiplicative group.  For a finite field, this group is always cyclic.  In particular, the nonzero elements of $\mathbb{F}_2(\alpha)$ form a cyclic group generated by $\alpha$, since $\alpha^2 = 1 + \alpha$.  The nonzero elements of $\mathbb{F}_3(\beta)$ form a cyclic group generated by $\beta$: since $\beta^3 = 2+2\beta \neq 1$, we know $\beta$ has order greater than $3$, thus it must have order $9$ (by Lagrange's Theorem).

Let $\gamma$ and $\delta$ be roots of $h(x)$ in $\mathbb{F}_2$ and $\mathbb{F}_3$, respectively.  Since $h(x)$ is irreducible over both fields, these elements both have degree $3$.  So the number of elements in $\mathbb{F}_2(\delta)$ and in $\mathbb{F}_3(\gamma)$ are $2^3 = 8$ and $3^3 = 27$, respectively.

\end{proof}

\item (Exercise 13 in DF \S 13.2.) Suppose $F = \mathbf{Q}(\alpha_1,\alpha_2,\ldots,\alpha_n)$ where $\alpha_i^2 \in \mathbf{Q}$ for $i = 1, 2, \ldots, n$.  Prove that $\sqrt[3]{2} \notin F$.

\begin{proof}

\ Let $F_0 = \Q$ and, for each $i = 1,2,\dots,n$, let $F_{i} = F_{i-1}(\alpha_i)$.  By previous theorems, $F_0 \subseteq F_1 \subseteq \cdots \subseteq F_n$ and $F_i = \Q(\alpha_1, \dots , \alpha_i)$ for all $i$.

If $\alpha_i \in F_{i-1}$, then $\alpha_i$ is the root of a linear polynomial over $F_{i-1}$, thus its minimal polynomial and hence, the degree of $F_i = F(\alpha_i)$ over $F_i$, is 1.  Otherwise, it is a root of the degree 2 polynomial $x^2 - \alpha_i^2 \in \Q[x] \subseteq F_{i-1}$ (since $\alpha_i^2 \in \Q$).  Since it is not the root of a linear polynomial, $x^2 - \alpha_i^2$ is its minimal polynomial, thus $F_i = F(\alpha_i)$ has degree 2 over $F_{i-1}$.  Therefore, $[F:\Q] = [F_n : F_{n-1}][F_{n-1}:F_{n-2}] \cdots [F_1 : F_0] = 2^k$ for some integer $k$.

We know that $\sqrt[3]{2}$ has degree $3$ over $\Q$, since $x^3 - 2$ is irreducible over $\Q$ by Eisenstein's Criterion ($2 \mid 2$, but $2 \nmid 1$ and $2^2 \nmid 2$).  Assume, for a contradiction, that $\sqrt[3]{2} \in F$.  Then $\Q \subset F(\sqrt[3]{2}) \subseteq F$, thus $[F(\sqrt[3]{2}) : \Q] = 3$ divides $[F:\Q] = 2^k$, a contradiction.

\end{proof}

\pagebreak

\item (Exercise 14 in DF \S 13.2.) Let $E/F$ be an extension and $\alpha \in E$ an element algebraic over $F$.  Prove that if $[F(\alpha):F]$ is odd, then $F(\alpha) = F(\alpha^2)$.

\begin{proof}

\ Let the degree of $F(\alpha)$ over $F$ be $2n+1$.  The set $\{ 1, \alpha, \alpha^2, \dots , \alpha^{2n+1} \}$ forms a basis for $F(\alpha)$ over $F$.  Thus, the set $\{1, \alpha^2, \alpha^4, \dots, \alpha^{2n} \}$ is also linearly independent over $F$.

Since $\alpha^2 \in F(\alpha)$, we know that $F(\alpha^2) \subseteq F(\alpha)$.  So the degree of $F(\alpha^2)$ over $F$ must divide that of $F(\alpha)$ over $F$, which is odd.  Odd numbers have no even divisors, so $[F(\alpha^2) : F]$ is odd.  Therefore, $\{1, \alpha^2, \alpha^4, \dots, \alpha^{2n} \}$ does not span $F(\alpha^2)$, since $2n$ is even.

Thus, there must be some other vector $\alpha^k$, from the given basis for $F(\alpha)$, which is in $F(\alpha)^2$ (this follows from the ``Replacement Theorem" of linear algebra).  Since all even powers of $\alpha$ from that basis are already represented in $\{1, \alpha^2, \alpha^4, \dots, \alpha^{2n} \}$, we know that $k$ is odd.  This also implies $k-1$ is even, thus $\alpha^{k-1} \in F(\alpha^2)$.  So $\dfrac{\alpha^k}{\alpha^{k-1}} = \alpha \in F(\alpha^2)$, which gives that $F(\alpha) \subseteq F(\alpha^2)$.  Clearly, $F(\alpha^2) \subseteq F(\alpha)$.  So $F(\alpha) = F(\alpha^2)$.

\end{proof}

\item (Exercise 16 in DF \S 13.2.) Let $K/F$ be an algebraic extension and let $R$ be a subring of $K$ containing $F$.  Show that $R$ is a subfield of $K$ containing $F$.

\begin{proof}

\ Since $R$ is a subring of $K$, all we need to show is that it is closed under multiplicative inverses.  Let $\alpha$ be a nonzero element of $R$.  We know $\alpha$ is algebraic over $F$ because $R$ is contained within an algebraic extension of $F$.  So $\alpha$ has a minimal polynomial $p(x) = x^n + c_{n-1}x^{n-1} + \cdots + c_1 x + c_0 \in F[x]$.  If $c_0 = 0$, then either $p(x) = x$, implying that $\alpha = 0$ (which we have assumed is not the case), or $p(x) = xq(x)$ for some nonconstant polynomial $q(x) \in F[x]$, contradicting that $p(x)$ is irreducible.  So $\frac{1}{c_0} \in F$.

We have $$\alpha^n + c_{n-1} \alpha^{n-1} + \cdots + c_1 \alpha + c_0 = 0$$
$$\implies \alpha (c_n \alpha^{n-1} + \cdots + \alpha c_2 + c_1) = -c_0$$
$$\implies \frac{1}{\alpha} = - \frac{1}{c_0}(c_n \alpha^{n-1} + \cdots + \alpha c_2 + c_1).$$
Since $\frac{1}{c_0}, c_1, \dots, c_n \in F \subseteq R$ and $\alpha \in R$, the right hand side is an element of $R$, thus $\frac{1}{\alpha} \in R$.



\end{proof}

\item (Exercise 17 in DF \S 13.2.) Let $f(x)$ be an irreducible polynomial of degree $n$ over a field $F$.  Let $g(x)$ be any polynomial in $F[x]$.  Prove that every irreducible factor of the composite polynomial $f(g(x))$ has degree divisible by $n$.

\begin{proof}

\ Let $h(x) \in F[x]$ be an irreducible factor of $f(g(x))$ in $F[x]$.  Since $f(x)$ and $h(x)$ are both irreducible, the degrees of the extensions $\dfrac{F[x]}{\big(f(x) \big)}$ and $\dfrac{F[x]}{\big(h(x) \big)}$ over $F$ are the respective degrees of $f(x)$ and $h(x)$ as polynomials.  Therefore, it suffices to show that $\dfrac{F[x]}{\big(f(x) \big)} \subseteq \dfrac{F[x]}{\big(h(x) \big)}$, since this will imply that $n \mid \left[ \dfrac{F[x]}{\big(f(x) \big)} : F \right] \left[ \dfrac{F[x]}{\big(h(x) \big)}:\dfrac{F[x]}{\big(f(x) \big)} \right] = \left[ \dfrac{F[x]}{\big(h(x) \big)} : F \right] = \deg(h(x))$.  This only requires finding an isomorphic copy of $\dfrac{F[x]}{\big(f(x) \big)}$ within $\dfrac{F[x]}{\big(h(x) \big)}$.

Define a map $\varphi : \dfrac{F[x]}{\big(f(x) \big)} \rightarrow \dfrac{F[x]}{\big(h(x) \big)}$ by $\overline{p(x)} \mapsto \overline{p(g(x))}$.  To show this map is well-defined, assume $p(x) \in \big( f(x) \big)$, so that $p(x) = r(x)f(x)$ for some $r(x) \in F[x]$.  Thus, $p(g(x)) = r(g(x))f(g(x)) \in \big( h(x) \big)$ because $h(x) \mid f(g(x))$.

It is clear that $\varphi$ is a ring homomorphism: for any $p(x), q(x), r(x) \in F[x]$ we have $\varphi(p(x)q(x) + r(x)) = p(g(x))q(g(x)) + r(g(x)) = \varphi(p(x))\varphi(q(x)) + \varphi(r(x))$.

Now, assume for a contradiction that $\varphi$ is the zero map.  Then for the polynomial $p(x) = 1$, we have $\varphi(\overline{p(x)}) = \overline{p(g(x))} = \overline{1} = \big( h(x) \big)$.  So $h(x)$ is a unit in $F[x]$ and thus a constant polynomial, contraditing that it is irreducible.  So $\varphi$ is injective, and thus an isomorphism onto a subfield of $\dfrac{F[x]}{\big(h(x) \big)}$.  By the argument in the first paragraph, this completes the proof.

\end{proof}

\item (Exercise 2 in DF \S 13.4.) Find a splitting field $K$ for $X^4+2$ over $\mathbf{Q}$, and determine $[K:\mathbf{Q}]$.

\begin{proof}

\ We can form the splitting field $K$ by adjoining all of the complex fourth roots of $-2$ to $\Q$.  One of these is $\sqrt[4]{2} e^{i\pi/4} = \dfrac{1+i}{\sqrt[4]{2}}$.  The others are $ -\dfrac{1+i}{\sqrt[4]{2}}$ and $\pm i\dfrac{1+i}{\sqrt[4]{2}} = \pm \dfrac{-1+i}{\sqrt[4]{2}}$, where $\sqrt[4]{2}$ is taken to be the positve, real fourth root of $2$.  However, $i$ can be formed by dividing two of these roots, and similarly any of the roots can be formed by multiplying $\dfrac{1+i}{\sqrt[4]{2}}$ by $-1$ or $\pm i$.  Therefore, $K = \Q \left(\dfrac{1+i}{\sqrt[4]{2}}, i \right)$.

Since $\dfrac{1+i}{\sqrt[4]{2}}$ is a root of the irreducible polynomial $x^4 + 2$ (again, this can be shown to be irreducible by applying Eisenstein's criterion with $p = 2$), we know that $\left[ \Q \left( \dfrac{1+i}{\sqrt[4]{2}} \right) : \Q \right] = 4$.  If we can find the degree of $K$ over $\Q \left( \dfrac{1+i}{\sqrt[4]{2}} \right)$, we can multiply it by $4$ to obtain the degree of $K$ over $\Q$.

Suppose, for a contradiction, that $i \in \Q \left( \dfrac{1+i}{\sqrt[4]{2}} \right) = \text{span} \left\{ 1, \dfrac{1+i}{\sqrt[4]{2}}, \left( \dfrac{1+i}{\sqrt[4]{2}} \right)^2, \left( \dfrac{1+i}{\sqrt[4]{2}} \right)^3 \right\}$ %= \\ \text{span} \left\{ 1, \dfrac{1+i}{\sqrt[4]{2}} , \sqrt{2}i, \sqrt[4]{2}(-1+i)  \right\}$
(over $\Q$).  Then there exist rational numbers $a,b,c$, and $d$ such that
\begin{align*}
i &= a + b \left( \dfrac{1+i}{\sqrt[4]{2}} \right) + c \left( \dfrac{1+i}{\sqrt[4]{2}} \right)^2 + d \left( \dfrac{1+i}{\sqrt[4]{2}} \right)^3
\\
&= a + b \left( \dfrac{1+i}{\sqrt[4]{2}} \right) + c \sqrt{2}i + d \sqrt[4]{2}(-1+i).
\\
&= a + b\frac{1}{\sqrt[4]{2}} + b\frac{1}{\sqrt[4]{2}}i + c\sqrt{2}i - d\sqrt[4]{2} + d\sqrt[4]{2}i
\\
&= \left( a + b\frac{1}{\sqrt[4]{2}} - d\sqrt[4]{2} \right) + \left( b\frac{1}{\sqrt[4]{2}} + c\sqrt{2} + d\sqrt[4]{2} \right)i
\end{align*}

\noindent Since $1$ and $i$ are linearly independent over $\mathbb{R}$, the real part must equal $0$ and the imaginary part must equal 1.  Multiplying $a + b\frac{1}{\sqrt[4]{2}} - d\sqrt[4]{2} = 0$ through by $\sqrt[4]{2}$ gives $$b + 2^{\frac{1}{4}} a - 2^{\frac{2}{4}} d = 0.$$  Since $x^4 - 2$ is irreducible over $\Q$, and $2^{\frac{1}{4}}$ is a root of this polynomial, we know that $\{1, 2^{\frac{1}{4}}, 2^{\frac{2}{4}} \}$ is linearly independent over $\Q$.  Therefore, $a = b = d = 0$.  Setting the imaginary part equal to $1$ and substiting $0$ for $b$ and $d$ gives $c\sqrt{2} = 1$, a contradiction because $\dfrac{1}{\sqrt{2}}$ is not rational.

Therefore, the degree of $K = \Q \left( \dfrac{1+i}{\sqrt[4]{2}}, i \right)$ over $\Q \left( \dfrac{1+i}{\sqrt[4]{2}} \right)$ is greater than 1.  However, its degree does not exceed $2$ since $\left[ \Q \left( \dfrac{1+i}{\sqrt[4]{2}} \right) \left( i \right) : \Q \left( \dfrac{1+i}{\sqrt[4]{2}} \right) \right] \leq \left[ \Q(i) : \Q \right] = 2$.  Therefore, the degree of $K$ over $\Q \left( \dfrac{1+i}{\sqrt[4]{2}} \right)$ is $2$.  So $$[K : \Q] = \left[ K : \Q \left( \dfrac{1+i}{\sqrt[4]{2}} \right) \right] \left[ \Q \left( \dfrac{1+i}{\sqrt[4]{2}} \right) : \Q \right] = 4 \cdot 2 = 8.$$

\end{proof}

\item (Exercise 3 in DF \S 13.4.) Find a splitting field $K$ for $X^4+X^2+1$ over $\mathbf{Q}$, and determine $[K:\mathbf{Q}]$.

\begin{proof}

\ Descarte's rule of signs reveals that this polynomial has no real roots.  However, we can apply the quadratic formula to find that $x^2 = \dfrac{-1 \pm \sqrt{-3}}{2} = \zeta_3, \zeta_3^2$.  The solutions to this equation give the set of roots of $x^4 + x^2 + 1$: $\{ \zeta_6, \zeta_6^2, \zeta_6^4, \zeta_6^5  \} = \left\{ \dfrac{\pm 1 \pm \sqrt{-3}}{2} \right\}$.  We now see that
%$\left(x + \dfrac{1 + \sqrt{-3}}{2}\right)\left(x + \dfrac{1 - \sqrt{-3}}{2}\right)\left(x - \dfrac{1 + \sqrt{-3}}{2}\right)\left(x - \dfrac{1 - \sqrt{-3}}{2}\right) = \left(x^2 + x + 1\right)\left(x^2 - x + 1\right) = x^4 + x^2 + 1$
$x^4 + x^2 + 1 = (x-\zeta_6)(x-\zeta_6^5)(x-\zeta_6^4)(x-\zeta_6^2) = (x - \zeta_6)(x - \overline{\zeta_6})(x + \zeta_6)(x + \overline{\zeta_6}) =  \left(x^2 + x + 1\right)\left(x^2 - x + 1\right)$
is reducible, although both of these quadratic factors are irreducible.

The splitting field $K$ must be $\Q(\zeta_6)$, since $K$ must contain $\zeta_6$, but the other roots are powers of this element.  The minimal polynomial for $K$ is $x^2 + x - 1$, since its roots are $\zeta_6$ and $\zeta_6^5$, neither of which are rational (so this degree 2 polynomial has no linear factors, and is thus irreducible over $\Q$).  Thus $K$ has degree 2 over $\Q$.

\end{proof}

\end{enumerate}
\end{document}









